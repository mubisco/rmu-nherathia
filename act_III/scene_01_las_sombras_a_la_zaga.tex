\section{Escena I --- Las Sombras a la Zaga}

\begin{multicols*}{2}

\textbf{Tipo de escena:} Supervivencia, Acecho (Stalking) y posible Combate Táctico.

\begin{readaloud}
El aire ya no huele a marismas ni a ceniza. Huele a hielo y a piedra antigua. Habéis dejado atrás las colinas de Valmor para adentraros en las estribaciones de las \textbf{Costillas del Norte}.

La niebla aquí es diferente. Es más fina, cristalina y letalmente fría. Vuestro aliento forma nubes densas frente a vosotros. El mapa que os dio Erun os guía hacia un paso estrecho entre dos picos que parecen colmillos rotos.

Pero no estáis solos.

Desde hace horas, sentís una presión en la nuca. Al mirar atrás, hacia el valle cubierto por el mar de nubes del Velo, no veis nada... todavía. Pero el viento trae un sonido metálico, como cadenas arrastradas sobre roca, y jadeos que no son humanos. La \textbf{Vigilia del Silente} os está cazando.
\end{readaloud}

\subsubsection*{El Juego del Gato y el Ratón}

Estáis en un sendero de cabras, con una pared de roca a un lado y una caída de 50 metros al otro. El terreno ofrece rocas grandes y recodos para ocultarse, pero la nieve cruje bajo las botas.

\textbf{Fase 1: La Detección (Maniobra Enfrentada)}

Los cazadores intentan acercarse sigilosamente para asegurar el tiro antes de soltar a los perros.

\begin{mechanicbox}{Percepción vs Acechar}
El GM debe tirar por el grupo de cazadores: \textbf{Acechar (Stalking) +60}.
Los jugadores tiran \textbf{Percepción (Alertness)}.

\begin{itemize}[nosep]
    \item \textbf{Éxito de los PJ (Percepción > Acechar):} Veis un destello de metal en la cresta superior o notáis que el viento cambia trayendo el olor de los perros antes de que os vean. \textbf{Tenéis la Iniciativa.} Podéis decidir qué hacer (ver Opciones A y B).
    
    \item \textbf{Fallo de los PJ (Acechar > Percepción):} No notáis nada hasta que un virote de ballesta golpea la roca junto a vuestra cabeza y los perros cargan. \textbf{Estáis Sorprendidos} (Ver Opción C).
\end{itemize}
\end{mechanicbox}

\begin{tcolorbox}[colback=white, colframe=black!75!white, title=\textbf{Detalles del Mapa Táctico: El Desfiladero del Viento}]

\textbf{Especificaciones Generales:}
\begin{itemize}[nosep]
    \item \textbf{Dimensiones:} 10 casillas de ancho x 30 casillas de largo (15m x 45m).
    \item \textbf{Escala:} 1 casilla = 1.5 metros (5 pies).
    \item \textbf{Iluminación:} Luz Difusa (Dim Light). La niebla limita la visión clara a 20m; más allá otorga Cobertura Visual Parcial (Soft Cover).
\end{itemize}

\textbf{Elementos del Terreno:}
\begin{itemize}[nosep]
    \item \textbf{Norte (Pared de Roca):} Muro inescalable. Contiene 2 o 3 \textbf{Grietas} (1.5m de profundidad) que otorgan bonificación a \textit{Acechar} y Cobertura Total.
    \item \textbf{Sur (El Abismo):} Caída de 50m. Un personaje empujado o que sufra un Crítico de \textit{Impacto/Desequilibrio} en el borde debe superar una Maniobra de Movimiento para no caer.
    \item \textbf{Centro (El Camino):} Ancho irregular (1.5m a 3m). Superficie de hielo negro: se considera \textbf{Terreno Difícil} (las maniobras de correr requieren tirada de Acrobacias).
    \item \textbf{La Curva Ciega (Este):} A 10m de la entrada enemiga, el camino gira tras una roca, bloqueando la Línea de Visión (LOS) inicial.
    \item \textbf{Obstáculos:} Rocas Grandes (Cobertura Dura Parcial) y Montículos de Nieve (Terreno Difícil, otorgan ocultación si el personaje está tumbado).
\end{itemize}

\end{tcolorbox}

\subsubsection*{Decisiones (Si ganaron la Percepción)}

Si los detectáis primero, tenéis unos segundos vitales antes de que doblen la esquina.

\paragraph{Opción A: La Emboscada (Combate Táctico)}
Os ocultáis tras las rocas y esperáis a que pasen para atacar.
\begin{itemize}[nosep]
    \item \textbf{Ventaja:} Ataque Sorpresa (+20 OB) y posibilidad de usar la habilidad \textbf{Emboscada (Ambush)}.
    \item \textbf{Objetivo:} Eliminar al Vigía antes de que toque el cuerno.
\end{itemize}

\paragraph{Opción B: Esconderse (Evitar el Encuentro)}
Intentáis ocultaros en las grietas o bajo la nieve para que pasen de largo.
\begin{itemize}[nosep]
    \item \textbf{Mecánica:} Tirada de \textbf{Acechar (Stalking)} vs la Percepción de los Mastines (+50 Olfato).
    \item \textbf{Dificultad:} Muy Difícil (-20) debido al olfato de los perros. Si tenéis hierbas de olor fuerte o camináis por el agua helada, la penalización baja a Medio (0).
    \item \textbf{Resultado:} Si tenéis éxito, la patrulla pasa de largo, siguiendo un rastro falso montaña arriba. Ganáis tiempo y ahorráis recursos.
\end{itemize}

\paragraph{Opción C: Emboscados (Si fallaron la Percepción)}
Los Vigías atacan primero.
\begin{itemize}[nosep]
    \item \textbf{Efecto:} Los PJ están \textbf{Sorprendidos} (Pierden 2 Puntos de Acción en el primer asalto o no actúan si están \emph{Flatfooted/Desprevenidos}). Los Mastines cargan y atacan con ventaja (+20 al ataque).
\end{itemize}

\subsubsection*{Los Cazadores}

\textbf{Escalado del Encuentro:}
\begin{itemize}[nosep]
    \item \textbf{Vigías Rastreadores:} 1 por cada 3 Jugadores (redondeando hacia arriba).
    \item \textbf{Mastines del Velo:} 1 por cada Jugador.
\end{itemize}

\begin{npcsheet}[Cazador de Élite][images/vigia_rastreador.png]{Vigía Rastreador (Nvl 5)}
    \appearance{Armadura de cuero negro, máscara de hierro sin boca, capa de piel de lobo blanco.}
    \motivation{Localizar, fijar y marcar. No arriesgará su vida innecesariamente.}
    \stats{
        \textbf{Nivel:} 5 (Ranger) \quad \textbf{PV:} 60 \\
        \textbf{AT:} 8 (Cuero Reforzado) \quad \textbf{DB:} 30 (Reflejos/Cobertura) \\
        \textbf{OB:} +80 (Ballesta Pesada), +60 (Espada Corta) \\
        \textbf{Habilidades:} Acechar +70, Percepción +50.
    }
    \secrets{\textbf{El Cuerno de Señal:} Su prioridad al entrar en combate es gastar 1 Asalto completo en soplar el cuerno. Si lo hace, alerta a la guarnición final (Ver "Consecuencias").}
\end{npcsheet}

\begin{npcsheet}[Bestia de Presa][images/mastin_velo.png]{Mastín del Velo (Nvl 3)}
    \appearance{Perros enormes sin piel, músculo vivo y hueso expuesto. Ojos violetas.}
    \stats{
        \textbf{Nivel:} 3 \quad \textbf{PV:} 45 \quad \textbf{AT:} 3 (Piel dura) \\
        \textbf{OB:} +50 (Mordisco Grande) \\
        \textbf{Especial:} Presa (Grapple). Si muerden, intentan derribar al objetivo.
    }
\end{npcsheet}

\subsubsection*{Consecuencias del Encuentro}

El resultado de esta escena afecta directamente al final del módulo (Escena V).

\begin{itemize}[nosep]
    \item \textbf{El Cuerno Suena:} Si el Vigía sopla el cuerno, un eco grave retumba en las montañas. La fuerza principal sabe dónde estáis. \textbf{Efecto:} En la Escena V, habrá un \textbf{Capitán de la Guardia (Nvl 6)} adicional esperando en el puente.
    \item \textbf{Silencio (Muerte o Sigilo):} Si matáis al Vigía antes de que sople el cuerno o lográis esconderos, ganáis ventaja. \textbf{Efecto:} En la Escena V, tendréis un asalto de sorpresa contra los guardias del puente.
\end{itemize}

\subsubsection*{El Descubrimiento del Paso}

Tras superar el peligro, llegáis a la cima del desfiladero. Ante vosotros, incrustada en la ladera como una herida geométrica, está la entrada a las ruinas.

No parece una construcción humana. Son bloques ciclópeos de piedra negra que forman un arco triangular.

\textbf{Tirada de Saber (Lore) - Difícil (-10):}
\begin{itemize}[nosep]
    \item \textbf{Éxito:} Reconocéis la arquitectura de los \textbf{Primeros Habitantes}. Pre-humana, pre-élfica. Una civilización que adoraba el silencio antes de que existiera el Velo.
    \item \textbf{Fallo:} La estructura os marea. La geometría parece "incorrecta", como si los ángulos no sumaran lo que deberían.
\end{itemize}

La \textbf{Brújula de Orim} gira violentamente y apunta hacia la oscuridad del arco. Una corriente de aire cálido y viciado sale de las profundidades.

\textbf{Comienza la Escena II — El Umbral de los Primeros.}

\end{multicols*}

\imagerim{images/entrada_dungeon.png}
