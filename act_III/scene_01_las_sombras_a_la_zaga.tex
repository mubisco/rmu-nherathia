\section{Escena I --- Las Sombras a la Zaga}

\begin{multicols}{2}

\textbf{Tipo de escena:} Supervivencia, Acecho (Stalking) y posible Combate Táctico.

\begin{readbox}
El aire ya no huele a marismas ni a ceniza. Huele a hielo y a piedra antigua. Habéis dejado atrás las colinas de Valmor para adentraros en las estribaciones de las \textbf{Costillas del Norte}.

La niebla aquí es diferente. Es más fina, cristalina y letalmente fría. Vuestro aliento forma nubes densas frente a vosotros. El mapa que os dio Erun os guía hacia un paso estrecho entre dos picos que parecen colmillos rotos.

Pero no estáis solos.

Desde hace horas, sentís una presión en la nuca. Al mirar atrás, hacia el valle cubierto por el mar de nubes del Velo, no veis nada... pero el viento trae un sonido metálico, como cadenas arrastradas sobre roca, y jadeos que no son humanos.
\end{readbox}

\subsubsection*{El Juego del Gato y el Ratón}

Estáis en un sendero de cabras, con una pared de roca a un lado y una caída de 50 metros al otro. El terreno ofrece rocas grandes y recodos para ocultarse, pero la nieve cruje bajo las botas.

\begin{mechanics}{Fase 1: La Detección (Maniobra Enfrentada)}
Los cazadores intentan acercarse para asegurar el tiro antes de soltar a los perros.
\begin{itemize}
    \item \textbf{Cazadores:} Tiran \textbf{Acechar (Stalking) +60}.
    \item \textbf{Jugadores:} Tiran \textbf{Percepción (Alertness)}.
\end{itemize}
\textbf{Resultados:}
\begin{itemize}
    \item \textbf{Ganan los PJ:} Detectan el brillo del metal o el olor de los perros. Tienen \textbf{Iniciativa} y pueden elegir entre Emboscar (Opción A) o Esconderse (Opción B).
    \item \textbf{Ganan los Cazadores:} Los PJ son sorprendidos. Un virote golpea la roca junto a ellos. Pasan a la Opción C (Sorpresa).
\end{itemize}
\end{mechanics}

\subsubsection*{Decisiones Tácticas}

\textbf{Opción A: La Emboscada (Combate)}
Os ocultáis tras las rocas y esperáis a que pasen para atacar.
\begin{itemize}
    \item \textbf{Ventaja:} Asalto de Sorpresa (+20 OB).
    \item \textbf{Objetivo Crítico:} Eliminar al Vigía antes de que toque el cuerno.
\end{itemize}

\paragraph{Opción B: Esconderse (El Dilema Vertical)}
El camino es una trampa, pero el entorno ofrece dos vías de escape extremas para evitar el combate:

\begin{itemize}
    \item \textbf{La Vía Alta (Cornisa Superior):} Trepar 4m hacia arriba.
    \begin{itemize}
        \item \textit{Ventaja:} Posición de tiro y defensa. Los perros no pueden subir.
        \item \textit{Riesgo:} Si os ven, no tenéis dónde huir.
    \end{itemize}
    \item \textbf{La Vía Baja (El Saliente del Abismo):} Descolgarse 4m hacia el vacío, hasta una grieta en la pared del precipicio.
    \begin{itemize}
        \item \textit{Ventaja:} Ocultación total. El viento ascendente borra vuestro olor para los perros.
        \item \textit{Riesgo:} Requiere \textbf{Trepar (Difícil)}. Un fallo es una caída de 50m. Si os detectan ahí abajo, estáis en desventaja severa (ataques desde arriba).
    \end{itemize}
\end{itemize}

\paragraph{Opción C: Emboscados (Si fallaron la Percepción)}
Los Vigías atacan primero.
\begin{itemize}
    \item \textbf{Efecto:} Los PJ están \textbf{Sorprendidos}. Los Mastines cargan y atacan con ventaja (+20 al ataque).
\end{itemize}

\subsubsection*{Consecuencias del Encuentro}

El resultado define la dificultad del final del módulo (Escena V).

\begin{itemize}
    \item \textbf{El Cuerno Suena:} Si el Vigía gasta 1 asalto vivo en tocar el cuerno, la alarma general se activa. \textbf{Efecto:} En la Escena V, habrá un \textbf{Capitán de la Guardia} y refuerzos esperando.
    \item \textbf{Silencio (Muerte Rápida o Sigilo):} Si evitáis el cuerno. \textbf{Efecto:} En la Escena V, el guardia estará distraído (Sorpresa Total).
\end{itemize}

\subsubsection*{El Descubrimiento del Paso}

Tras superar el peligro, llegáis a la cima. Ante vosotros, incrustada en la ladera, está la entrada a las ruinas: bloques de piedra negra que forman un arco triangular imposible.

La \textbf{Brújula de Orim} gira violentamente y apunta hacia la oscuridad del arco.

\textbf{Continuar a la Escena II — El Umbral de los Primeros.}

\end{multicols}

\imagerim{images/cornisa_montana.png}

% --- SECCIÓN TÁCTICA ---
\newpage

\begin{tacticalscene}{El Desfiladero del Viento (I-A)}

\textbf{Descripción para los Jugadores:}
\begin{readbox}
El sendero es un filo de navaja. A vuestra izquierda, la pared de roca sube vertical, ofreciendo una cornisa estrecha cubierta de matorrales a unos cuatro metros de altura.

A vuestra derecha, el abismo. Pero si miráis con cuidado, veis una grieta o saliente unos metros por debajo del borde, apenas visible entre la niebla del precipicio.

Delante, la niebla se agita. Primero veis el vapor del aliento de los perros, luego el brillo de las máscaras de hierro.
\end{readbox}

\textbf{Geometría y Terreno (Para el GM):}
\begin{itemize}
    \item \textbf{Dimensiones:} Pasillo largo de 15m x 45m (10 x 30 casillas).
    \item \textbf{Nivel 0 (Camino):} Hielo negro. \textbf{Terreno Difícil}. Correr requiere \textbf{Acrobacias - Fácil (+20)} o caer (Stun 1 asalto).
    \item \textbf{Nivel +4m (Cornisa Alta - Oeste):} Plataforma de 1.5m de ancho.
    \begin{itemize}
        \item \textit{Acceso:} \textbf{Trepar - Medio}.
        \item \textit{Ventaja:} Cobertura Parcial y Vegetación (Ocultación). Inmune a carga de perros.
    \end{itemize}
    \item \textbf{Nivel -4m (Saliente Bajo - Este):} Grieta en la pared del abismo.
    \begin{itemize}
        \item \textit{Acceso:} \textbf{Trepar - Difícil (-10)}.
        \item \textit{Ventaja:} \textbf{Ocultación Total} visual y olfativa.
        \item \textit{Peligro:} Si son atacados desde arriba, los enemigos tienen +20 OB (Posición elevada) y el PJ no tiene DB de escudo contra proyectiles.
    \end{itemize}
\end{itemize}

\tcbline

\textbf{ESCALADO DEL ENCUENTRO}
\begin{itemize}
    \item \textbf{Vigías Rastreadores:} 1 por cada 3 Jugadores (redondeando hacia arriba).
    \item \textbf{Mastines del Velo:} 1 por cada \textbf{Vigía} (Ratio 1:1).
    \item \textit{Ejemplo para 4 PJs:} 2 Vigías + 2 Mastines.
\end{itemize}

\tcbline

\textbf{ENEMIGOS}

\begin{statblock}[images/vigia_rastreador.png]{Vigía Rastreador}{Nvl 5 (Ranger) | PV 60 | AT 8 | DB 30}
    \textbf{Objetivo:} Tocar el cuerno (1 Asalto completo, sin defensa). Luego, disparar y huir.
    \textbf{Ataques:}
    \begin{itemize}
        \item \textbf{Ballesta Pesada (Rango):} +80 OB. Alcance 60m.
        \\ \textit{Tabla:} \textbf{Crossbow}. \textit{Tamaño:} \textbf{Medium (M)}.
        \\ \textit{Críticos:} Perforación (Puncture).
        
        \item \textbf{Espada Corta (Melee):} +60 OB.
        \\ \textit{Tabla:} \textbf{Short Sword}. \textit{Tamaño:} \textbf{Medium (M)}.
    \end{itemize}
\end{statblock}

\begin{statblock}[images/mastin_velo.png]{Mastín del Velo}{Nvl 3 (Animal) | PV 45 | AT 3 | DB 20}
    \textbf{Apariencia:} Perros de presa desollados, músculo vivo y ojos violetas.
    \textbf{Táctica:} Cargar y derribar. Atacan en jauría (+5 OB por cada aliado adyacente).
    \textbf{Ataques:}
    \begin{itemize}
        \item \textbf{Fauces (Melee):} +50 OB.
        \\ \textit{Tabla:} \textbf{Bite}. \textit{Tamaño:} \textbf{Medium (M)}.
        \\ \textit{Especial:} Si el ataque hace crítico, realiza una maniobra de \textbf{Presa (Grapple)} gratuita (+40) para tirar al suelo al objetivo.
    \end{itemize}
\end{statblock}

\end{tacticalscene}
