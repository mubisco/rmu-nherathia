\section{Escena I --- Las Sombras a la Zaga}

\begin{multicols*}{2}

\textbf{Tipo de escena:} Supervivencia extrema, acecho inverso y combate táctico en terreno difícil.

\begin{readaloud}
El aire ya no huele a marismas ni a ceniza. Huele a hielo y a piedra antigua. Habéis dejado atrás las colinas de Valmor para adentraros en las estribaciones de las \textbf{Costillas del Norte}.

La niebla aquí es diferente. Es más fina, cristalina y letalmente fría. Vuestro aliento forma nubes densas frente a vosotros. El mapa de Erun os guía hacia un paso estrecho entre dos picos que parecen colmillos rotos.

Pero no estáis solos.

Desde hace horas, sentís una presión en la nuca. Al mirar atrás, hacia el valle cubierto por el mar de nubes del Velo, veis movimientos puntuales. Formas oscuras que se mueven rápido, ignorando el cansancio. La \textbf{Vigilia del Silente} no ha renunciado a su presa. Una avanzadilla de cazadores está cortando distancia rápidamente.
\end{readaloud}

\subsubsection*{El Terreno: El Desfiladero del Viento}

Estáis en un sendero de cabras, con una pared de roca a un lado y una caída de 50 metros al otro. El suelo está cubierto de nieve dura y hielo negro.

\textbf{Estado del Grupo:}
Debido a la altitud y el ritmo forzado, todos los PJ deben hacer una tirada de \textbf{Aguante (Endurance) - Medio}.
\begin{itemize}
    \item \textbf{Fallo:} -10 a la actividad por Fatiga acumulada y el frío.
\end{itemize}

\subsubsection*{La Decisión Táctica}

Los perseguidores os alcanzarán en menos de 10 minutos. No podéis correr más rápido que ellos en este terreno (ellos van ligeros, vosotros cargados). Tenéis que decidir cómo recibirles.

\paragraph{Opción A: La Emboscada (Combate)}
El terreno os favorece. Podéis ocultaros tras rocas y atacar cuando pasen.
\begin{itemize}
    \item \textbf{Ventaja:} Ataque sorpresa (+20 OB) y cobertura.
    \item \textbf{Riesgo:} Si alguno escapa, alertará al grueso de la tropa que viene detrás (aumentando la dificultad de la Escena V final).
\end{itemize}

\paragraph{Opción B: El Señuelo (Sigilo/Astucia)}
Intentar ocultar el rastro y dejar un señuelo falso (una capa, huellas falsas hacia un barranco) para que pasen de largo.
\begin{itemize}
    \item \textbf{Mecánica:} Tirada de \textbf{Supervivencia} (para borrar huellas) y \textbf{Acechar} (para esconderse).
    \item \textbf{Riesgo:} Los mastines tienen un olfato sobrenatural (+50 a Percepción olfativa). Si fallan, serán descubiertos en posición desventajosa.
\end{itemize}

\subsubsection*{Los Cazadores}

La avanzadilla dobla el recodo. El viento trae el sonido de cadenas y jadeos guturales.

Son 3 figuras. Un \textbf{Vigía Rastreador} y dos \textbf{Mastines del Velo}.

\begin{npcsheet}[Cazador de Élite][images/vigia_rastreador.png]{Vigía Rastreador (Nvl 5)}
    \appearance{Armadura ligera de cuero negro, máscara de hierro sin boca, capa de piel de lobo blanco.}
    \motivation{Marcar a los objetivos para que la "Artillería" mágica pueda fijarlos.}
    \stats{
        \textbf{Nivel:} 5 (Ranger/Explorador) \quad \textbf{PV:} 60 \\
        \textbf{AT:} 8 (Cuero Reforzado) \quad \textbf{DB:} 30 (Reflejos) \\
        \textbf{OB:} +80 (Arco Largo), +60 (Espada Corta) \\
        \textbf{Habilidades:} Rastreo +90, Percepción +60.
    }
    \secrets{Lleva un \textbf{Cuerno de Llamada}. Si tiene una acción libre, lo soplará. Si suena, en 1d10 asaltos llegará una Sombra Mayor (ver Cap. II) invocada por el sonido.}
\end{npcsheet}

\begin{npcsheet}[Bestia Corrupta][images/mastin_velo.png]{Mastín del Velo (Nvl 3)}
    \appearance{Perros enormes, sin piel, músculo vivo y hueso expuesto. Sus ojos brillan con luz violeta.}
    \stats{
        \textbf{Nivel:} 3 \quad \textbf{PV:} 45 \quad \textbf{AT:} 3 (Piel dura) \\
        \textbf{OB:} +50 (Mordisco Grande) \\
        \textbf{Especial:} Si muerden, hacen una presa (Grapple) automática.
    }
\end{npcsheet}

\subsubsection*{Resolución del Encuentro}

\textbf{Si combaten:}
Deben eliminar al Vigía antes de que toque el cuerno. Si el cuerno suena, el eco rebota en las montañas y el cielo se oscurece momentáneamente: la "mirada" de Nherath se ha posado sobre ellos (aumenta la dificultad de encuentros futuros).

\textbf{Si se esconden:}
Si superan la tirada opuesta (Sigilo vs Percepción de los perros), los mastines se confunden con el señuelo y el grupo pasa de largo, perdiéndose montaña arriba. Los PJ ganan tiempo valioso.

\subsubsection*{El Descubrimiento del Paso}

Tras superar a los cazadores, llegáis a la cima del desfiladero. Ante vosotros, incrustada en la ladera de la montaña como una herida geométrica, está la entrada a las ruinas.

No parece una construcción humana. Son bloques ciclópeos de piedra negra, perfectamente lisos, que forman un arco triangular.

Erun (si viaja con vosotros) o vuestro propio conocimiento (Lore) os confirma la sospecha:
\begin{displayquote}
\emph{"Arquitectura de los Primeros. Esto estaba aquí antes de que los hombres aprendieran a apilar piedras. Y mirad..."}
\end{displayquote}

La \textbf{Brújula de Orim} gira violentamente y apunta, no al norte magnético, sino directamente hacia la oscuridad del arco. Pero hay algo más: una corriente de aire cálido y viciado sale de las profundidades.

Huele a tierra húmeda, a raíces antiguas... y a algo podrido.

\textbf{Comienza la Escena II — El Umbral de los Primeros.}

\end{multicols*}
