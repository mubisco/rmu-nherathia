\section{Escena II --- El Umbral de los Primeros}

\imagerim{images/entrada_dungeon.png}

\begin{multicols}{2}

\textbf{Tipo de escena:} Exploración, Lore (Investigación), Puzle místico y Transición.

\begin{readbox}
La subida final ha sido agónica. El viento aquí arriba no solo corta la piel, sino que ahoga el sonido, creando un silencio antinatural.

Frente a vosotros, incrustada en la ladera, la entrada a las ruinas viola la geometría. Bloques de piedra negra, lisa como el vidrio, forman un arco triangular invertido de unos 9 metros de altura (6 casillas).

Bajo el arco, una losa de oscuridad sólida y mate absorbe la luz de vuestras antorchas.

Sin embargo, hay una contradicción. Vuestros ojos ven un muro impenetrable, pero si cerráis los ojos un instante, el viento suena diferente. Suena hueco. Como si soplara a través de un túnel abierto que vuestra vista se niega a aceptar.
\end{readbox}

\subsubsection*{El Misterio de la Arquitectura}

La estructura causa vértigo y náuseas leves al mirarla fijamente. Los ángulos parecen moverse.

\begin{mechanics}{Investigación: La Geometría No Euclidiana}
\textbf{Tirada de Maniobra Estática (Elegir una):}
\begin{itemize}
    \item \textbf{Ciencia: Arquitectura - Muy Difícil (-20):} Para deducir que la estructura utiliza geometría no euclidiana para engañar a los sentidos.
    \item \textbf{Saber Arcano (Lore: Magical) - Difícil (-10):} Para reconocer una "Barrera de Percepción", una defensa pasiva que usa la mente del intruso en su contra.
\end{itemize}
\textbf{Éxito:} Confirman que es obra de los \textbf{Primeros Habitantes}. No usaban llaves físicas, sino resonancias y conceptos psíquicos.
\end{mechanics}

\subsubsection*{El Puzle: El Umbral de la Fe Ciega}

La puerta es físicamente intangible pero psíquicamente impenetrable.
\begin{itemize}
    \item Si intentan tocarla o cruzarla mirando, sus manos chocan contra un campo de fuerza generado por su propio rechazo a la realidad imposible.
    \item Si lanzan una piedra, rebota.
\end{itemize}

\textbf{La Brújula de Orim:}
Al acercarse al umbral, la brújula no gira loca. La aguja se queda \textbf{completamente inmóvil} apuntando al centro de la oscuridad, ignorando el norte magnético. Orim no mentía: marca el camino, pero no lo abre.

\begin{mechanics}{Resolución: El Salto de Fe}
Para cruzar, los personajes deben negar a sus propios sentidos. Deben avanzar hacia la pared sólida con los ojos cerrados (o vendados), suprimiendo el instinto de supervivencia que les grita que van a chocar.

\textbf{La Pista de Yenar (Acto I):}
\emph{"La puerta no está cerrada, vuestros ojos sí. Para entrar donde la luz no existe, debéis dejar de buscarla."}

\textbf{Tirada:} \textbf{Autodisciplina (Self Discipline) -- Difícil (-10).}
\begin{itemize}
    \item \textbf{Modificador:} Si se vendan los ojos físicamente, obtienen un bono de \textbf{+10}.
    \item \textbf{Ayuda:} Si se cogen de la mano y uno guía, solo el guía hace la tirada por todos.
\end{itemize}
\end{mechanics}

\subsubsection*{Consecuencias}

\textbf{Fallo (El Choque Psíquico):}
El personaje duda en el último segundo. Su cuerpo se tensa esperando el golpe.
\begin{itemize}
    \item \textbf{Efecto:} Choca contra la barrera invisible. Sufre \textbf{1 asalto de Aturdimiento (Stun)} y es repelido hacia atrás. Su mente ha hecho real la pared.
\end{itemize}

\textbf{Éxito (La Inmersión):}
El personaje avanza hacia la oscuridad sólida. El cuerpo espera el impacto... pero no llega.

\begin{readbox}
Sientes un frío oleoso atravesarte la piel, como si cruzaras una cascada de agua helada y espesa. El sonido del viento cambia de golpe, volviéndose cavernoso y resonante.

Al abrir los ojos, ya no estás en la nieve. Estás al otro lado. La pared de oscuridad está a tu espalda, y frente a ti se extiende un túnel de piedra negra, iluminado por hongos bioluminiscentes pálidos. El aire es cálido, húmedo y huele a tierra antigua.
\end{readbox}

\subsubsection*{Transición al Interior}

Una vez que el primero cruza, puede llamar a los demás ("¡Es seguro!"), lo que otorga un bono de \textbf{+50} a la tirada de Autodisciplina de los restantes, o simplemente extender la mano a través del velo negro y arrastrarlos dentro (sin tirada).

\textbf{Continuar a la Escena III — Ecos de la Convivencia.}

\end{multicols}
