\section{Escena II --- El Umbral de los Primeros}

\begin{multicols*}{2}

\textbf{Tipo de escena:} Exploración, Lore, Puzle místico y Transición atmosférica.

\begin{readaloud}
Habéis dejado atrás la nieve y a los cazadores, subiendo por la cicatriz de la montaña. Aquí arriba, el viento aúlla con un tono diferente, hueco, como si soplara a través de una flauta rota.

Frente a vosotros, la entrada a las ruinas no es una cueva natural ni una construcción humana. Es una violación de la geometría. Bloques de piedra negra, lisa como el vidrio y fría como el vacío, se incrustan en la ladera formando un arco triangular invertido. Las líneas de la estructura parecen moverse si las miráis fijamente, causando náuseas leves.

Sin embargo, algo suaviza esta arquitectura alienígena. Enredaderas gruesas y antiguas, ahora congeladas y petrificadas, cubren los bordes del arco como venas grises. Alguien —o algo— intentó reclamar este lugar para la naturaleza hace mucho tiempo.

La puerta no está cerrada con madera ni hierro. Simplemente, la oscuridad bajo el arco es sólida. Una losa de negrura perfecta que rechaza la luz de vuestras antorchas.
\end{readaloud}

\subsubsection*{El Misterio de la Entrada}

Este lugar fue construido por los **Primeros Habitantes** [1], mucho antes de que Nherath o los humanos caminaran por estas tierras. Los Guardianes Extraplanares (el Druida, el Monje y la Maga) lo usaron como refugio, sellándolo después.

\textbf{Investigación (Percepción/Lore):}

\begin{itemize}
    \item \textbf{Percepción (Vista) - Difícil (-10):} Bajo la capa de hielo y las enredaderas petrificadas, hay tallas en la piedra negra. No son letras, son patrones de ondas y espirales que recuerdan al sonido visualizado.
    \item \textbf{Lore: Historia/Mitos - Muy Difícil (-20):} (Erun tiene éxito automático). Esta arquitectura pertenece a la "Era del Silencio Activo". Los Primeros no usaban llaves físicas; usaban resonancias, recuerdos o conceptos.
    \item \textbf{Lore: Natura/Druida - Medio (+0):} Las enredaderas no crecieron aquí por azar. Fueron cantadas. Es magia druídica de alto nivel, usada para "sellar" o "contener" lo que hay dentro.
\end{itemize}

\subsubsection*{El Puzle: La Llave de la Memoria}

La puerta no tiene cerradura física. Si intentan golpearla o picarla, las herramientas rebotan sin dejar marca (AT 20, como muro de fuerza).

\textbf{La Reacción de la Brújula:}
La **Brújula de Orim** [2] se vuelve loca al acercarse a menos de 5 metros de la oscuridad sólida. La aguja gira violentamente y emite un zumbido, vibrando al unísono con la piedra.

\begin{tcolorbox}[colback=blue!5, colframe=blue!40, title=Resolviendo el Sello]
Para abrir el camino, los jugadores deben conectar las pistas que han recibido a lo largo de la campaña.

\textbf{Pista 1 (Yenar en la Celda):} \emph{"Las ruinas no se abren. Se dejan entrar cuando recuerdas su nombre antiguo."} [3] y \emph{"Id hacia donde la niebla se vuelve torpe."} [4].

\textbf{Pista 2 (La Brújula):} Orim dijo que marcaba \emph{"los Ecos antiguos"} [2], no el norte.

\textbf{Solución:}
Un personaje debe sostener la brújula frente a la oscuridad y pronunciar (o pensar con fuerza) la "contraseña" conceptual.
No es una palabra mágica, es una intención. Deben evocar el concepto de \textbf{"Recordar"} o \textbf{"Romper el Silencio"}.

Si los jugadores están atascados, pueden hacer una tirada de \textbf{Intuición (Intuition)} o \textbf{Runas}. Un éxito les recuerda la frase de Yenar: \emph{"Buscad la Corona de Fuego Pálido..."} [5]. Al decir "Fuego Pálido" o "Corona", la puerta reacciona.
\end{tcolorbox}

\subsubsection*{La Apertura}

Cuando se resuelve el puzle (ya sea por deducción o tirada de suerte):

\begin{readaloud}
La aguja de la brújula se detiene de golpe, apuntando hacia el centro de la oscuridad. Se oye un sonido que no es físico: es como el recuerdo de una campana tañendo muy lejos.

Las enredaderas petrificadas crujen y se apartan ligeramente, no como plantas que se mueven, sino como si la piedra misma cambiara de forma. La losa de oscuridad se disuelve, no como humo, sino como un párpado que se abre.

Una bocanada de aire escapa del interior. Es aire cálido, húmedo y viciado. Huele a tierra fértil encerrada durante décadas, a ozono mágico... y, muy al fondo, al dulzor empalagoso de la podredumbre.
\end{readaloud}

\subsubsection*{Transición al Interior}

El camino está abierto. Es un túnel descendente, perfectamente liso y circular, que se adentra en las entrañas de la montaña.

Al cruzar el umbral, sentís una presión en los oídos. La "realidad" aquí dentro es más densa. La influencia del Velo intenta entrar, pero las protecciones de los Guardianes (las enredaderas y sellos) aún mantienen una burbuja de aislamiento, aunque está debilitada y corrupta.

\textbf{Nota para el DJ:} Si los jugadores deciden descansar aquí antes de bajar, es un lugar seguro (santuario relativo). La entrada se puede "cerrar" desde dentro manipulando la brújula a la inversa.

\textbf{Comienza la Escena III — Ecos de la Convivencia.}

\end{multicols*}
