\section{Escena V --- El Paso del Gigante}

\imagerim{images/bifurcation.png}

\begin{multicols}{2}

\textbf{Tipo de escena:} Decisión narrativa, Epílogo y Consecuencia Táctica.

\begin{readbox}
El Santuario de la Raíz muere a vuestras espaldas. El techo de la cámara del druida se desploma, sellando el mal en la oscuridad.

Solo queda una salida: la grieta natural que se abrió tras el trono. Una corriente de aire gélido entra por ella. Corréis por el túnel ascendente hasta salir a la superficie, cegados por la luz blanca de la nieve.

Estáis en una cornisa de alta montaña, en la vertiente norte de las Costillas del Norte. El aire es tan fino que marea. Frente a vosotros, el sendero se bifurca.
\end{readbox}

\subsubsection*{La Bifurcación del Destino}

En el cruce de caminos, un hito ominoso marca la frontera: una pica de hierro negro con una cabeza de orco congelada clavada en la punta. El viento agita un estandarte tribal hecho de piel humana.

\textbf{El Camino del Norte (La Libertad Salvaje):}
El sendero desciende hacia los valles exteriores. Según el mapa de Erun, lleva al \textbf{Paso de Khar-Vel}.
\begin{itemize}
    \item \textbf{Lo que sabéis:} Es territorio de Clanes Orcos. Brutal y peligroso, pero allí el "Silente" no tiene poder.
    \item \textbf{El Precio:} La \textbf{Raíz del Horizonte} vibra con dolor. Si cruzáis la frontera, el vínculo con la tierra se romperá y el bastón se convertirá en madera muerta. Perderéis la reliquia, pero salvaréis la vida.
\end{itemize}

\textbf{El Camino del Oeste (El Retorno):}
Un sendero de cabras gira hacia el oeste y luego serpentea de vuelta hacia el sur, reingresando en los bosques de Nherathia.
\begin{itemize}
    \item \textbf{Lo que sabéis:} Lleva de vuelta al dominio del Velo, permitiendo mantener la Reliquia despierta.
    \item \textbf{El Precio:} Volvéis a la guerra. Y la Vigilia os estará esperando.
\end{itemize}

\begin{gmnote}[Decisión de Grupo: El Fin del Preludio] Esta elección marca el final de nuestra introducción a Nherathia.
\textbf{Si eligen el Norte (Supervivencia):} La historia tiene un cierre satisfactorio y definitivo. Los personajes escapan del horror y conservan la vida. Es el final perfecto si queréis dejar esta experiencia como una historia corta de terror y supervivencia.
\textbf{Si eligen el Sur (Resistencia):} Es una declaración de intenciones. Los jugadores aceptan convertir este "aperitivo" en una campaña completa, explorando los secretos del Velo y luchando una guerra de reconquista.
Pregúntales: \textbf{¿Sentís que la historia de vuestros personajes termina con su libertad, o que su verdadera batalla acaba de empezar?} \end{gmnote}

\subsubsection*{Resolución A: El Exilio (Norte)}

Si eligen el Norte, la escena es narrativa.

\begin{readbox}
Dáis la espalda a Nherathia. Al cruzar el hito de la cabeza de orco, el bastón en vuestras manos se vuelve gris y ligero. La magia lo abandona con un suspiro.

Camináis hacia las tierras salvajes. Sois libres. Habéis sobrevivido al Velo, aunque Nherathia se queda sola en la oscuridad.
\end{readbox}
\textbf{Fin de la Aventura: Final "Supervivientes".}

\subsubsection*{Resolución B: El Retorno (Sur)}

Si eligen volver al Sur, la Reliquia brilla con fuerza. El camino desciende suavemente, dejando la roca desnuda para entrar en una zona de coníferas y arbustos cubiertos de escarcha.

Aquí es donde vuestras acciones pasadas cobran su precio.

\paragraph{Caso 1: Silencio Previo (Éxito en Escena I)}
Si lograsteis que no sonara el cuerno:

\begin{readbox}
El sendero se abre a una pequeña explanada boscosa. Allí, un solitario \textbf{Vigía Rastreador} está apoyado contra un pino, limpiando la nieve de sus botas. Su ballesta está descargada a su lado. No espera que nadie venga desde las ruinas malditas.
\end{readbox}

\textbf{Situación:} El enemigo está \textbf{Distraído}.
\begin{itemize}
    \item \textbf{Sigilo:} Los PJ reciben un bono de \textbf{+30 a Acechar} si deciden rodearlo y marcharse sin combatir.
    \item \textbf{Combate:} Si atacan, tienen un asalto de \textbf{Sorpresa Total} (+20 OB, el enemigo no actúa).
\end{itemize}

\paragraph{Caso 2: La Alarma (Fracaso en Escena I)}
Si el cuerno sonó, saben que estáis ahí.

\begin{readbox}
La explanada no está vacía. Una patrulla de la Vigilia ha montado un perímetro defensivo entre los árboles. Dos guardias vigilan el sendero con escudos levantados, y un capitán da instrucciones en voz baja. Sabían que vendríais.
\end{readbox}

\textit{Id al \textbf{Encuentro Táctico V-A} para resolver el combate.}

\subsubsection*{Cierre del Módulo}

Tras superar el encuentro (o esquivarlo):

\begin{readbox}
Con el camino despejado, os internáis en la espesura. Tenéis un mapa, una Reliquia despierta y la voluntad de resistir. La guerra por Nherathia acaba de empezar.
\end{readbox}

\begin{center}
\textbf{FIN DEL MÓDULO: ECOS DEL ALBA PERDIDA}
\end{center}

\end{multicols}

% --- SECCIÓN TÁCTICA ---
\newpage

\begin{tacticalscene}{La Explanada del Bosque (V-A)}

\textbf{Configuración del Escenario:}
\begin{itemize}
    \item \textbf{Terreno:} Una zona abierta de 30m x 30m salpicada de pinos y rocas grandes.
    \item \textbf{Cobertura:} Los árboles y rocas ofrecen \textbf{Cobertura Dura Parcial (+20 DB)} y abundantes lugares para esconderse.
    \item \textbf{Vegetación:} Los arbustos densos cuentan como \textbf{Ocultación (Concealment)} pero no paran proyectiles.
    \item \textbf{Iluminación:} Luz de día (fría y clara).
\end{itemize}

\tcbline

\textbf{ESCALADO DEL ENCUENTRO (Caso 2: Alarma)}

Ajustar la fuerza enemiga según el número de PJs:
\begin{itemize}
    \item \textbf{3 Jugadores:} 1 Capitán + 1 Guardia de Choque + 1 Vigía Rastreador.
    \item \textbf{4 Jugadores:} 1 Capitán + 2 Guardias de Choque + 1 Vigía Rastreador.
    \item \textbf{5 Jugadores:} 1 Capitán + 2 Guardias de Choque + 2 Vigías Rastreadores.
    \item \textbf{6 Jugadores:} 1 Capitán + 3 Guardias de Choque + 2 Vigías Rastreadores.
\end{itemize}

\textit{Nota: Si es el Caso 1 (Silencio), solo hay 1 Vigía Rastreador (Distraído).}

\tcbline

\textbf{ENEMIGOS}

\begin{statblock}[images/vigia_rastreador.png]{Vigía Rastreador}{Nvl 5 (Rogue) | PV 60 | AT 8 | DB 20}
    \textbf{Táctica:} Se mantiene en Cobertura. Dispara al PJ con menos armadura.
    \textbf{Ataques:}
    \begin{itemize}
        \item \textbf{Ballesta Pesada (Rango):} +80 OB. Alcance 60m. \textit{Tabla:} \textbf{Crossbow}. \textit{Tamaño:} \textbf{Medium (M)}.
        \item \textbf{Espada Corta (Melee):} +60 OB. \textit{Tabla:} \textbf{Short Sword}. \textit{Tamaño:} \textbf{Medium (M)}.
    \end{itemize}
\end{statblock}

\begin{statblock}[images/guardia_choque.png]{Guardia de Choque}{Nvl 4 (Fighter) | PV 75 | AT 10 | DB 40}
    \textbf{Táctica:} Bloquear el paso. Usan "Parada Parcial" (+20 DB) para trabar a los PJs.
    \textbf{Ataques:}
    \begin{itemize}
        \item \textbf{Maza de Guerra (Melee):} +70 OB.
        \\ \textit{Tabla:} \textbf{War Hammer}. \textit{Críticos:} Aplastamiento (Krush).
        \item \textbf{Golpe de Escudo (Melee):} +50 OB. \textit{Tabla:} \textbf{Shield Bash}. \textit{Tamaño:} \textbf{Small (S)}.
    \end{itemize}
\end{statblock}

\begin{statblock}[images/capitan_guardia.png]{Capitán de la Guardia}{Nvl 6 (Paladin) | PV 95 | AT 14 | DB 30}
    \textbf{Profesión:} Paladin (Canalización). \textbf{PP:} 12.
    \textbf{Listas de Conjuros (Spell Lists):}
    \begin{itemize}
        \item \textbf{Holy Shields (Base) - Rango 6:} \textit{Uso:} Lanza \emph{Protection Prayer II} (6 PP) en el asalto 1.
        \\ \textit{Efecto:} +10 DB y +10 RR a todos los aliados en 10' (3m).
        \item \textbf{Inspiring Ways (Base) - Rango 6:} \textit{Uso:} Lanza \emph{Courage} (1 PP) si los PJs usan miedo.
        \item \textbf{Concussion's Way (Open) - Rango 4:} \textit{Uso:} \emph{Clotting True} o \emph{Pain Relief} para curarse.
    \end{itemize}
    \textbf{Ataques:}
    \begin{itemize}
        \item \textbf{Espada Ancha (Melee):} +90 OB. \textit{Tabla:} \textbf{Broadsword}.
    \end{itemize}
    \textbf{Equipo:} Cota de Mallas (AT 14), Poción de Curación (x1).
\end{statblock}

\end{tacticalscene}
