\section{Escena V --- El Paso del Gigante}

\imagerim{images/bifurcation.png}

\begin{multicols*}{2}

\textbf{Tipo de escena:} Decisión narrativa, Epílogo y posible Combate Final.

\begin{readaloud}
El Santuario de la Raíz muere a vuestras espaldas. El techo de la cámara del druida se desploma, sellando el mal en la oscuridad.

Solo queda una salida: la grieta natural que se abrió tras el trono. Una corriente de aire gélido entra por ella. Corréis por el túnel ascendente hasta salir a la superficie, cegados por la luz blanca de la nieve.

Estáis en una cornisa de alta montaña, en la vertiente norte de las Costillas del Norte. El aire es tan fino que marea. Frente a vosotros, el sendero se bifurca.
\end{readaloud}

\subsubsection*{La Bifurcación del Destino}

En el cruce de caminos, un hito ominoso marca la frontera: una pica de hierro negro con una cabeza de orco congelada clavada en la punta. El viento agita un estandarte tribal hecho de piel humana.

\textbf{El Camino del Norte (La Libertad Salvaje):}
El sendero desciende hacia los valles exteriores. Según el mapa de Erun, lleva al \textbf{Paso de Khar-Vel}.
\begin{itemize}
    \item \textbf{Lo que sabéis:} Es territorio de Clanes Orcos. Brutal y peligroso, pero allí el "Silente" no tiene poder. Es la salida de Nherathia.
    \item \textbf{El Precio:} La \textbf{Raíz del Horizonte} vibra con dolor al acercarse al norte. Si cruzáis esa frontera, el vínculo con la tierra se romperá y el bastón se convertirá en madera muerta. Perderéis la reliquia, pero salvaréis la vida.
\end{itemize}

\textbf{El Camino del Oeste (El Retorno):}
Un sendero de cabras gira hacia el oeste y luego serpentea de vuelta hacia el sur, reingresando en las montañas de Nherathia.
\begin{itemize}
    \item \textbf{Lo que sabéis:} Lleva de vuelta al dominio del Velo, pero os permite mantener la Reliquia y usar su poder para luchar.
    \item \textbf{El Precio:} Volvéis a la boca del lobo. Y la Vigilia os estará esperando.
\end{itemize}

\begin{tcolorbox}[colback=blue!5, colframe=blue!40, title=Decisión de Campaña]
Este es el momento de pausar y preguntar a los jugadores:
\textbf{¿Queréis acabar la historia aquí (Huida) o queréis iniciar la campaña de reconquista (Retorno)?}
\end{tcolorbox}

\subsubsection*{Resolución A: El Exilio (Norte)}

Si eligen el Norte, la escena es narrativa.

\begin{readaloud}
Dáis la espalda a Nherathia. Al cruzar el hito de la cabeza de orco, el bastón en vuestras manos se vuelve gris y ligero. La magia lo abandona con un suspiro.

Camináis hacia las tierras salvajes. Sois libres. Habéis sobrevivido al Velo, aunque Nherathia se queda sola en la oscuridad.
\end{readaloud}
\textbf{Fin de la Aventura: "Supervivientes".}

\subsubsection*{Resolución B: El Retorno (Sur)}

Si eligen volver al Sur, la Reliquia brilla con una luz ámbar intensa, aceptando vuestra decisión. Pero el camino no está despejado.

Dependiendo de lo ocurrido en la \textbf{Escena I}:

\paragraph{Caso 1: Silencio Previo (Éxito en Escena I)}
Si lograsteis que no sonara el cuerno:

\begin{readaloud}
Avanzáis con sigilo. En un recodo del camino, veis a un solitario \textbf{Vigía Rastreador}. Está sentado sobre una roca, mirando hacia el valle, distraído. No espera que nadie salga vivo de las ruinas.
\end{readaloud}

\textbf{Encuentro:} 1 Vigía Rastreador (Nvl 5).
\textbf{Ventaja:} Sorpresa Total. Pueden evitarlo (Sigilo Medio) o eliminarlo fácilmente en un asalto sorpresa.

\paragraph{Caso 2: La Alarma (Fracaso en Escena I)}
Si el cuerno sonó, la Vigilia ha desplegado una red de contención.

\begin{readaloud}
Al girar el recodo, os encontráis con una barricada improvisada. Escudos negros bloquean el paso y ballestas apuntan desde las rocas superiores. Sabían que vendríais.
\end{readaloud}

\textbf{Encuentro (Dificultad Alta):}
\begin{itemize}
    \item \textbf{1 Capitán de la Guardia (Nvl 6):} Lidera y da órdenes (+10 a iniciativa de sus aliados).
    \item \textbf{2 Vigías Rastreadores (Nvl 5):} En terreno elevado (Rocas, Cobertura Parcial).
    \item \textbf{2 Guardias de Choque (Nvl 4):} Bloqueando el camino en cuerpo a cuerpo.
\end{itemize}

\textbf{La Ayuda de la Reliquia:}
Como habéis elegido luchar, la Raíz del Horizonte despierta. El portador puede activar su poder \textbf{"Santuario de Raíces"} como una acción instantánea (0 AP), creando una zona de protección o enredando a los enemigos (Tratar como hechizo \emph{Entangle} de área).

\begin{tcolorbox}[colback=white, colframe=black!75!white, title=\textbf{Detalles del Mapa Táctico: El Bloqueo de la Cornisa}]
\textbf{Especificaciones Generales:} \begin{itemize} \item \textbf{Dimensiones:} Un pasillo largo y curvo. 4 a 6 casillas de ancho (6-9 metros) x 20 casillas de largo. \item \textbf{Escala:} 1 casilla = 1.5 metros (5 pies). \item \textbf{Iluminación:} \textbf{Luz Brillante (Bright Light)}. El sol se refleja en la nieve. Los personajes que miren hacia el sur (hacia los guardias) tienen un penalizador de -10 a Percepción por el resplandor de la nieve si no llevan protección ocular. \end{itemize}
\textbf{Elementos del Terreno:} \begin{itemize} \item \textbf{El Sendero (Zona de Muerte):} Suave pendiente ascendente hacia los enemigos. \begin{itemize} \item \textbf{Suelo:} Nieve apisonada y hielo. Se considera \textbf{Terreno Difícil} (Maniobras de movimiento a -20). \item \textbf{El Borde (Este):} Las casillas del borde derecho dan al vacío. Una caída de 300 metros. Cualquier personaje que reciba un crítico de \textit{Desequilibrio (Unbalance)} o \textit{Impacto (Impact)} junto al borde debe hacer una Maniobra de \textbf{Acrobacias - Medio} para no caer (o quedar colgado). \end{itemize}
\item \textbf{La Barricada (El Cuello de Botella):}
A 15 casillas de la entrada de los PJ.
\begin{itemize}
    \item Ocupa todo el ancho del camino excepto 1 casilla estrecha.
    \item Otorga \textbf{Cobertura Dura (Hard Cover, +50 DB)} a los 2 Guardias de Choque que están tras ella.
    \item Cruzarla requiere una Maniobra de \textbf{Escalar/Saltar - Difícil} o romperla (tiene 50 PV).
\end{itemize}

\item \textbf{Los Nidos de Tirador (Altura):}
Dos repisas en la pared de roca (Oeste), a 3 metros (2 casillas) de altura sobre el suelo.
\begin{itemize}
    \item \textbf{Ocupantes:} 2 Vigías Rastreadores con ballestas.
    \item \textbf{Ventaja:} Tienen línea de visión clara sobre los PJ y \textbf{Cobertura Parcial} contra ataques desde abajo.
    \item \textbf{Acceso:} Solo se puede subir trepando (\textbf{Escalar - Difícil}) o volando.
\end{itemize}
\end{itemize}
\textbf{Posicionamiento Inicial:} \begin{itemize} \item \textbf{Zona A (PJ):} Entran por el Norte. Tienen 10 metros de terreno abierto antes de la barricada. \item \textbf{Zona B (Enemigos):} \begin{itemize} \item \textbf{2x Guardias de Choque:} Tras la barricada, en postura defensiva (Parada completa o preparando ataque si alguien se acerca). \item \textbf{1x Capitán:} Detrás de los guardias, dando bonos de Liderazgo (+10 OB a sus aliados). \item \textbf{2x Rastreadores:} En los nidos elevados, preparando la acción "Disparar al primero que lance un hechizo o cargue". \end{itemize} \end{itemize}
\end{tcolorbox}

\subsubsection*{Cierre del Módulo}

Tras superar el encuentro:

\begin{readaloud}
Con el camino despejado, os ocultáis en los bosques de las laderas. Tenéis un mapa, una Reliquia despierta y la voluntad de resistir.

Abajo, en el valle, la Torre del Silente sigue dominando el paisaje, pero por primera vez, su sombra no parece infinita. La guerra por Nherathia acaba de empezar.
\end{readaloud}

\begin{center}
\textbf{FIN DEL MÓDULO: ECOS DEL ALBA PERDIDA}
\end{center}

\end{multicols*}
