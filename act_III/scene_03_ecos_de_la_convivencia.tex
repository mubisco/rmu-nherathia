\section{Escena III --- Ecos de la Convivencia}

\begin{multicols*}{2}

\textbf{Tipo de escena:} Exploración de Mazmorra (Dungeon Crawl), Puzles y Combate.

\begin{readaloud}
El túnel de los Primeros Habitantes termina abruptamente, desembocando en una arquitectura diferente. Aquí, la piedra negra y alienígena ha sido trabajada toscamente por manos humanas hace décadas. Muros de mampostería dividen el espacio en estancias habitables.

El aire es cálido y huele a ozono, sudor antiguo y tierra húmeda.

Llegáis a una sala central circular. Es el corazón de un hogar abandonado. Hay una mesa llena de mapas podridos, restos de comida petrificada y tres sillas.

Pero no estáis solos. El aire brilla con luz residual. Figuras translúcidas —ecos de memoria atrapados en el tiempo— parpadean en la sala. Veis a tres personas discutir en silencio:
\begin{itemize}
    \item Un \textbf{Humano} con túnica verde que golpea la mesa con frustración (El Druida).
    \item Una \textbf{Mujer} con ropas llenas de sigilos que señala un libro (La Maga).
    \item Un \textbf{Tabaxi} (felino humanoide) que medita apartado, limpiando un bastón (El Monje).
\end{itemize}
\end{readaloud}

\subsubsection*{El Hub: El Atrio de la Decisión}

Esta sala conecta las tres áreas privadas de los Guardianes.

\begin{itemize}
    \item \textbf{Oeste:} Un arco de madera simple. Lleva a los aposentos del Monje.
    \item \textbf{Este:} Una cortina de cuentas de cristal. Lleva a los aposentos de la Maga.
    \item \textbf{Norte:} Una puerta de piedra maciza, sellada herméticamente por raíces grises y runas brillantes. Es el camino al Druida (Escena IV).
\end{itemize}

\textbf{El Puzle de la Puerta Norte:}
La puerta tiene dos huecos vacíos en su superficie.
\begin{itemize}
    \item Un hueco circular y profundo: Requiere \textbf{"El Peso de la Disciplina"} (Ala Oeste).
    \item Un hueco con forma de llama: Requiere \textbf{"La Luz del Entendimiento"} (Ala Este).
\end{itemize}

\textbf{Investigación (Mesa Central):}
Si examinan los mapas, confirman que los Guardianes planearon dividirse tras fracasar en su misión de "sanar" el Velo. El Druida se quedó aquí para contener algo; los otros partieron a morir (confirmando las tumbas o pistas que hayan visto antes).

\subsection*{A. Ala Oeste: El Dojo del Eco}

\begin{readaloud}
El pasillo desemboca en una sala amplia y austera. El suelo está cubierto de esteras de paja podridas. Hay muñecos de entrenamiento de madera y hierro en las esquinas.

Al entrar, el eco del Monje Tabaxi parpadea en el centro de la sala, ejecutando una kata perfecta.
\end{readaloud}

\subsubsection*{La Trampa: El Suelo del Equilibrio}
El suelo de esta sala está encantado para probar el equilibrio. Baldosas sueltas ceden bajo el peso si no se pisa con precisión.

\textbf{Mecánica:} Para cruzar la sala sin activar las defensas, cada PJ debe superar una maniobra de \textbf{Acrobacias (Acrobatics) - Medio}.
\begin{itemize}
    \item \textbf{Fallo:} Una baldosa cede. Se escucha un clic metálico. Los muñecos de entrenamiento se activan.
\end{itemize}

\subsubsection*{Combate: El Autómata de Entrenamiento}
Si fallan la trampa (o si tocan el cofre del fondo), el muñeco principal cobra vida. Es un constructo de madera reforzada con hierro, diseñado para no detenerse nunca.

\begin{npcsheet}[Constructo Menor][images/automata_madera.png]{Autómata del Dojo (Nvl 4)}
    \appearance{Maniquí de madera con articulaciones de bronce y cuatro brazos que giran con cuchillas romas.}
    \stats{
        \textbf{Nivel:} 4 \quad \textbf{PV:} 80 \quad \textbf{AT:} 10 (Madera Dura) \\
        \textbf{DB:} 40 (Movimientos programados) \\
        \textbf{OB:} +70 (Golpe Múltiple - 2 ataques por asalto) \\
        \textbf{Inmunidades:} Aturdimiento (Stun), Sangrado, Mental.
    }
    \secrets{\textbf{Punto Débil:} Si se le hace un crítico de \emph{Desequilibrio (Unbalance)}, pierde su DB durante 2 asaltos.}
\end{npcsheet}

\textbf{Recompensa (El Objeto Clave):}
En un pequeño altar al fondo, descansa una esfera de metal negra, increíblemente pesada para su tamaño. Es \textbf{"El Peso de la Disciplina"}.
\begin{tcolorbox}[colback=yellow!10, colframe=black!50, title=Botín adicional]
Junto a los restos del autómata, encontráis un arcón de madera de cedro:
\begin{itemize}
    \item \textbf{Brazales de la Corteza:} Brazales de madera pulida que vibran al ser golpeados. Cuentan como \textbf{Defensa Mágica +10 DB} (no se apila con escudo). [Treasure Law Sec 6.6]
    \item \textbf{Aceite de la Víbora:} Un frasco pequeño. Aplicar sobre la piel (1 asalto). Otorga los efectos de \emph{Velocidad (Haste I)} durante 3 asaltos.
\end{itemize}
\end{tcolorbox}

\subsection*{B. Ala Este: El Scriptorium Fractal}

\begin{readaloud}
Esta sala desafía la lógica. Estanterías llenas de libros mohosos cubren las paredes, pero algunas estanterías flotan a medio metro del suelo. Espejos rotos cubren el techo, reflejando una habitación que no coincide con la realidad.

En el centro, un atril de cristal sostiene una llama que no quema ni se mueve.
\end{readaloud}

\subsubsection*{El Peligro: Magia Residual}
La magia de la hechicera se ha corrompido. Al entrar, los PJ sienten náuseas.
\textbf{Tirada de RR Mental (Voluntad):}
\begin{itemize}
    \item \textbf{Fallo:} El personaje ve "doble". Penalizador de -10 a todas las acciones por vértigo mientras esté en esta sala.
\end{itemize}

\subsubsection*{Encuentro: La Anomalía Arcana}
Al acercarse al atril para coger la llama, la energía residual del cuarto coalesce en una forma agresiva. No es un monstruo biológico, es un "error" mágico viviente.

\begin{npcsheet}[Elemental/Energía][images/anomalia_arcana.png]{Anomalía Cromática (Nvl 4)}
    \appearance{Una esfera flotante de luz cambiante y fragmentos de espejo que emite un zumbido eléctrico.}
    \stats{
        \textbf{Nivel:} 4 \quad \textbf{PV:} 50 \quad \textbf{AT:} 1 (Energía) \\
        \textbf{DB:} 60 (Parpadeo/Desplazamiento) \\
        \textbf{Ataque:} +60 (Rayo de Energía - Crítico de Electricidad o Calor aleatorio). Rango 20m.
    }
    \secrets{\textbf{Inestable:} Cada vez que recibe daño físico, libera una pequeña descarga (Ataque +30 a todos en 2m). Es vulnerable a ataques de \emph{Disipar Magia} o armas de hierro frío.}
\end{npcsheet}

\textbf{Recompensa (El Objeto Clave):}
Al derrotar a la anomalía, el atril se vuelve seguro. Los PJ pueden tomar \textbf{"La Luz del Entendimiento"}, que resulta ser un cristal con una llama perpetua atrapada dentro (funciona como antorcha inagotable).
\begin{tcolorbox}[colback=purple!10, colframe=black!50, title=Botín adicional]
Entre los cristales rotos de la anomalía, varios objetos han permanecido intactos:
\begin{itemize}
    \item \textbf{Lente de Decifrado:} Un monóculo con marco de plata. Otorga un bono de \textbf{+15 a Runas (Runes)} al intentar leer textos mágicos o antiguos.
    \item \textbf{2x Pergaminos de Disipación:} Contienen el hechizo \emph{Dispel Magic (Nvl 5)}. Permiten a un usuario de magia intentar cancelar un efecto mágico activo.
\end{itemize}
\end{tcolorbox}

\subsection*{La Apertura del Sanctasanctórum}

Con las dos llaves en su poder (\textbf{"El Peso"} y \textbf{"La Luz"}), los PJ regresan al Atrio Central.

Al colocar la Esfera y el Cristal en los huecos de la puerta norte, el mecanismo responde. No hay trampas, solo una aceptación mecánica. Las runas de la puerta brillan con color ámbar y los cerrojos de piedra se retraen con un sonido profundo, como el de la tierra moviéndose.

La puerta de piedra se abre hacia dentro, revelando oscuridad y el sonido de una respiración lenta y cavernosa.

Del interior no sale aire viciado, sino un olor a bosque profundo, a tierra mojada y a flores nocturnas... pero con un matiz dulzón a podredumbre. El camino al Druida está abierto.

\textbf{Comienza la Escena IV — El Jardín de Piedra.}

\end{multicols*}
