\section{Escena III --- Ecos de la Convivencia}

\begin{multicols}{2}

\textbf{Tipo de escena:} Exploración de Mazmorra (Dungeon Crawl), Puzles y Combate Táctico.

\begin{readbox}
El túnel de los Primeros Habitantes termina abruptamente, desembocando en una arquitectura diferente. Aquí, la piedra negra y alienígena ha sido trabajada toscamente por manos humanas hace décadas. Muros de mampostería dividen el espacio en estancias habitables.

El aire es cálido y huele a ozono, sudor antiguo y tierra húmeda.

Llegáis a una sala central circular. Es el corazón de un hogar abandonado. Hay una mesa llena de mapas podridos, restos de comida petrificada y tres sillas.

Pero no estáis solos. El aire brilla con luz residual. Figuras translúcidas —ecos de memoria atrapados en el tiempo— parpadean en la sala. Veis a tres personas discutir en silencio: un \textbf{Humano} con túnica verde que golpea la mesa, una \textbf{Mujer} rodeada de libros y un \textbf{Tabaxi} que medita apartado.
\end{readbox}

\subsubsection*{El Hub: El Atrio de la Decisión}

Esta sala conecta las áreas privadas de los Guardianes. En la mesa, los mapas confirman que los Guardianes se dividieron tras fracasar en su misión.

\textbf{La Puerta del Norte (El Objetivo):}
Una puerta de piedra maciza, sellada herméticamente por raíces grises y runas, bloquea el camino hacia el Druida. Tiene dos huecos vacíos que actúan como llaves:
\begin{itemize}
    \item Un hueco circular y profundo: Requiere \textbf{"El Peso de la Disciplina"} (Ala Oeste).
    \item Un hueco con forma de llama: Requiere \textbf{"La Luz del Entendimiento"} (Ala Este).
\end{itemize}

\subsection*{A. Ala Oeste: El Dojo del Eco}

Un arco de madera simple lleva a esta zona. El pasillo huele a serrín y aceite viejo.

Si decidís entrar y recuperar la esfera, preparaos para probar vuestro equilibrio.

\textit{Id al \textbf{Encuentro Táctico III-A} (Pág. Siguiente) para resolver la Sala del Dojo.}

\textbf{Resolución y Recompensa:}
Tras derrotar al Autómata, el silencio vuelve al Dojo. En un altar al fondo recogéis la esfera pesada de hierro negro. Además, en un arcón de cedro encontráis el equipo del monje:

\begin{itembox}{Tesoro del Monje}
\begin{itemize}
    \item \textbf{Brazales de la Corteza:} Brazales de madera pulida. Cuentan como \textbf{Defensa Mágica +10 DB} (no se apila con escudo).
    \item \textbf{Aceite de la Víbora:} Un frasco pequeño. Aplicar sobre la piel (1 asalto). Otorga los efectos de \emph{Velocidad (Haste I)} durante 3 asaltos.
\end{itemize}
\end{itembox}

\subsection*{B. Ala Este: El Scriptorium Fractal}

Una cortina de cuentas de cristal separa esta zona del atrio. Al apartarlas, sentís una estática que eriza el vello de los brazos.

La magia aquí se ha estancado y corrompido. Necesitaréis voluntad de hierro.

\textit{Id al \textbf{Encuentro Táctico III-B} (Pág. Siguiente) para resolver el Scriptorium.}

\textbf{Resolución y Recompensa:}
Al disipar la Anomalía, el caos se ordena. Podéis tomar el cristal con la llama perpetua del atril. Entre los libros quemados, salváis algunos objetos útiles:

\begin{itembox}{Tesoro de la Maga}
\begin{itemize}
    \item \textbf{Lente de Descifrado:} Un monóculo con marco de plata. Otorga un bono de \textbf{+15 a Runas (Runes)}.
    \item \textbf{2x Pergaminos de Disipación:} Contienen el hechizo \emph{Dispel Magic (Nvl 5)}.
\end{itemize}
\end{itembox}

\subsection*{La Apertura del Sanctasanctórum}

Con las dos llaves en vuestro poder, regresáis al Atrio.

Al colocar la Esfera y el Cristal en los huecos, las runas brillan con color ámbar y los cerrojos de piedra se retraen con un sonido profundo, como el de la tierra moviéndose.

La puerta se abre, revelando un olor a bosque profundo y flores nocturnas... con un matiz dulzón a podredumbre.

\textbf{Continuar a la Escena IV — El Jardín de Piedra.}

\end{multicols}

% --- HOJA TÁCTICA 1: DOJO ---
\newpage

\begin{tacticalscene}{Ala Oeste: El Dojo del Eco (III-A)}

\textbf{Descripción para los Jugadores:}
\begin{readbox}
Entráis en una sala rectangular y austera. El suelo no es de piedra, sino de tablas de madera que gimen bajo vuestro peso, algunas podridas por la humedad.

En el centro, el "eco" fantasmal del Monje Tabaxi parpadea una última vez, realizando una kata defensiva perfecta antes de desvanecerse.

En las esquinas hay armeros vacíos y muñecos de entrenamiento destrozados. Pero uno, al fondo, está intacto. Es un constructo de madera y bronce con cuatro brazos. Cuando pisáis la primera tabla, su cabeza gira 180 grados con un chasquido seco. No tiene rostro, solo una runa grabada donde debería estar la boca.
\end{readbox}

\textbf{Geometría y Terreno (Para el GM):}
\begin{itemize}
    \item \textbf{Dimensiones:} Rectángulo de 9m (ancho) x 15m (largo) (6x8 casillas).
    \item \textbf{Suelo Inestable:} Todo el suelo cuenta como \textbf{Terreno Difícil} (Peligroso). Correr o cargar requiere una tirada de \textbf{Acrobacias - Medio} o el PJ tropieza (pierde el asalto).
    \item \textbf{Columnas:} Hay 4 columnas de madera (Cobertura Dura) que sostienen el techo.
    \item \textbf{Posición Inicial:} Los PJ entran por el Sur. El Autómata está al Norte, protegiendo el altar.
\end{itemize}

\tcbline

\textbf{ENEMIGO ÚNICO}

\begin{statblock}[images/automata_madera.png]{Autómata de Entrenamiento}{Nvl 4 | PV 80 | AT 10 | DB 40}
    \textbf{Tipo:} Constructo (Inmune a Sangrado, Stun, Mental, Veneno). \\
    \textbf{Debilidad:} Fuego (x2 Daño). Un crítico de \emph{Desequilibrio (Unbalance)} anula su DB durante 1 asalto. \\
    \textbf{Táctica:} No se mueve del sitio. Espera a que se acerquen. Si le atacan a distancia, usa la acción de \emph{Defensa Total} (+40 DB adicional).
    \textbf{Ataques:}
    \begin{itemize}
        \item \textbf{Giro de Bastón (Melee):} +70 OB. 2 Ataques por asalto a objetivos adyacentes.
        \\ \textit{Tabla:} \textbf{Quarterstaff}.
    \end{itemize}
\end{statblock}

\end{tacticalscene}

% --- HOJA TÁCTICA 2: SCRIPTORIUM ---
\newpage

\begin{tacticalscene}{Ala Este: El Scriptorium Fractal (III-B)}

\textbf{Descripción para los Jugadores:}
\begin{readbox}
Esta sala marea. Las estanterías repletas de libros mohosos no siguen líneas rectas; se curvan y algunas parecen flotar a medio metro del suelo. Espejos rotos cubren el techo, reflejando una habitación que no coincide con la realidad: en el reflejo, la sala está ardiendo.

En el centro, sobre un atril de cristal, flota la llama que buscáis. Pero el aire a su alrededor chisporrotea. Una esfera de luz caótica y colores imposibles se materializa, emitiendo un zumbido que os hace sangrar la nariz.
\end{readbox}

\textbf{Geometría y Terreno (Para el GM):}
\begin{itemize}
    \item \textbf{Dimensiones:} Sala circular de 12m (6 casillas) de diámetro.
    \item \textbf{Estanterías Flotantes:} Actúan como \textbf{Cobertura Dura} móvil. Cambian de posición cada asalto (iniciativa 0), bloqueando líneas de visión aleatorias.
    \item \textbf{Espejos:} Los ataques a distancia sufren un penalizador de \textbf{-20 OB} debido a los reflejos confusos.
    \item \textbf{Peligro Ambiental:} Al inicio del combate, todos tiran \textbf{RR Mental (Nvl 4)}. Fallo: \textbf{-10 a todas las acciones} por vértigo.
\end{itemize}

\tcbline

\textbf{ENEMIGO ÚNICO}

\begin{statblock}[images/anomalia_arcana.png]{Anomalía Cromática}{Nvl 4 | PV 50 | AT 1 | DB 60}
    \textbf{Tipo:} Elemental de Energía (Inmune a críticos físicos normales 'A', solo recibe daño puro. Vulnerable a magia).
    \textbf{Defensa:} Su DB 60 proviene de un efecto constante de \emph{Blur/Desplazamiento}.
    \textbf{Táctica:} Flota a 2m del suelo. Ataca al que lleve metal o use magia.
    \textbf{Ataques:}
    \begin{itemize}
        \item \textbf{Arco Voltaico (Rango):} +60 OB. Alcance 20m.
        \\ \textit{Tabla:} \textbf{Shock Bolt}.
        \\ \textit{Críticos:} Electricidad (Electricity).
    \end{itemize}
    \textbf{Muerte:} Al llegar a 0 PV, explota. Ataque de Bola de Energía (Shock Ball) +30 en radio de 3m.
\end{statblock}

\end{tacticalscene}
