\section{Escena III --- Ecos de la Convivencia}

\begin{multicols*}{2}

\textbf{Tipo de escena:} Exploración, Lore (arqueología), Trampa ambiental y Combate táctico.

\begin{readaloud}
El túnel desemboca en una serie de cámaras excavadas con precisión geométrica. La piedra negra de los Primeros Habitantes sigue aquí, pero el lugar fue habitado y adaptado por humanos hace décadas.

El aire es cálido y húmedo, con un olor dulzón a fermentación.

Lo que veis no es una mazmorra, sino los restos de un hogar. Hay un área común con una mesa de roble podrida, restos de hamacas de tela que cuelgan como telarañas y estanterías saqueadas por el tiempo. Aquí vivieron los tres Guardianes. Aquí comieron, planearon y durmieron mientras el mundo exterior caía bajo la sombra de Nherath.

Pero algo ha entrado. Un musgo violeta y gris cubre las esquinas, y las raíces que cuelgan del techo palpitan con un fluido oscuro. El refugio se ha convertido en un invernadero enfermo.
\end{readaloud}

\subsubsection*{La Sala de los Mapas (Lore)}

En la mesa central, bajo una capa de polvo y esporas, hay varios mapas desplegados. Están muy deteriorados, pero la tinta mágica ha resistido.

\textbf{Investigación (Percepción / Lore: Historia):}
\begin{itemize}
    \item \textbf{El Plan de Guerra:} Los mapas muestran movimientos de tropas de hace 50 años. Flechas que apuntan hacia las Marismas del Sur (el dominio de Nherath).
    \item \textbf{La Dispersión:} Hay tres marcas recientes (en la escala temporal del mapa) sobre el papel.
    \begin{itemize}
        \item Una marca roja aquí, en las \textbf{Ruinas del Norte}.
        \item Una marca azul en el \textbf{Este}, hacia la costa (donde estaría la Maga).
        \item Una marca amarilla en el \textbf{Sur}, profundo en el bosque (donde iría el Monje).
    \end{itemize}
\end{itemize}

\subsubsection*{El Diario del Jardinero}

En un rincón que parece haber sido un laboratorio de herboristería, encontráis un diario encuadernado en corteza. Pertenece al Druida. Las últimas páginas son legibles:

\begin{tcolorbox}[colback=yellow!10, colframe=black!50, title=El Diario de Kaelen (El Druida)]
\emph{"La herida en mi costado no cierra. La magia de este lugar... no cura, transforma. He convencido a Jyn (el Monje) y a Sarra (la Maga) de que se marchen. Deben esconder sus reliquias lejos de aquí. Yo me quedaré."}

\emph{"Siento cómo la Raíz me llama. El bastón está echando raíces en mi propia sangre. Si me fundo con el santuario, podré mantener la puerta cerrada desde dentro. Seré el tapón en la botella. Que la tierra me perdone por lo que voy a convertirme."}
\end{tcolorbox}

\subsubsection*{El Jardín Corrupto (El Obstáculo)}

Para llegar a la cámara interior (donde está el cuerpo del Druida), debéis cruzar lo que fue el jardín hidropónico de los Guardianes.

Ahora es una jungla de enredaderas pálidas y flores que gotean un néctar soporífero. El suelo está cubierto de \textbf{Vides del Velo} (una mutación de las \emph{Dreamvines}).

\begin{mechanicbox}{Peligro: El Polen del Olvido}
Al entrar en la sala, las flores reaccionan al calor corporal y sueltan una nube de esporas.
\textbf{Tirada de RR Física (Constitución) - Nivel 5:}
\begin{itemize}
    \item \textbf{Fallo:} El personaje cae en un sueño profundo y lleno de pesadillas. No puede despertar hasta que reciba daño o sea sacudido vigorosamente (1 acción completa).
    \item \textbf{Efecto Secundario:} Mientras duermen, las vides atacan automáticamente (Presa 100\%).
\end{itemize}
\end{mechanicbox}

\subsubsection*{Combate: Las Vides Estranguladoras}

En cuanto alguien pisa el centro de la sala (o cae dormido), las plantas atacan. No son monstruos errantes, son el sistema defensivo del santuario, ahora loco y corrupto.

\begin{npcsheet}[Planta Carnívora Mutada][images/vides_velo.png]{Vides del Velo (Nvl 4)}
    \appearance{Lianas gruesas de color grisáceo que se mueven como serpientes constrictoras. Tienen espinas que supuran un líquido negro.}
    \stats{
        \textbf{Nivel:} 4 \quad \textbf{PV:} 60 (Por cepa) \quad \textbf{AT:} 3 (Fibrosa) \\
        \textbf{Tamaño:} Grande (Big) [Reduce críticos] \\
        \textbf{OB:} +60 (Presa/Grapple) \\
        \textbf{Ataque:} Presa Grande (Large Grapple). Si logran una presa, inyectan veneno (RR Física o -10 actividad).
    }
    \secrets{Son extremadamente vulnerables al \textbf{Fuego} (Doble daño y pánico). Si los PJ compraron Fuego de Alquimista a Garel, este es el momento de usarlo.}
\end{npcsheet}

\textbf{Resolución:}
\begin{itemize}
    \item \textbf{Cepa Central:} Hay 3 cepas principales bloqueando la puerta del fondo.
    \item \textbf{Cortar y Correr:} No hace falta matarlas a todas. Si causan suficiente daño (20 PV) a una cepa, esta se retrae un asalto, permitiendo el paso.
    \item \textbf{Uso de la Reliquia (Opcional):} Si algún jugador tiene una habilidad druídica o usa magia de naturaleza, puede intentar "calmar" las plantas momentáneamente (Maniobra de Influencia/Canalización Difícil).
\end{itemize}

\subsubsection*{La Puerta al Sanctasanctórum}

Tras superar el jardín, llegáis a una puerta de piedra sellada con raíces petrificadas que forman un sello hermético.

A través de las grietas, se ve una luz ámbar pulsante, rítmica como un corazón lento. Es la \textbf{Raíz del Horizonte}, latiendo en la oscuridad. Pero junto a ella, una sombra se mueve. El Druida os espera.

\textbf{Comienza la Escena IV — El Jardín de Piedra.}

\end{multicols*}
