\section{Escena IV --- El Jardín de Piedra}

\begin{multicols*}{2}

\textbf{Tipo de escena:} Boss Fight (Combate con fases), Puzzle Táctico y Clímax Emocional.

\begin{readaloud}
La puerta de raíces se abre con un gemido de madera seca.

La cámara final es inmensa, una cúpula natural iluminada por grietas en el techo por las que se filtra la luz de la luna, cayendo como lanzas de plata sobre la oscuridad.

En el centro, sobre un montículo de tierra negra que parece respirar, se alza un árbol colosal. Pero no es de madera; es de piedra blanca, veteada de gris.

Fusionado con el tronco, a medio camino entre ser hombre y ser rama, está el cuerpo de \textbf{Aelor}, el Druida Guardián. Su piel es corteza, su cabello es musgo seco. No está muerto, pero tampoco vivo. Es una cáscara preservada por el Velo.

En sus manos, clavado en el suelo como una estaca que impide que algo salga, brilla el bastón. \textbf{La Raíz del Horizonte}. La madera del bastón pulsa con luz ámbar, luchando contra las sombras que intentan trepar por él.

Cuando entráis, los ojos del Druida se abren. No tienen iris, solo una luz violeta vacía. El suelo tiembla. Las raíces a vuestros pies se despiertan.
\end{readaloud}

\subsubsection*{El Enemigo: El Guardián Corrupto}

El Velo ha animado el cuerpo de Aelor y lo utiliza como una marioneta para defender la posición.

\textbf{Mecánica del Combate:}
El Guardián está \textbf{Inmóvil} (fusionado al árbol central), pero controla todo el terreno.
El objetivo no es matar a Aelor (su cuerpo es durísimo), sino \textbf{arrancar el Bastón} para romper el flujo de energía corrupta.

\begin{npcsheet}[Jefe Final - Muerto Viviente/Planta][images/guardian_corrupto.png]{Aelor, El Guardián Corrupto}
    \appearance{Un torso humanoide gigante de madera y piedra emergiendo de un árbol. Sus brazos son ramas largas con garras de obsidiana.}
    \stats{
        \textbf{Nivel:} 8 (Boss) \quad \textbf{PV:} 200 (Ver "El Vínculo") \\
        \textbf{AT:} 10 (Piel de Piedra/Corteza) \\
        \textbf{DB:} 20 (Dura piel) \\
        \textbf{OB:} +90 (Golpe de Rama - Aplastamiento Grande) \\
        \textbf{Ataque a Distancia:} +70 (Espinas - Rango 20m)
    }
    \secrets{\textbf{Inmunidades:} Aturdimiento, Sangrado, Mental. Vulnerable al Fuego (x2 Daño).}
\end{npcsheet}

\subsubsection*{Peligros del Entorno (Lair Actions)}

Al inicio de cada asalto, el Guardián usa una acción gratuita del entorno:

\begin{enumerate}
    \item \textbf{Raíces Sujetadoras:} Todos los PJ deben hacer una maniobra de \textbf{Acrobacias/Evasión (Medio)}. Fallo: Quedan \textbf{Inmovilizados} (Presa 50\%) por raíces que brotan del suelo.
    \item \textbf{Drenaje Vital:} Si alguien está apresado por raíces, pierde \textbf{5 PV} que el Guardián absorbe para curarse.
\end{enumerate}

\subsubsection*{El Objetivo: La Raíz del Horizonte}

El bastón está clavado en el "corazón" del árbol. Para finalizar el combate, un personaje debe llegar hasta él, agarrarlo y superar una tirada de fuerza o voluntad para arrancarlo.

\textbf{El Aura de la Reliquia:}
El área a 3 metros del bastón es "Terreno Sagrado".
\begin{itemize}
    \item Dentro de este círculo, los PJ ganan \textbf{+20 RR} contra los hechizos del Guardián.
    \item El Guardián no puede atacar con raíces dentro de este círculo (solo con sus garras).
\end{itemize}

\textbf{Arrancar el Bastón:}
Requiere una Acción de Asalto Completo estando adyacente.
\begin{itemize}
    \item \textbf{Tirada:} \textbf{Fuerza (Brawn)} o \textbf{Voluntad (Will)} - Difícil (-10).
    \item \textbf{Ayuda:} Si el Guardián recibe más de 20 PV de daño en ese asalto, su agarre se debilita (+20 a la tirada de arrancar).
    \item \textbf{Éxito:} El bastón se libera. (Ver Desenlace).
\end{itemize}

\subsubsection*{Desenlace: La Liberación}

En el momento en que el bastón es arrancado de la tierra corrupta:

\begin{readaloud}
Un destello de luz ámbar, cálida como un atardecer de verano, estalla desde el bastón, cegando la luz violeta de los ojos del Guardián.

El cuerpo de piedra y madera de Aelor se arquea y emite un sonido que no es un grito, sino un suspiro profundo, como un árbol viejo cayendo en el bosque.

Las raíces negras se marchitan instantáneamente, convirtiéndose en polvo gris. La corrupción se evapora. El cuerpo del Druida se desploma hacia adelante, separándose del árbol, ahora inerte.

Por un segundo, antes de que el cuerpo se deshaga en tierra fértil, veis una sonrisa en su rostro de corteza.

\emph{"Gracias... el camino está... al norte."}
\end{readaloud}

\subsubsection*{Recompensa: La Primera Reliquia}

El silencio vuelve a la sala, pero ahora es un silencio natural, tranquilo. Tenéis en vuestras manos la \textbf{Raíz del Horizonte}.

\begin{tcolorbox}[colback=green!5, colframe=green!40, title=Reliquia: La Raíz del Horizonte (Estado Dormido)]
\imagerim{images/raiz_horizonte.png}
Un bastón de fresno petrificado con vetas de luz ámbar.
\begin{itemize}
    \item \textbf{Bono:} +15 OB (Bastón).
    \item \textbf{Poder Pasivo:} +20 a Supervivencia y Orientación. El portador siente dónde está el norte y el agua pura.
    \item \textbf{Poder Activo (1/día):} \textbf{El Círculo de Raíces}. Clavándolo en el suelo, crea un refugio de 3m de radio. El Velo no puede cruzarlo y elimina la fatiga por clima dentro.
    \item \textbf{Resonancia:} Vibra suavemente si apunta hacia el Este o el Sur (donde están las otras reliquias perdidas).
\end{itemize}
\end{tcolorbox}

De repente, la sala empieza a temblar. No por magia, sino por inestabilidad estructural. Sin la magia del Guardián sosteniéndolo, el santuario colapsa. Una corriente de aire fresco entra por una grieta que se ha abierto tras el trono del árbol.

Es la salida.

\textbf{Comienza la Escena V (Final) — El Paso del Gigante.}

\end{multicols*}
