\section{Escena IV --- El Jardín de Piedra}

\begin{multicols*}{2}

\textbf{Tipo de escena:} Boss Fight (Combate con fases), Puzle Táctico y Clímax Emocional.

\begin{readbox}
La puerta de raíces se abre con un gemido de madera seca.

La cámara final es inmensa, una cúpula natural iluminada por grietas en el techo por las que se filtra la luz de la luna, cayendo como lanzas de plata sobre la oscuridad.

En el centro, sobre un montículo de tierra negra que parece respirar, se alza un árbol colosal de piedra blanca veteada de gris.

Fusionado con el tronco, a medio camino entre ser hombre y ser rama, está el cuerpo de \textbf{Poderun}. Su piel es corteza, su cabello es musgo seco. No está muerto, pero tampoco vivo; es una cáscara preservada por el Velo y la magia druídica retorcida.

En sus manos, clavado en el suelo como una estaca que impide que algo salga, brilla el bastón: \textbf{La Raíz del Horizonte}. La madera del bastón pulsa con luz ámbar, luchando contra las sombras que intentan trepar por él.

Cuando entráis, los ojos de Poderun se abren. No tienen iris, solo una luz violeta vacía. El suelo tiembla.
\end{readbox}

\subsubsection*{El Guardián del Sello}

Poderun no puede moverse del sitio (está enraizado), pero su magia impregna toda la sala. Su mente está fragmentada; os ve como parásitos intentando devorar las raíces sagradas.

El combate es inevitable, pero no es una lucha convencional. Poderun se regenera constantemente gracias a la tierra corrupta.

\textbf{El Objetivo:} El bastón está clavado en el "corazón" del árbol. Mientras siga conectado a la tierra, Poderun es inmortal. Para vencer, debéis arrancarlo.

\textit{Ver \textbf{Encuentro Táctico IV-A} al final de la escena para el mapa, las estadísticas de Poderun y las reglas de extracción.}

\subsubsection*{Desenlace}

En el momento en que la Raíz del Horizonte es liberada y el vínculo se rompe:

\begin{readbox}
Un destello de luz ámbar, cálida como un atardecer de verano, estalla desde el bastón, cegando la oscuridad violeta del Velo.

El cuerpo de piedra y madera de Poderun se arquea hacia atrás. La corteza de su rostro se resquebraja y cae, revelando por un instante la piel humana que hubo debajo. Sus ojos violetas parpadean y se vuelven de un color avellana natural, llenos de lágrimas.

\emph{"...la tierra... recuerda..."} —susurra con una voz que suena a hojas secas pisadas.

Luego, el cuerpo se deshace en polvo y tierra fértil. El árbol de piedra detrás de él se agrieta y se oscurece, quedando inerte y silencioso.
\end{readbox}

\subsubsection*{Recompensa: La Primera Reliquia}

El silencio vuelve a la sala. Entre el montón de tierra que fue el druida, encontráis una llave de hierro negro y una bolsa de componentes que no se ha podrido.

Pero el verdadero premio está en vuestras manos.

\begin{relicbox}{La Raíz del Horizonte (Estado Dormido)}
\imagerim{images/raiz_horizonte.png}
Un bastón de fresno petrificado con vetas de luz ámbar.
\begin{itemize}
    \item \textbf{Bono:} +15 OB (Bastón).
    \item \textbf{Poder Pasivo:} El portador ignora los penalizadores por terreno difícil natural (bosque/barro).
    \item \textbf{Poder Activo (1/día):} \textbf{Santuario de Raíces}. Clavándolo en el suelo (2 asaltos), crea una cúpula de raíces de 3m de radio. El Velo no puede cruzarla y proporciona descanso seguro (recuperación de Fatiga).
    \item \textbf{Resonancia:} Vibra suavemente si apunta hacia el Este o el Sur (donde están las otras reliquias).
\end{itemize}
\end{relicbox}

\begin{gmnote}[Nota de Lore]
Un personaje con \emph{Lore: History} o \emph{Read Runes} puede notar que el nombre grabado en la base del bastón no es "Poderun", sino "Pod...", seguido de runas en un idioma extranjero (Común de Faerûn) que significan "El Viajero".
\end{gmnote}

La cúpula empieza a temblar. Con la muerte del Guardián, el santuario se está volviendo inestable. Una grieta se abre tras el trono del árbol, revelando un túnel de salida hacia el aire frío de la montaña.

\textbf{Continuar a la Escena V — El Paso del Gigante.}

\end{multicols*}

% --- SECCIÓN TÁCTICA ---
\newpage

\begin{tacticalscene}{El Sanctasanctórum de Piedra (IV-A)}

\textbf{Configuración del Escenario:}
\begin{itemize}
    \item \textbf{Dimensiones:} Sala circular abovedada (30m diámetro).
    \item \textbf{Iluminación:} Penumbra (-20 a percepción/ataques distancia). Tres haces de luz lunar (Luz normal) caen aleatoriamente.
    \item \textbf{El Árbol:} Centro de la sala. Cobertura Dura. Poderun está fusionado en el lado Sur.
    \item \textbf{Zonas Seguras:} 4 losas de piedra negra dispersas. Inmunes al efecto de suelo.
\end{itemize}

\begin{mechanics}{Condición Ambiental: El Suelo Vivo}
En la \textbf{Fase de Mantenimiento (Upkeep)} de cada asalto, cualquier personaje que toque el suelo de tierra (no las losas) debe realizar una Maniobra:
\begin{itemize}
    \item \textbf{Tirada:} \textbf{Movimiento/Acrobacias - Medio}.
    \item \textbf{Fallo:} Las raíces sujetan al PJ. Queda \textbf{Inmovilizado} (0 AP movimiento, -20 acciones) hasta liberarse (Maniobra de Fuerza o cortar raíces).
    \item \textbf{Fallo Absoluto:} Sufre ataque de Presa (Grapple) +40.
\end{itemize}
\end{mechanics}

\begin{mechanics}{Objetivo: La Extracción (Maniobra Acumulada)}
Un PJ adyacente al bastón debe gastar su asalto completo para liberarlo. Objetivo: \textbf{100\% Acumulado}.
\begin{enumerate}
    \item \textbf{Elegir Habilidad (Dificultad Difícil -10):}
    \begin{itemize}
        \item Fuerza: \textit{Weight-training (St)}.
        \item Voluntad/Magia: \textit{Attunement} o \textit{Mental Focus}.
    \end{itemize}
    \item \textbf{El Precio del Dolor:} Si el PJ recibe daño (PV) mientras extrae, \textbf{resta los PV sufridos al \% acumulado}.
\end{enumerate}
\end{mechanics}

\tcbline

\textbf{ENEMIGOS}

\begin{statblock}[images/poderun_boss.png]{Poderun (Druida Corrupto)}{Nvl 7 | PV 150 | AT 10 (Piedra) | DB 20}
    \textbf{Tipo:} Constructo Grande (Inmune a Stun, Sangrado, Veneno). \\
    \textbf{Debilidad:} Fuego (x2 Daño, Críticos suben un nivel de severidad). \\
    \textbf{Ataques:}
    \begin{itemize}
        \item \textbf{Ramas-Garra (Melé):} +80 OB. Alcance 3m.
        \\ \textit{Tabla:} \textbf{Club} (Garrote). \textit{Tamaño:} \textbf{Big (B)}.
        \item \textbf{Lluvia de Espinas (Rango):} +60 OB. Alcance 20m.
        \\ \textit{Tabla:} \textbf{Stinger} (Aguijón). \textit{Tamaño:} \textbf{Medium (M)}.
    \end{itemize}
    \textbf{Hechizos:} \emph{Plant Mastery} (Animar raíces), \emph{Nature's Wrath}.
\end{statblock}

\end{tacticalscene}
