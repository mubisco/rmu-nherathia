\chapter{Nherathia}
\label{chap:nherathia}

\section{Visión General}

La región conocida actualmente como \textbf{Nherathia}, antaño llamada \emph{Eldharen} en fuentes antiguas, es un territorio costero aislado de aproximadamente 150 por 150 kilómetros. Popularmente se la denomina \emph{el Reino Silente}, nombre que alude tanto a su decadencia política como a los extraños fenómenos ambientales que la caracterizan.

Nherathia se presenta como una cuenca cerrada por barreras naturales hostiles. Salvo por su costa oriental, dominada por el puerto de Portimar, la región resulta extremadamente difícil de abandonar. Montañas, bosques ancestrales y extensas marismas configuran una geografía que actúa como prisión natural, reforzada por una persistente atmósfera de opresión y aislamiento.

\section{Clima y Condiciones Ambientales}

El clima de Nherathia es templado-frío de influencia oceánica, con una humedad constante durante todo el año. Las precipitaciones son frecuentes, generalmente en forma de lluvias finas y persistentes, mientras que las tormentas violentas son poco comunes.

Un fenómeno ampliamente documentado por cronistas y viajeros es la llamada \emph{Inversión Silente}: nieblas densas que permanecen en valles, riberas y marismas incluso durante las horas centrales del día. Estas brumas reducen la visibilidad, amortiguan los sonidos y contribuyen a una sensación general de quietud antinatural. En amplias zonas del interior, la noche no implica un cambio radical respecto al día, sino una progresión gradual hacia condiciones aún más inseguras.

\section{Hidrografía}

Los ríos constituyen la columna vertebral del territorio y las principales rutas de comunicación.

\subsection{Río del Humo}
Conocido en textos antiguos como \emph{Eldhárien}, el \textbf{Río del Humo} nace en las montañas del norte. Sus aguas son frías y rápidas, y es habitual observar vapores ascendiendo de su cauce, especialmente al amanecer y al anochecer. Históricamente ha marcado la frontera natural entre los asentamientos humanos y los territorios orcos.

\subsection{Río Pardo}
El \textbf{Río Pardo}, llamado \emph{Valcair} en registros mercantiles antiguos, atraviesa las llanuras centrales. Arrastra grandes cantidades de limo oscuro, fertilizando los campos pero también enturbiando el agua. A lo largo de su curso se concentran aldeas rurales empobrecidas, dependientes casi por completo de su caudal.

\subsection{Río Silente}
El \textbf{Río Silente} surge de la confluencia del Río del Humo y el Río Pardo. Su corriente es lenta y profunda, y su curso atraviesa el corazón económico de la región. Es notable la ausencia de sonido en muchas de sus riberas, circunstancia que ha alimentado supersticiones y temores desde tiempos antiguos.

\subsection{Agua Negra}
El \textbf{Agua Negra}, antiguo \emph{Thalas-Mir}, marca la frontera meridional. Su cauce se vuelve cada vez más lento y oscuro hasta disolverse en las marismas del sur. El reflejo del cielo en sus aguas suele aparecer distorsionado, incluso en condiciones de luz clara.

\subsection{Marismas de Nherath}
Las \textbf{Marismas de Nherath}, conocidas popularmente como \emph{el Limo Vivo}, constituyen el delta final del Agua Negra. El terreno es inestable, cambiante y peligroso. La humedad es constante, la visibilidad reducida y la orientación extremadamente difícil. Estas marismas son consideradas el epicentro de las influencias espirituales más intensas de la región.

\section{Orografía}

\subsection{Cordillera Norte}
Las \textbf{Costillas del Norte}, registradas en mapas antiguos como las Montañas de Khar-Dûm, forman una cadena abrupta y escarpada. Sus picos afilados y valles estrechos dificultan el paso. Esta cordillera está habitada por clanes orcos belicosos, que controlan los escasos pasos transitables.

\subsection{Cordillera Oeste}
Las llamadas \textbf{Montañas Mudas}, o Muro de Vael-Tor en textos eruditos, constituyen una barrera casi infranqueable. Son montañas antiguas, erosionadas, ricas en grietas, simas y cavernas profundas. Existen rumores persistentes sobre antiguos reinos subterráneos, pero ningún acceso superficial funcional permanece abierto.

\subsection{Colinas Interiores}
Las \textbf{Tierras Cansadas} son una sucesión de colinas suaves y mesetas erosionadas en el interior de la cuenca. Antiguamente intensamente cultivadas, hoy muestran signos claros de agotamiento del suelo y albergan numerosas ruinas dispersas de asentamientos abandonados.

\section{Bosques y Fronteras}

\subsection{El Bosque del Sur}
El \textbf{Bosque del Sur}, llamado \emph{Sylva Aelthir} en fuentes antiguas, es un bosque primigenio y extremadamente denso que marca la frontera con el imperio élfico. No existen caminos permanentes en su interior. Las patrullas élficas rara vez son vistas, pero su presencia es incuestionable. Cruzar esta frontera es considerado una sentencia de muerte.

\section{Torres Fronterizas}

\subsection{Torre del Juramento Roto}
La \textbf{Torre del Juramento Roto}, conocida antiguamente como \emph{Khar-Nemeth}, se alza en ruinas cerca del Paso de Khar-Vel. Fue un bastión defensivo clave en épocas pasadas y escenario de pactos fallidos con los clanes orcos del norte. Hoy sirve como advertencia silenciosa más que como fortificación.

\subsection{Torre del Umbral Verde}
La \textbf{Torre del Umbral Verde}, o \emph{Ael-Varyn} según la tradición oral, se encuentra en el límite septentrional del Bosque del Sur. Nunca fue tomada por la fuerza; simplemente fue abandonada cuando la expansión humana alcanzó su límite definitivo. Marca el punto donde los caminos se disuelven y el territorio deja de obedecer a reglas humanas.

\section{Asentamientos Mayores}

Aunque Nherathia alberga numerosas aldeas rurales dispersas, solo dos núcleos urbanos pueden considerarse verdaderos centros de poder territorial. Su importancia no radica únicamente en su población, sino en su posición estratégica dentro de la geografía regional.

\subsection{Portimar}

\textbf{Portimar} es el principal puerto de Nherathia y su capital de facto. Situada en la Costa de Cenbruma, controla el único acceso marítimo viable hacia el exterior. Su bahía natural, protegida parcialmente de las tormentas oceánicas, permite el atraque de naves mercantes y militares durante la mayor parte del año.

Desde un punto de vista geográfico, Portimar actúa como una válvula de escape: toda salida legal del territorio pasa por su puerto. Esto le confiere una influencia desproporcionada sobre el comercio, la migración y la política regional. La ciudad depende del interior para alimentos y materias primas, pero el interior depende de Portimar para cualquier contacto con el mundo exterior.

La niebla costera es persistente, y los accesos terrestres a la ciudad están bien definidos y fácilmente controlables, reforzando su papel como punto de paso obligado.

\subsection{Entrerríos}

\textbf{Entrerríos} se alza en el punto donde el Río del Humo y el Río Pardo confluyen para formar el Río Silente. Esta posición la convierte en el principal centro económico del interior y en el eje de distribución de bienes agrícolas, fluviales y manufacturados.

Geográficamente, Entrerríos es un nudo de tránsito. Casi todas las rutas fluviales y terrestres del interior pasan por sus inmediaciones o dependen de su infraestructura. Su prosperidad está ligada al control del agua y de los cruces, más que a la producción directa.

La ciudad se encuentra rodeada de tierras fértiles pero agotadas, y su crecimiento ha sido contenido por la degradación del suelo y por la creciente inseguridad en las rutas nocturnas. A pesar de ello, sigue siendo el corazón económico de Nherathia y un punto clave para comprender el equilibrio —o la falta de él— en la región.

\section{Conclusión Geográfica}

La geografía de Nherathia no es neutral. Montañas, ríos, bosques y marismas actúan conjuntamente para aislar, contener y desgastar a sus habitantes. El viaje es siempre lento, peligroso y costoso; la huida, una empresa casi imposible. Esta disposición natural convierte a la región en un territorio ideal para el estancamiento, la decadencia y el dominio indirecto, donde el control no necesita imponerse por la fuerza constante, sino que emana del propio paisaje.

