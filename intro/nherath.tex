%------------------------------------------------------------
% Nherath, El Eco Eterno
% Entrada canónica para módulo de campaña
%------------------------------------------------------------

\chapter{Nherath, El Eco Eterno}
\label{chap:nherath}

\textit{El Silente, Señor del Último Recuerdo, Amo de las Marismas}

\section{Identidad}

\begin{itemize}
  \item \textbf{Nombre verdadero:} Nherath
  \item \textbf{Títulos litúrgicos:} El Eco Eterno; El Señor del Último Recuerdo
  \item \textbf{Nombre tabú entre el pueblo:} El Silente
  \item \textbf{Símbolo:} El Sol Muerto --- un disco negro rodeado por una corona fracturada de fuego pálido
  \item \textbf{Dominios:} muerte, memoria, obediencia, silencio
\end{itemize}

\section{Naturaleza}

Nherath no nació como dios ni como espíritu.  
Fue un hechicero mortal que descubrió los principios del \textbf{ascenso divino}: la verdad prohibida de que un alma puede volverse dios si su nombre perdura en la fe tras la muerte.

Aterrorizado por la posibilidad del olvido, Nherath rechazó morir.  
Mediante rituales hoy perdidos, se ató al mundo como \textbf{espectro consciente}, sellando su alma en un cuerpo de niebla, recuerdo y voluntad.

Desde entonces, su existencia depende de un principio simple y cruel:  
\textbf{mientras alguien lo recuerde, lo tema o lo nombre, Nherath existe.}

\begin{quote}
``El alma se disuelve en el olvido.  
Yo soy el recuerdo que se niega a morir.''
\end{quote}

\section{Fe y propósito}

Nherath no busca un imperio ni una expansión territorial tradicional.  
Su ambición es más absoluta: un \textbf{culto perfecto}, un mundo donde cada mente viva mantenga su nombre presente, aunque sea como un susurro de terror.

Aspira a convertirse en un dios verdadero, pero su fe está \textbf{corrompida}:  
no se alimenta del amor ni de la esperanza, sino del miedo, la emoción más persistente del alma mortal.

Por ello, su doctrina central impone el \textbf{rezo silencioso}.  
La voz es peligrosa; el pensamiento, eterno.

El culto enseña que:
\begin{itemize}
  \item temer al Silente es creer en él;
  \item dudar de él también lo nutre;
  \item incluso odiarlo lo fortalece, mientras su nombre persista.
\end{itemize}

\section{El tabú del nombre}

El nombre \textbf{Nherath} rara vez se pronuncia en voz alta.

\begin{itemize}
  \item El pueblo llano lo llama \textbf{El Silente}, convencido de que decir su nombre atrae su mirada.
  \item Los clérigos utilizan títulos rituales para evitar invocarlo directamente.
  \item Los rebeldes lo nombran \textbf{El Olvidado}, creyendo que borrar su recuerdo es la única forma de destruirlo.
\end{itemize}

En los altares del culto puede leerse una inscripción repetida hasta la obsesión:

\begin{quote}
``Mientras alguien me tema, existiré.  
Mientras alguien me nombre, reinaré.''
\end{quote}

\section{Relación con la fe y la muerte}

Para Nherath, la muerte es un error de diseño divino.  
Los dioses tradicionales se alimentan de emociones frágiles, destinadas a apagarse con el tiempo.

El miedo, en cambio, perdura.

Bajo su dominio, los muertos no son simples cadáveres animados.  
Son \textbf{memorias vivas}: fragmentos conscientes atrapados entre mundos, obligados a rezar sin descanso para mantener su nombre resonando en el tejido de la realidad.

\section{Dominio actual}

Nherath reina desde las \textbf{Marismas del Sur}, un territorio donde su influencia distorsiona la realidad:

\begin{itemize}
  \item el agua se espesa y oscurece;
  \item el aire huele a hierro húmedo;
  \item las campanas dejan de sonar;
  \item el silencio adquiere peso.
\end{itemize}

El pueblo vive bajo una \textbf{teocracia del silencio}.  
Cada palabra pronunciada puede ser escuchada por su eco.

Sus servidores, los \textbf{Vigías del Silencio}, repiten su nombre únicamente en pensamiento durante los rituales nocturnos, y custodian templos donde los muertos \emph{rezan por los vivos}.

\section{La contradicción central}

Nherath es una paradoja encarnada:

\begin{itemize}
  \item Inmortal, pero mortalmente temeroso de ser olvidado.
  \item Dios en potencia, pero vacío por dentro.
  \item Eco eterno, pero sin voz propia.
\end{itemize}

Su poder crece con el miedo, pero cada generación teme menos y recuerda menos.  
Esto lo obliga a intensificar la represión y reforzar su culto.

En lo más profundo de su conciencia espectral existe una grieta imposible de cerrar:  
el terror de que algún día no quede nadie vivo para recordarlo.

\section{Nombres y alusiones}

\begin{center}
\begin{tabular}{ll}
\textbf{Cultura} & \textbf{Nombre usado} \\
\hline
Pueblo llano & El Silente \\
Clérigos & El Eco Eterno / El Señor del Último Recuerdo \\
Orcos del Norte & El Padre del Silencio \\
Rebelión del Alba & El Olvidado \\
Textos antiguos & Nherath de los Mil Ecos \\
\end{tabular}
\end{center}

\section{Consecuencias narrativas}

\begin{itemize}
  \item El nombre verdadero \textbf{Nherath} puede ser clave en rituales de confrontación o exorcismo.
  \item Pronunciarlo puede atraer su atención directa.
  \item Borrar su recuerdo es potencialmente su destrucción definitiva.
  \item Cantar su nombre con amor o compasión podría quebrar la coherencia de su fe basada en el miedo.
\end{itemize}

\section{Iconografía}

\begin{center}
\includegraphics[width=0.42\textwidth]{sol-muerto.png}
\end{center}
