\chapter{Historia}
\label{chap:historia}

La historia de Nherathia, antaño conocida como Eldharen, no es una línea clara de hechos, sino una superposición de recuerdos incompletos, silencios impuestos y versiones interesadas. Lo que sigue constituye una síntesis canónica destinada al Director de Juego, diferenciando los grandes periodos históricos sin pretender una cronología exacta.

\section{Antes del Reino: los Primeros Habitantes}

Antes de cualquier reino humano existió en la región una civilización no humana, conocida de forma imprecisa como los Primeros Habitantes. Su época pertenece al terreno de la leyenda y del mito: para el pueblo llano son apenas nombres fragmentarios en cuentos antiguos, y para el régimen actual, una etapa irrelevante o inexistente.

No hay constancia clara de su desaparición, solo de su ausencia. Dejaron tras de sí ruinas primigenias y vestigios de un conocimiento ajeno a la magia posterior: prácticas vinculadas a la memoria persistente, al silencio activo y a la obediencia ritualizada. Este legado no fue comprendido, sino parcialmente saqueado y malinterpretado por culturas humanas posteriores.

\section{Primeras Épocas Conocidas: la Ascensión de Nherath}

En un periodo impreciso de las primeras eras humanas surge la figura de Nherath. Las versiones populares discrepan sobre su origen, pero la interpretación más sólida lo sitúa como un elfo del sur: altivo, elitista y convencido de la inferioridad de los pueblos humanos.

Impulsado por la ambición y el desprecio hacia los límites impuestos por la tradición, Nherath buscó activamente el conocimiento prohibido de los Primeros Habitantes. A través de estos saberes alcanzó una forma de inmortalidad espectral, perdiendo progresivamente su naturaleza original.

Durante un tiempo dominó la región mediante el miedo y la obediencia forzada. Su gobierno no se sustentaba en leyes ni estructuras, sino en la presencia constante de su poder.

Finalmente surgió una figura recordada como el Primer Héroe —posiblemente de origen popular o noble— que logró unir a diversos grupos en una resistencia organizada. Este conflicto es conocido como la Guerra del Alba. Nherath fue derrotado, pero no destruido. Su esencia quedó fragmentada y ligada al mundo de forma incompleta, hecho que sería ignorado o minimizado por generaciones posteriores.

\section{La Era de los Reyes de Eldharen}

Tras la Guerra del Alba se fundó el reino de Eldharen. Sus reyes no gobernaban por derecho divino, sino por aclamación, siguiendo el precedente del Primer Héroe. Fue una era de prosperidad relativa, reconstrucción y expansión comercial.

Sin embargo, esta estabilidad se sostuvo sobre pactos implícitos de silencio. Ciertas ruinas no debían explorarse, determinados textos no debían estudiarse y el pasado anterior al reino no debía investigarse más allá de lo permitido. La memoria de Nherath fue deliberadamente erosionada, transformada en advertencia vaga o mito lejano.

Las crónicas oficiales posteriores idealizan esta era como un tiempo de equilibrio y razón, ocultando el miedo latente que la atravesaba y la fragilidad real de su paz.

\section{El Último Rey y la Conspiración de los Nobles}

Durante el reinado del Último Rey de Eldharen, una facción de nobles conspiró en secreto. Buscaban poder, estabilidad o trascendencia recurriendo a aquello que había sido derrotado pero no eliminado: Nherath.

El ritual tuvo éxito. Tras un periodo convulso, marcado por intrigas y tensiones internas, Nherath regresó plenamente al mundo y asesinó personalmente al rey. Este acto marcó la caída inmediata del reino. No hubo transición ordenada ni reconstrucción posible.

En los momentos finales del reino llegaron desde Faerûn tres figuras extraplanares: un monje tabaxi, un druida humano y una maga humana. Intentaron auxiliar al rey y detener a Nherath, pero fracasaron. Permanecieron en la región, intentando redimirse de su error, explorando ruinas primigenias y buscando formas de contener el daño causado.

Los objetos personales de los Guardianes quedaron alterados por las fuerzas que intentaron manipular, transformándose en reliquias. Con el tiempo, los tres desaparecieron. Su legado quedó fragmentado y sería posteriormente malinterpretado o instrumentalizado.

\section{El Retorno del Silente y la Teocracia}

Tras la muerte del Último Rey, Nherath consolidó su dominio. Ejecutó o forzó juramentos de lealtad a los nobles fieles al antiguo régimen y sustituyó las estructuras de poder por una élite alineada con su retorno. Aunque se retiró físicamente a su refugio en las Marismas de Nherath, su influencia se volvió omnipresente.

En pocos años comenzaron a manifestarse consecuencias crecientes: criaturas olvidadas reaparecieron, los viajes se volvieron peligrosos y la realidad misma pareció reaccionar a nombres, recuerdos y promesas. Inicialmente estos fenómenos carecían de explicación común.

Pasados dos o tres años, el pueblo empezó a referirse colectivamente a esta nueva condición del mundo como El Velo. Con el tiempo, el régimen teocrático apropiaría el término, reinterpretándolo como manifestación inevitable —o incluso sagrada— del poder del Silente.

Hace aproximadamente cincuenta o sesenta años, la teocracia se instauró de forma definitiva. Desde entonces, la historia ha sido reescrita, el pasado vigilado y la memoria convertida en instrumento de control. Recordar mal es un delito; recordar demasiado, una herejía.

