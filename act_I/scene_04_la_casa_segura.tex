\section{Escena IV --- La Casa Segura}

\begin{multicols*}{2}

\textbf{Tipo de escena:} Respiro breve, luto, entrega de información clave y asedio narrativo.

\begin{readaloud}
El túnel de huida no os lleva a un lugar limpio. Os escupe en un desagüe lateral, en un callejón lleno de basura y niebla cerca de los muelles. Lira, cojeando y cubierta de polvo, no os deja deteneros. Os guía por un laberinto de calles traseras hasta una curtiduría de aspecto abandonado.

Golpea la puerta con un ritmo específico: tres golpes, silencio, dos golpes.
La puerta se abre. El calor de un hogar y el olor a estofado os golpean. Dentro hay cinco personas armadas que se levantan de golpe, sonriendo con alivio... hasta que ven que Orim no está con vosotros.
\end{readaloud}

\subsubsection*{Localización: La Curtiduría (Base Célula 4)}

Un taller humilde con pieles colgadas que amortiguan el sonido exterior. Es el hogar de la célula de resistencia local.

\paragraph{El Luto}
El silencio cae sobre la sala como una losa. Lira no llora, pero su voz es dura como el pedernal cuando informa a sus compañeros.

\begin{displayquote}
\emph{"Se quedó en el muro. El Verdugo estaba allí."}
\end{displayquote}

Los otros rebeldes bajan la mirada, limpiando sus armas con movimientos nerviosos. No hay reproches hacia los PJ, solo una aceptación cansada. Saben que ese es el precio habitual.

\subsubsection*{Revelaciones y Lore}

Mientras los miembros de la célula aseguran las ventanas y ofrecen comida rápida a los PJ, Lira se sienta con ellos para explicar la situación. Es crucial que entiendan a qué se han enfrentado.

\paragraph{Sobre el enemigo invencible (El Vigía)}
Si los jugadores mencionan que sus armas no hacían daño al Vigía en la prisión, Lira asiente con gravedad y ofrece una información vital:

\begin{displayquote}
\emph{"Os enfrentasteis a un Alto Vigía dentro de los cimientos de la Casa del Silencio. Allí, sobre la piedra consagrada, el Velo los hace inmortales. Son extensiones del edificio."}
\end{displayquote}

Hace una pausa y os mira fijamente, asegurándose de que entendéis la diferencia.

\begin{displayquote}
\emph{"Pero fuera de sus templos... sangran. Recordadlo. Si os encontráis a uno en campo abierto, se les puede matar. Son monstruos, no dioses."}
\end{displayquote}

\paragraph{La Pared de los Mapas}
Lira señala una pared cubierta de papeles clavados unos sobre otros. Son mapas de Portimar y de la región de Nherathia, pero todos tienen tachaduras, correcciones y zonas en blanco.

\begin{displayquote}
\emph{"Intentamos dibujar el mundo, pero el Velo cambia lo que olvidamos. Calles que ayer existían, hoy son muro. Pueblos que nadie recuerda haber visitado..."}
\end{displayquote}

Arranca uno de los mapas más generales (El Mapa Incompleto) y lo pone sobre la mesa.

\begin{displayquote}
\emph{"Orim creía que este mapa mostraba el camino a las Ruinas del Norte, fuera de la influencia del Silente. Lleváoslo. Quizá vosotros podáis completarlo."}
\end{displayquote}

\subsubsection*{El Asedio Narrativo}

Apenas habéis guardado el mapa cuando la atmósfera cambia.
Uno de los vigías de la ventana se gira, pálido.
\emph{"Están aquí. La niebla se está espesando en la calle."}

No hay gritos de ataque, ni flechas rompiendo cristales. Simplemente, el sonido de la calle desaparece. Luego, un golpe lento y pesado en la puerta principal.
\emph{BUM... BUM...}

Lira desenfunda su arma. Ya no hay tristeza en su rostro, solo determinación.

\begin{displayquote}
\emph{"Nos han rastreado. Sabíamos que pasaría."}
\end{displayquote}

\paragraph{La Decisión}
Los rebeldes empiezan a mover muebles para bloquear la puerta. Saben que no van a ganar, solo van a comprar tiempo. Lira se gira hacia vosotros y os entrega una \textbf{Brújula de Latón} que pertenecía al anciano.

\begin{displayquote}
\emph{"Esta brújula no marca el Norte, marca los Ecos antiguos. Os guiará por las catacumbas hasta la salida de la ciudad."}
\end{displayquote}

Lira señala una trampilla oculta bajo las pieles de curtido.

\begin{displayquote}
\emph{"¡Idos! Nosotros retendremos a los Vigías aquí. Haced que la muerte de Orim valga la pena."}
\end{displayquote}

\subsubsection*{Huida a las Catacumbas y El Sellado}

La puerta principal empieza a astillarse bajo los golpes. Se oyen los zumbidos de la magia del Silente al otro lado. Lira os empuja hacia la oscuridad de la trampilla.

\emph{"¡Corred! ¡Y no miréis atrás!"}

Una vez estáis abajo, veis a Lira arrodillarse junto al marco de la trampilla. Saca un talismán de piedra gris, idéntico al que usó en los calabozos, y lo coloca sobre la madera.

\begin{displayquote}
\emph{"Cuando cierre esto, la casa olvidará que esta salida existe. No podréis volver. Y ellos no podrán seguiros."}
\end{displayquote}

Lira os dedica una última mirada, una mezcla de envidia y esperanza, y cierra la trampilla de golpe.

\paragraph{El Olvido}
Desde abajo, escucháis el crujido de la madera transformándose. Al alzar la antorcha, veis cómo la trampilla desaparece. La madera se vuelve piedra sólida, fusionándose con el techo del túnel.

Arriba, los sonidos del combate se cortan abruptamente, como si alguien hubiera apagado el mundo. Ahora solo hay silencio y oscuridad hacia delante.

\textbf{Comienza la Escena V — Las Catacumbas y la Huida Final.}

\end{multicols*}
