\section{Escena IV --- La Casa Segura}
\begin{multicols}{2}

\textbf{Tipo de escena:} Respiro tenso, revelación parcial, exposición del sistema del Silente.

\begin{readaloud}
El pasadizo desemboca en una trampilla baja, oculta tras una estantería rota.  
Cuando se cierra a vuestras espaldas, el ruido del combate desaparece por completo, como si nunca hubiera existido.

La estancia es pequeña, mal iluminada por una lámpara de aceite cubierta con tela oscura. Las paredes están reforzadas con vigas antiguas y el suelo es de tierra apisonada. No hay símbolos religiosos. No hay ventanas.

Por primera vez desde vuestra llegada a Portimar, nadie os observa.

O al menos, nadie visible.
\end{readaloud}

\subsubsection*{Localización: La Casa Segura}

La Casa Segura es una vivienda olvidada en los niveles bajos de la ciudad, demasiado humilde para atraer vigilancia directa y demasiado sólida para haberse derrumbado. Ha sido usada y abandonada muchas veces.

\paragraph{Aspecto general}

Habitación principal con una mesa baja, bancos de madera, sacos de grano vacíos y una trampilla que conduce al pasadizo. Mapas incompletos clavados en una pared; marcas de tiza en el suelo; un pequeño brasero apagado.

\paragraph{Sensaciones}

Aire viciado pero estable; silencio espeso; olor a aceite rancio y humedad antigua. El corazón tarda en bajar el ritmo. La sensación de persecución no desaparece del todo.

\paragraph{Indicio del Velo Nocturno}

Aquí, el Velo no se manifiesta de forma directa. Precisamente por eso el lugar resulta inquietante: es un vacío, un punto ciego en la ciudad.

\subsubsection*{PNJ relevantes}

\begin{npcsheet}[Miembro de la Resistencia]{Orim Velsh}
  \appearance{Hombre mayor, ciego, rostro surcado de cicatrices finas; viste ropas de artesano gastadas.}
  \voice{Serena, grave; habla despacio, como si escuchara algo más que a sus interlocutores.}
  \motivation{Mantener viva la memoria real de Portimar; proteger a quienes aún no han sido borrados.}
  \secrets{Percibe el Velo sin verlo; cree que el Silente no necesita presencia física para gobernar.}
\end{npcsheet}

\begin{npcsheet}[Miembro de la Resistencia]{Lira Doven}
  \appearance{Ropa de cuero manchada, gesto tenso, manos aún firmes pese al cansancio.}
  \voice{Directa, contenida; evita rodeos innecesarios.}
  \motivation{Sacar a los marcados de la ciudad antes de que desaparezcan.}
  \secrets{No confía plenamente en nadie recién rescatado; espera traiciones.}
\end{npcsheet}

\subsubsection*{Desarrollo de la escena}

Lira cierra la trampilla y apoya la espalda contra la pared durante unos segundos. Luego se recompone y hace una seña para que os sentéis.

Orim Velsh permanece sentado en un banco, las manos apoyadas en un bastón corto. No os mira, pero gira la cabeza exactamente hacia quien habla.

\paragraph{Primeras palabras}

Orim rompe el silencio:

\begin{displayquote}
\emph{Si habéis llegado hasta aquí, es que aún recordáis quiénes sois. Eso ya os pone en peligro.}
\end{displayquote}

Lira añade, sin suavizar el tono:

\begin{displayquote}
\emph{No estáis a salvo. Solo estáis fuera de la vista.}
\end{displayquote}

\paragraph{Lo que revela la Resistencia}

Sin dar nombres prohibidos ni explicaciones completas, Orim y Lira exponen una verdad fragmentada:

\begin{itemize}
  \item Portimar no se gobierna solo con guardias, sino con \textbf{olvido}.
  \item El Silente no necesita aparecer; su control se ejerce mediante registros, rituales y memoria dirigida.
  \item Las desapariciones no son castigos: son \textbf{correcciones}.
  \item La Resistencia no busca derrocar el sistema, sino \textbf{sobrevivir fuera de él}.
\end{itemize}

\paragraph{Advertencia}

Orim alza ligeramente el bastón:

\begin{displayquote}
\emph{El error más común es creer que el Velo persigue. No lo hace. Espera.}
\end{displayquote}

\subsubsection*{Interacción con los PJ}

Permite que los jugadores hagan preguntas. Las respuestas nunca son completas.

\paragraph{¿Quién es el Silente?}

\begin{displayquote}
\emph{Una voz que ya no necesita garganta.}
\end{displayquote}

\paragraph{¿Por qué ellos fueron arrestados?}

\begin{displayquote}
\emph{Porque alguien os escuchó antes de que debiera.}
\end{displayquote}

\paragraph{¿Se puede luchar contra el Velo?}

Lira responde primero:

\begin{displayquote}
\emph{No. Solo se puede aprender dónde no mirar.}
\end{displayquote}

\subsubsection*{Descanso tenso}

Los PJ pueden recuperar el aliento, curar heridas leves o recomponerse. Sin embargo, el descanso nunca es completo.

Durante este tiempo, introduce detalles inquietantes:

\begin{itemize}
  \item Uno de los PJ cree oír pasos sobre la casa, pero nadie más los oye.
  \item La lámpara parpadea sin motivo aparente.
  \item Orim murmura nombres que no reconoce, luego guarda silencio.
\end{itemize}

\subsubsection*{Mecánicas opcionales (RMU)}

\paragraph{Recuperación limitada}

Permite recuperar Fatiga leve o realizar curaciones menores, pero no elimina penalizadores mentales relacionados con miedo o tensión.

\paragraph{Empatía / Influencia}

Los PJ pueden intentar ganarse la confianza de la Resistencia. El éxito mejorará el trato en escenas posteriores; el fracaso refuerza la desconfianza de Lira.

\paragraph{Percepción}

Detectar sonidos o presencias ambiguas. El DJ decide si son reales o producto del estrés.

\subsubsection*{Ganchos y detalles del Velo Nocturno}

\begin{itemize}
  \item Orim afirma que algunos mapas cambian con el tiempo.
  \item La Casa Segura ha sido abandonada antes “porque empezó a recordar”.
  \item Lira observa a uno de los PJ como si intentara decidir si ya es demasiado tarde para él.
\end{itemize}

\subsubsection*{Cierre de la escena}

Tras un breve intercambio, Lira recoge su equipo y señala una segunda salida, oculta bajo una losa de piedra.

\begin{displayquote}
\emph{No podemos quedar aquí. El silencio dura poco en Portimar.}
\end{displayquote}

Orim asiente lentamente.

\begin{displayquote}
\emph{Bajo la ciudad hay caminos antiguos. Algunos aún no han sido escritos.}
\end{displayquote}

La losa se desliza, revelando un descenso estrecho y oscuro.

\textbf{Comienza la Escena V — Las Catacumbas y la Huida Final.}

\end{multicols}
