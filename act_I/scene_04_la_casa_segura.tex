\section{Escena IV --- La Casa Segura}

\begin{multicols}{2}

\textbf{Tipo de escena:} Respiro breve, Luto, Exposición de Lore y Huida Narrativa.

\begin{readbox}
El túnel de huida no os lleva a un lugar limpio. Os escupe en un desagüe lateral, en un callejón lleno de basura y niebla cerca de los muelles.

Lira, cojeando y cubierta de polvo de piedra, no os deja deteneros. Os guía por un laberinto de calles traseras hasta una curtiduría de aspecto abandonado. El olor a orina y productos químicos es intenso, perfecto para ocultar el olor a sangre.

Golpea la puerta con un ritmo específico: tres golpes, silencio, dos golpes.

La puerta se abre. El calor de un hogar y el olor a estofado os golpean. Dentro hay cinco personas armadas con cuero remendado que se levantan de golpe, sonriendo con alivio... hasta que ven que Orim no está con vosotros.
\end{readbox}

\subsubsection*{El Precio de la Huida}

El silencio cae sobre la sala como una losa. Lira no llora, pero su voz es dura como el pedernal cuando informa a sus compañeros.

\emph{"Se quedó en el muro. El Verdugo estaba allí."}

Los otros rebeldes bajan la mirada. Una mujer mayor os acerca cuencos de estofado caliente y vendas.

\begin{mechanics}{Recuperación (First Aid)}
Este es el único momento de descanso antes del final del Acto.
\begin{itemize}
    \item \textbf{Curación:} Los PJs pueden realizar tiradas de \textbf{Primeros Auxilios (First Aid) - Medio}.
    \item \textbf{Efecto:} Detiene sangrados y recupera \textbf{10 PV} si tienen éxito.
    \item \textbf{Fatiga:} Comer el estofado recupera \textbf{20 Puntos de Fatiga} instantáneamente.
\end{itemize}
\end{mechanics}

\subsubsection*{La Calma tras la Tormenta}

Mientras coméis, Lira se limpia la sangre de la cara con un trapo húmedo. Por primera vez, os mira no como a carga, sino como a personas. Se sienta frente a vosotros, dejando su espada corta sobre la mesa, al alcance de la mano.

\begin{npcsummary}[Líder de la Célula 4]{Lira Doven}
    \appearance{Se ha quitado el casco. Tiene el pelo corto, pegado por el sudor, y ojos oscuros que han visto demasiado. A pesar de su juventud, tiene las manos llenas de callos de usar armas.}
    \voice{Directa, pragmática, pero con un tono de cansancio profundo. No endulza la verdad.}
    \motivation{Asegurar que la muerte de Orim sirva para salvar la información de Yenar.}
\end{npcsummary}

\begin{readbox}
\emph{"Soy Lira Doven. Siento la bienvenida, pero en Portimar las cortesías murieron hace tiempo."}

Señala los muros cubiertos de pieles.

\emph{"Estáis en una casa segura de la Resistencia. O lo que queda de ella. Fuera de estas paredes, el Velo lo cubre todo. Si estáis aquí, es porque el Silente os quería borrar... y porque Orim pensó que valíais la pena."}
\end{readbox}

\subsubsection*{Lo que Lira Sabe (Información para los Jugadores)}

Es probable que los jugadores tengan muchas preguntas. Lira responde con honestidad brutal. Utiliza estos puntos para guiar la conversación:

\begin{itemize}
    \item \textbf{¿Dónde estamos?} "En Portimar. Antes era una ciudad portuaria; ahora es una granja de almas. El Velo... esa niebla que visteis... se alimenta de recuerdos. Si os quedáis mucho tiempo, olvidaréis vuestro nombre. Si os quedáis para siempre, os convertís en *Ecos*."
    \item \textbf{¿Por qué somos importantes?} "Porque el Velo intentó borrar a Yenar Strig y falló con vosotros. Vuestras mentes retienen lo que él sabía. Sois el único mapa que nos queda."
    \item \textbf{Sobre los Enemigos (Regla de Oro):} Si preguntan por qué sus ataques no funcionaban en la prisión: "Dentro de la Casa del Silencio, sobre piedra consagrada, los Vigías son inmortales. Son parte del edificio. Pero fuera... fuera sangran. Recordadlo: en campo abierto se les puede matar."
    \item \textbf{¿Y ahora qué?} "Ahora corréis. El Velo sabe que estáis aquí. Esta casa segura no lo será por mucho tiempo."
\end{itemize}

\subsubsection*{Las Herramientas del Camino}

Lira señala una pared cubierta de papeles clavados unos sobre otros. Son mapas de Portimar y de la región, llenos de tachaduras.

\begin{readbox}
\emph{"Intentamos dibujar el mundo, pero el Velo cambia lo que olvidamos. Calles que ayer existían, hoy son muro."}

Arranca un mapa general y os lo pone sobre la mesa, junto a una extraña brújula de latón que saca de su bolsillo.
\end{readbox}

\begin{itembox}{El Legado de la Resistencia}
Lira os entrega dos objetos vitales para la campaña:

\begin{itemize}
    \item \textbf{El Mapa Incompleto:} Muestra la región al norte de Portimar. Hay grandes áreas en blanco donde la tinta se ha desvanecido, pero marca una ruta hacia las montañas: \textit{"Los Pasos del Gigante"}.
    \item \textbf{La Brújula de Ecos:} Una brújula de latón oxidado.
    \begin{itemize}
        \item \textbf{Efecto:} No señala el Norte magnético. Señala la mayor concentración de magia antigua o "Ecos" cercana.
        \item \textbf{Ahora mismo:} La aguja gira violentamente y apunta hacia el suelo, hacia las catacumbas.
    \end{itemize}
\end{itemize}
\end{itembox}

\subsubsection*{El Asedio}

Apenas guardáis los objetos, la atmósfera cambia. La luz de las velas parpadea y se vuelve verdosa. Uno de los vigías de la ventana se gira, pálido.

\emph{"La niebla se está espesando. Ya no veo la calle."}

No hay gritos de ataque. Simplemente, el sonido de la ciudad desaparece. Luego, un golpe lento y demoledor en la puerta principal hace temblar el edificio.

\textbf{BUM... BUM...}

\begin{readbox}
Lira desenfunda su arma. Ya no hay tristeza en su rostro, solo determinación suicida.

\emph{"Nos han rastreado. Sabíamos que pasaría. ¡Moveos! Orim compró vuestra salida de la prisión; nosotros compraremos vuestra salida de la ciudad."}
\end{readbox}

Lira patea una alfombra de pieles, revelando una trampilla de madera oscura.

\emph{"Las catacumbas viejas. La brújula os guiará. ¡Idos!"}

\subsubsection*{El Sellado del Camino}

Si los jugadores intentan quedarse a luchar, Lira los empuja físicamente hacia el agujero.

\emph{"¡No podéis ganar esto! ¡Si morís aquí, Yenar murió para nada!"}

Una vez estáis abajo, en la oscuridad húmeda del túnel, veis a Lira arrodillarse junto al marco de la trampilla. Saca un talismán de piedra gris, idéntico al que usó en los calabozos.

\begin{readbox}
\emph{"Cuando cierre esto, la casa olvidará que esta salida existe. No podréis volver. Corred hacia el norte... y haced que sangre."}

Lira cierra la trampilla de golpe.
\end{readbox}

Desde abajo, escucháis el crujido antinatural de la madera transformándose en piedra. La trampilla desaparece, fusionándose con el techo del túnel como si nunca hubiera existido. Arriba, los sonidos del combate comienzan... y luego se cortan abruptamente cuando el Velo aísla la realidad.

Estáis solos en la oscuridad.

\textbf{Continuar a la Escena V — La Huida Final.}

\end{multicols}
