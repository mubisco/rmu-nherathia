\section{Escena V --- La Huida Final}

\begin{multicols*}{2}

\textbf{Tipo de escena:} Dungeon Crawl, Terror (sigilo/combate), Revelación mística y Acción final.

\begin{readaloud}
La trampilla se ha fusionado con el techo. No hay vuelta atrás.
El aire aquí abajo es frío y huele a agua estancada y tiempo muerto. La luz de vuestra antorcha apenas empuja la oscuridad unos metros; más allá, las sombras parecen tener peso propio.

La Brújula de Orim vibra en vuestras manos. Su aguja no señala una dirección fija, sino que oscila como si buscara un latido en la profundidad.
\end{readaloud}

\subsubsection*{Localización: El Osario de los Sin Nombre}

Antiguas catacumbas pre-imperiales. Techos bajos, nichos en las paredes llenos de huesos que han sido limados hasta borrar cualquier rasgo distintivo.

\paragraph{Sensaciones}
Silencio absoluto roto solo por el goteo de agua; sensación de ser observado por las cuencas vacías de los cráneos; frío húmedo.

\subsubsection*{Encuentro: Los Penitentes}

Al cruzar una sala amplia con columnas rotas, los PJ escuchan un arrastrar de cadenas y respiraciones rasposas.
Son los \textbf{Penitentes}: prisioneros que murieron aquí abajo y el Velo no les permitió irse.
\begin{itemize}
    \item \textbf{Descripción:} Humanoides escuálidos. Tienen máscaras de hierro oxidadas \emph{fundidas} a la piel de la cara. No tienen ojos.
    \item \textbf{Mecánica de Terror:} Son ciegos. Se guían por el sonido. Si los PJ superan tiradas de \textbf{Sigilo}, pueden evitarlos o atacar con ventaja (Sorpresa). Si hacen ruido, los Penitentes cargan con un alarido metálico.
\end{itemize}

\begin{npcsheet}[Muertos Vivientes Menores][images/penitente.jpg]{Penitentes del Velo}
\appearance{Piel grisácea, máscaras de hierro oxidadas, dedos convertidos en garras.}
\motivation{Destruir cualquier sonido que rompa su silencio.}
\secrets{Son sensibles a la luz fuerte; el fuego les aterroriza.}
\end{npcsheet}

\subsubsection*{El Hallazgo: El Cadáver Borrado}

Tras el encuentro, en un recodo del túnel donde se filtra agua salada del puerto, encuentran un cadáver "fresco". No está podrido, pero su piel es lisa, como cera derretida. No tiene cara.

En sus manos, aferra una bolsa de cuero encerado. Es el botín.

\paragraph{El Diario del Fugitivo}
El diario está escrito con letra temblorosa. Las últimas páginas revelan la verdad:
\begin{displayquote}
\emph{"Llegué al agua. Creí que era libre. Pero al tocar el mar... empecé a deshacerme. El agua salada no acepta lo que el Silente ha tocado. No podemos salir por mar. Nos disolvemos. Dioses, mi mano... ya no tiene dedos."}
\end{displayquote}

\paragraph{Tesoro (Loot)}
Dentro de la bolsa encuentran equipo que el fugitivo robó antes de morir:
\begin{itemize}
    \item \textbf{Daga de Acero Estelar (Starfall):} Una hoja oscura, fría al tacto. Material meteórico superior (+10 a la rotura, golpea como mágica).
    \item \textbf{2 Pociones de Curación:} Líquido ámbar espeso. (Recuperan 1d10 PV).
    \item \textbf{Runa de Hoja (Bladerune):} Un papel con un glifo complejo. Al frotarlo contra un arma, esta se vuelve mágica (+10) durante 10 minutos.
    \item \textbf{Monedas Antiguas:} Una bolsa con monedas de plata acuñadas con leones, de una era anterior al Velo (Valor: 50 monedas de plata).
\end{itemize}

\subsubsection*{La Profecía de Yenar}

Cuando los jugadores comprenden que están atrapados (no pueden volver a la ciudad, no pueden huir por mar), la desesperación activa el "regalo" de Yenar.

La \textbf{Brújula de Orim} emite un chasquido. La aguja se clava violentamente hacia el \textbf{Norte}.
En la mente de todos los PJ, resuena la voz de Yenar Strig, clara y lúcida, sin la locura de la celda:

\begin{displayquote}
\emph{"El agua lava el nombre hasta borrarlo. Solo la piedra recuerda.\\
Buscad la Corona de Fuego Pálido donde el norte se rompe.\\
Allí... el silencio tiene grietas.\\
Allí... podréis recuperar vuestros nombres."}
\end{displayquote}

\textbf{El Click:} Ahora tienen un destino (Norte/Ruinas) y una motivación (Curar su maldición antes de desvanecerse como el cadáver).

\subsubsection*{Clímax: La Posta Militar}

El túnel asciende y termina en una rejilla oxidada que da al exterior.
Salen a la superficie en una noche cerrada y neblinosa. Están en los establos traseros de una \textbf{Posta de la Vigilia}, a las afueras de Portimar.

\paragraph{Situación}
\begin{itemize}
    \item \textbf{Objetivo:} Robar caballos y huir hacia el norte antes de que den la alarma.
    \item \textbf{Oposición:} Dos guardias humanos (aburridos, jugando a los dados) y un \textbf{Mastín del Velo} (un perro de presa deforme) durmiendo cerca.
\end{itemize}

\paragraph{Resolución}
Los jugadores pueden optar por:

\begin{itemize}
    \item \textbf{Sigilo:} Robar los caballos sin despertar al perro.
    \item \textbf{Fuerza:} Eliminar a los guardias rápidamente (tienen las armas y la sorpresa).
    \item \textbf{Distracción:} Soltar a los caballos para crear el caos y montar sobre la marcha.
\end{itemize}

\subsubsection*{Cierre del Capítulo}

Con los caballos (o corriendo a pie si todo sale mal), los PJ se adentran en la niebla espesa de Nherathia. A sus espaldas, las campanas de alarma de Portimar empiezan a sonar, pero el sonido se ahoga en la bruma.

Miráis al Norte. La brújula brilla con luz tenue. El viaje hacia lo desconocido ha comenzado.

\textbf{FIN DEL CAPÍTULO I}

\end{multicols*}
