\section{Escena V --- La Huida Final}

\begin{multicols}{2}

\textbf{Tipo de escena:} Dungeon Crawl, Terror (sigilo), Loot y Huida Final.

\begin{readbox}
La trampilla se ha fusionado con el techo de piedra. No hay vuelta atrás.

El aire aquí abajo es frío y huele a agua estancada y tiempo muerto. La luz de vuestra antorcha apenas empuja la oscuridad unos metros; más allá, las sombras parecen tener peso propio.

La \textbf{Brújula de Ecos} vibra en vuestras manos. Su aguja no señala una dirección fija, sino que oscila violentamente, como si buscara un latido en la profundidad.
\end{readbox}

\subsubsection*{El Osario de los Sin Nombre}

Estáis en unas catacumbas pre-imperiales. Los techos son bajos y las paredes están forradas de nichos llenos de huesos que han sido limados hasta borrar cualquier rasgo distintivo.

Al cruzar una sala amplia con columnas rotas, escucháis un sonido metálico: cadenas arrastrándose sobre la piedra y respiraciones rasposas.

\begin{mechanics}{Desafío: Los Penitentes (Sigilo)}
Los \textbf{Penitentes} son prisioneros muertos que el Velo no dejó marchar. Tienen máscaras de hierro fundidas a la piel. Son ciegos, pero su oído es sobrenatural.

\textbf{Maniobra Enfrentada:}
\begin{itemize}
    \item \textbf{PJs (Activos):} Tirada de \textbf{Acechar (Stalking)}.
    \item \textbf{Penitentes (Pasivos):} Tirada de \textbf{Percepción (Bono +50 por Oído agudo)}.
\end{itemize}

\textbf{Modificadores:}
\begin{itemize}
    \item \textbf{Oscuridad:} +20 a Acechar (ellos no ven).
    \item \textbf{Silencio:} Si los jugadores no hablan ni llevan armaduras pesadas ruidosas (+10).
\end{itemize}

\textbf{Fallo:} Los Penitentes aúllan y cargan. (Ver \textit{Encuentro Táctico V-A}).
\end{mechanics}

\subsubsection*{El Hallazgo: El Cadáver Borrado}

En un recodo del túnel, donde se filtra agua salada del puerto, encontráis un cadáver "fresco". Su piel es lisa, como cera derretida; no tiene rostro. En sus manos aferra una bolsa de cuero encerado.

Dentro hay un diario con las últimas páginas legibles:

\begin{displayquote}
\emph{"Llegué al agua. Creí que era libre. Pero al tocar el mar... empecé a deshacerme. El agua salada no acepta lo que el Silente ha tocado. No podemos salir por mar. Nos disolvemos. Dioses, mi mano... ya no tiene dedos."}
\end{displayquote}

\begin{itembox}{El Botín del Fugitivo}
El fugitivo robó equipo de la guarnición antes de huir.
\begin{itemize}
    \item \textbf{Daga de Acero Estelar (Starfall):} Hoja negra, fría al tacto. Material meteórico. Cuenta como mágica y otorga \textbf{+10 BO} intrínseco.
    \item \textbf{2x Viales de Curación (Heal V):} Líquido ámbar y denso.
    \begin{itemize}
        \item \textbf{Efecto:} Beberla (2 AP) lanza instantáneamente \emph{Heal V}. Recupera \textbf{25 PV} fijos
    \end{itemize}
    \item \textbf{Aceite de Afilado (Lotion Potion):} Un frasco de cristal con un aceite grisáceo.
    \begin{itemize}
        \item \textbf{Mecánica:} Se aplica sobre un arma (4 Asaltos / 1 min).
        \item \textbf{Efecto:} Imbuye el hechizo \emph{Weapon II}. El arma se vuelve mágica y gana \textbf{+10 BO} durante 24 horas (o hasta que se limpie el aceite).
    \end{itemize}
    \item \textbf{Bolsa de Monedas:} 50 monedas de plata imperiales antiguas.
\end{itemize}
\end{itembox}

\subsubsection*{La Revelación del Norte}

La desesperación de saber que no podéis huir por mar activa el último regalo de la resistencia. La \textbf{Brújula de Ecos} emite un chasquido metálico y deja de oscilar.

La aguja se clava, rígida y brillante, apuntando hacia una única dirección: el \textbf{Norte}.

No hay voces ni profecías. Solo una certeza fría: el Sur es la prisión, el Este y Oeste son el mar que disuelve, y el Norte... el Norte es lo desconocido, pero es el único camino que la aguja permite.

\subsubsection*{Clímax: La Posta Militar}

El túnel asciende hasta una rejilla oxidada. Al abrirla, salís a la superficie en una noche cerrada y neblinosa.

Estáis en los establos traseros de una \textbf{Posta de la Vigilia}, a las afueras de Portimar. El aire es fresco, pero huele a peligro.

Hay tres caballos ensillados. Dos guardias juegan a los dados sobre un barril, y un enorme \textbf{Mastín del Velo} duerme cerca de la puerta. Tenéis que robar los caballos para llegar a las montañas.

\textit{Id al \textbf{Encuentro Táctico V-B} para resolver la huida.}

\subsubsection*{Cierre del Capítulo I}

Con los caballos al galope (o corriendo hacia la espesura), dejáis atrás las campanas de alarma de Portimar.

Miráis al Norte. La brújula brilla con luz tenue. La ciudad prisión queda atrás, pero el Velo se extiende infinito ante vosotros. La guerra por vuestra memoria ha comenzado.

\textbf{FIN DEL ACTO I.}

\end{multicols}

\newpage

% --- ENCUENTRO OPCIONAL ---

\begin{tacticalscene}{Catacumbas: Los Penitentes (V-A)}

\textbf{Nota para el GM:}
Este combate solo ocurre si fallaron la tirada de sigilo en las catacumbas.

\textbf{Configuración:}
\begin{itemize}
    \item \textbf{Dimensiones:} Pasillo estrecho de piedra húmeda. 3m de ancho x 15m de largo (2x10 casillas).
    \item \textbf{Iluminación:} Oscuridad Total (Penalizadores de combate a ciegas si no usan antorcha).
    \item \textbf{Enemigos:} 1 Penitente por Jugador.
\end{itemize}

\tcbline

\textbf{ENEMIGOS}

\begin{statblock}[images/penitente.jpg]{Penitente del Velo}{Nvl 2 (Undead) | PV 35 | AT 1 | DB 10}
    \textbf{Características:} Ciegos (inmunes a ilusiones visuales, -50 a combatir en silencio total). Se guían por el sonido.
    \textbf{Ataques:}
    \begin{itemize}
        \item \textbf{Garras Oxidadas (Melee):} +40 OB.
        \\ \textit{Tabla:} \textbf{Claw}. \textit{Críticos:} Slash (S).
        \item \textbf{Presa del Ahogado:} Si hacen crítico, inician Presa (\textit{Grapple}) automáticamente.
    \end{itemize}
\end{statblock}

\end{tacticalscene}

% --- ENCUENTRO CLÍMAX ---
\newpage

\begin{tacticalscene}{La Posta Militar (V-B)}

\textbf{Configuración del Escenario:}
\begin{itemize}
    \item \textbf{Dimensiones:} Patio de establos cercado. 15m x 15m (10x10 casillas).
    \item \textbf{Terreno:} Paja y barro (Normal). Vallas de madera (Cobertura ligera).
    \item \textbf{Objetivo:} Montar a caballo (Maniobra de Montar - Medio) y salir por el Norte (Borde superior del mapa).
\end{itemize}

\imagerim{images/posta.png}

\tcbline

\textbf{ESCALADO DEL ENCUENTRO}
\begin{itemize}
    \item \textbf{Guardias de la Posta:} 1 por Jugador.
    \item \textbf{Mastín del Velo:} 1 por Grupo (Si hay 5+ jugadores, añadir un segundo Mastín).
\end{itemize}

\tcbline

\textbf{ENEMIGOS}

\begin{statblock}[images/guardia_posta.png]{Guardia de la Posta}{Nvl 2 (Fighter) | PV 45 | AT 9 | DB 20}
    \textbf{Equipo:} Armadura de cuero endurecido, Lanza corta.
    \textbf{Táctica:} Uno intenta bloquear la salida, el resto ataca a los PJs que intenten montar. Si uno queda libre, corre a tocar la campana (3 asaltos).
    \textbf{Ataques:}
    \begin{itemize}
        \item \textbf{Lanza (Melee):} +55 OB.
        \\ \textit{Tabla:} \textbf{Spear}. \textit{Críticos:} Puncture (P).
    \end{itemize}
\end{statblock}

\begin{statblock}[images/mastin_velo.png]{Mastín del Velo}{Nvl 3 (Animal) | PV 60 | AT 3 | DB 30}
    \textbf{Descripción:} Perro de presa enorme, piel negra y ojos lechosos.
    \textbf{Habilidad Especial:} \textbf{Aullido Paralizante} (1/día). Todos en 10m deben pasar RR Física o quedar \textbf{Aturdidos} 1 asalto.
    \textbf{Ataques:}
    \begin{itemize}
        \item \textbf{Mordisco (Melee):} +60 OB.
        \\ \textit{Tabla:} \textbf{Bite}. \textit{Tamaño:} Medium.
    \end{itemize}
\end{statblock}

\end{tacticalscene}
