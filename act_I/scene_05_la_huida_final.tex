\section{Escena V --- Las Catacumbas y la Huida Final}
\begin{multicols}{2}

\textbf{Tipo de escena:} Exploración opresiva, persecución velada, cierre del capítulo.

\begin{readaloud}
El descenso es estrecho y abrupto. La losa se cierra tras vosotros con un roce de piedra antigua, y la luz de la lámpara apenas logra arañar la oscuridad.

El aire cambia de inmediato: más frío, más seco, cargado con un olor viejo, como polvo y hueso. Cada paso resuena demasiado, multiplicándose en ecos que no siempre coinciden con vuestros movimientos.

Bajo Portimar, la ciudad se vuelve silencio.
\end{readaloud}

\subsubsection*{Localización: Las Catacumbas Antiguas}

Las catacumbas preceden al Templo-Fortaleza y a gran parte de la ciudad actual. Son restos de estructuras olvidadas, criptas reutilizadas y túneles excavados a lo largo de generaciones.

\paragraph{Aspecto general}

Galerías de piedra irregular, muros cubiertos de marcas erosionadas, nichos vacíos o sellados con losas agrietadas. En algunos tramos, el suelo desciende bruscamente o está cubierto de restos de derrumbes antiguos.

\paragraph{Sensaciones}

Frío constante; aire inmóvil; sensación de profundidad creciente; la impresión persistente de estar siendo observados desde lugares que la luz no alcanza.

\paragraph{Indicio del Velo Nocturno}

Los ecos se repiten con ligeras variaciones, como si algo los imitara. En ocasiones, los sonidos parecen recordar pasos anteriores.

\subsubsection*{Desarrollo de la escena}

El avance no es rápido. Las catacumbas obligan a moverse con cuidado, a elegir rutas y a confiar en indicaciones incompletas.

Lira avanza en cabeza durante parte del trayecto, pero pronto cede el liderazgo a los PJ: la Resistencia conoce salidas, no caminos seguros.

\paragraph{Señales del pasado}

Durante el recorrido, los PJ pueden encontrar:

\begin{itemize}
  \item Inscripciones antiguas parcialmente borradas.
  \item Restos de ofrendas secas en nichos vacíos.
  \item Símbolos circulares grabados y luego martilleados hasta quedar irreconocibles.
\end{itemize}

Nada se explica. Todo sugiere que la ciudad ha olvidado deliberadamente.

\paragraph{Presencia en la oscuridad}

En varios momentos, introduce la sensación de persecución:

\begin{itemize}
  \item Pasos lejanos que se detienen cuando el grupo se detiene.
  \item Sombras que parecen alargarse cuando la luz se aleja.
  \item La sensación de que alguien recuerda el camino mejor que ellos.
\end{itemize}

No es necesario mostrar un perseguidor visible. El peligro es la \textbf{atención}.

\subsubsection*{Desafíos durante la huida}

La escena puede resolverse como una secuencia de desafíos encadenados.

\paragraph{Orientación}

Elegir rutas entre bifurcaciones.
\begin{itemize}
  \item Éxito: progreso fluido.
  \item Fracaso: retrocesos, bucles, aumento de tensión.
\end{itemize}

\paragraph{Sigilo}

Evitar ser detectados por patrullas subterráneas o presencias indefinidas.
\begin{itemize}
  \item Un fallo no implica combate inmediato, sino cercanía creciente del peligro.
\end{itemize}

\paragraph{Resistencia mental}

El peso del silencio y los ecos exige tiradas para no perder concentración.
\begin{itemize}
  \item El fracaso puede implicar Fatiga mental o penalizadores temporales.
\end{itemize}

\subsubsection*{Mecánicas opcionales (RMU)}

\paragraph{Percepción}

Detectar cambios sutiles en el entorno: ecos anómalos, corrientes de aire, vibraciones en la piedra.

\paragraph{Miedo}

Ante la certeza de que algo los sigue sin mostrarse. El fallo genera penalizadores temporales o reacciones instintivas.

\paragraph{Fatiga}

Cada tramo fallido o demora innecesaria puede acumular Fatiga leve.

\subsubsection*{Ganchos y detalles del Velo Nocturno}

\begin{itemize}
  \item Un PJ reconoce una marca que no recuerda haber visto antes.
  \item Una voz repite una frase dicha al inicio del capítulo.
  \item Un túnel parece cerrarse tras el grupo, aunque nadie oye derrumbes.
\end{itemize}

\subsubsection*{La salida}

Finalmente, el túnel asciende.

Un aliento de aire fresco rompe el silencio. La humedad cambia. El eco se debilita.

\begin{readaloud}
Una rejilla oxidada cede con esfuerzo. Más allá, la noche de Portimar se abre ante vosotros desde los barrios bajos, lejos del muelle y del Templo-Fortaleza.

La ciudad sigue allí. Oscura. Vigilante.

Pero por primera vez desde vuestra llegada, no estáis dentro de ella.
\end{readaloud}

\subsubsection*{Cierre del capítulo}

Lira se detiene junto a la salida y os observa uno a uno.

\begin{displayquote}
\emph{Recordad esto: Portimar no persigue a quienes huyen. Persigue a quienes permanecen.}
\end{displayquote}

Orim no ha descendido con vosotros. Su voz resuena solo una vez, desde algún lugar imposible de precisar:

\begin{displayquote}
\emph{Mientras sepáis quiénes sois, el Velo aún no ha ganado.}
\end{displayquote}

La rejilla se cierra tras vosotros.

\textbf{Fin del Capítulo I — Sombras de Portimar.}

\end{multicols}
