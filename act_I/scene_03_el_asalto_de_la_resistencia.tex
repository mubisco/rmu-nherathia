\section{Escena III --- El Asalto de la Resistencia}

\begin{multicols*}{2}

\textbf{Tipo de escena:} Acción caótica, ruptura del orden, primer contacto con la Resistencia.

\begin{readaloud}
El estruendo no es una explosión limpia. Es un golpe sordo que recorre la piedra como un latido violento. El suelo vibra bajo vuestros pies. Polvo antiguo cae del techo y la antorcha del pasillo parpadea, casi apagándose.

Durante un instante, nadie grita. Luego llegan los sonidos: metal contra piedra, órdenes cortadas a mitad, un alarido ahogado que se extingue demasiado rápido. La cerradura de vuestra celda salta con un chasquido seco.

La puerta se abre de golpe. Una figura encapuchada entra con urgencia, ignorándoos por completo, y se arrodilla junto al anciano en el rincón.
\end{readaloud}

\subsubsection*{Localización: Calabozos inferiores}

Los pasillos forman un entramado confuso de corredores estrechos. No están pensados para el tránsito habitual, sino para movimientos discretos.

\paragraph{Sensaciones}
Calor repentino tras el frío de la celda; olor a humo mojado; el eco constante de pasos y choques.
\paragraph{Indicio del Velo}
En algunos tramos, los sonidos llegan antes que los hechos que los producen.

\subsubsection*{Desarrollo de la escena}

La Resistencia ha emergido en un \textbf{pozo de suministros} cercano, "borrando" la roca mediante un antiguo ritual de olvido. Han reducido a los guardias del pasillo y han corrido hacia las celdas.

\paragraph{El Chequeo de la Memoria}
Lira Doven busca la mirada de Yenar Strig. Tras un segundo, lo suelta con frustración. El viejo se desploma, convertido en un cascarón vacío.
\emph{"Llegamos tarde. El Velo ya se lo ha llevado."} —murmura ella con rabia.

\begin{npcsheet}[Miembro de la Resistencia][images/lira_doven.png]{Lira Doven}
\appearance{Mujer joven, cabello oscuro recogido, ropa de cuero gastada, cicatriz en la ceja.}
\voice{Baja, rápida, imperativa.}
\motivation{Sacar a los prisioneros marcados antes de que el Velo los reclame.}
\secrets{Reconoce señales tempranas de ``borrado''.}
\end{npcsheet}

Entonces, se gira hacia vosotros buscando señales de "borrado".
\emph{"¿Habló? ¿Os dijo algo antes de callar para siempre?"}

Si los jugadores confirman:
\emph{"Entonces vosotros sois la copia de seguridad."} —Hace una señal al pasillo—. \emph{"¡Orim, dáselo! ¡Nos los llevamos!"}

Un segundo resistente lanza al interior de la celda un fardo pesado: son vuestros cinturones y armas, recuperados tras noquear a los guardias de la entrada.
\emph{"Armaos. Rápido. Si queréis seguir existiendo, tenéis que luchar ahora."}

\subsubsection*{Combate: La Trampa del Pasillo}

Al salir, el grupo ve que el pasillo de huida está bloqueado por varios guardias. Detrás de ellos, inmóvil como una estatua de hierro, observa un \textbf{Vigía del Silente} de alto rango.

\paragraph{Fase 1: Los Peones}
Los guardias cargan contra los PJ. El Vigía no se mueve, simplemente observa con la cabeza ligeramente inclinada.
\begin{itemize}[nosep]
    \item \textbf{Mente Colmena:} Cuando cae el primer guardia, los otros giran la cabeza al unísono hacia el hueco, en silencio absoluto.
    \item \textbf{Tirada de Miedo:} Al ver esto, todos los PJ hacen una TR de Miedo (vs. Nivel del Vigía). Si fallan, penalizador igual a la diferencia del fallo.
\end{itemize}

\paragraph{Fase 2: El Avance del Silente}
Una vez derrotados los guardias, el Vigía desenvaina su vara, que emite un zumbido doloroso, y avanza caminando.
\textbf{La Revelación:} Cualquier ataque de los PJ contra el Vigía es desviado con una facilidad insultante (Parada 150+). Si un PJ logra golpearlo, la hoja rebota en su armadura sin que él parezca notarlo.

\begin{displayquote}
El Vigía alza su vara. No ataca con furia, sino con la precisión de un verdugo aburrido. Con un solo golpe, quiebra la guardia del personaje más fuerte y lo lanza contra la pared.
\emph{"Vuestro silencio es... imperfecto."}
\end{displayquote}

\paragraph{El Sacrificio de Orim}
Es evidente que no podéis ganar. El Vigía alza el arma para el golpe de gracia.
Entonces, Orim se interpone. El anciano ciego se lanza contra el Vigía, no con un arma, sino aferrando un saquito de pólvora alquímica y runas brillantes.

\begin{displayquote}
Orim gira la cabeza hacia Lira, sonriendo con tristeza bajo la venda.
\emph{"Lira... asegúrate de que recuerden."}
\end{displayquote}

\textbf{Explosión de Olvido:} Orim detona la carga. No es fuego, es una onda de vacío que distorsiona el pasillo y hace retroceder al Vigía momentáneamente, envolviéndolos en sombras.

\begin{npcsheet}[Vigía del Silente][images/guardia_silente.png]{El Verdugo}
\appearance{Máscara de hierro completa, casi 2 metros de altura.}
\motivation{No tiene prisa. Sabe que eventualmente atrapará a todos.}
\secrets{Es invulnerable a armas convencionales en este encuentro.}
\end{npcsheet}

\subsubsection*{Huida por el túnel y El Borrado}

\emph{"¡CORRED!"} —grita Lira, arrastrando a los PJ hacia el agujero en la pared mientras el humo de la explosión de Orim aún ciega al Vigía.

El grupo se lanza al túnel "desdibujado". El sacrificio de Orim os ha dado segundos vitales. Lira, con lágrimas de rabia en los ojos, saca el talismán de piedra gris y lo rompe contra el suelo del túnel.

\paragraph{El efecto del Borrado}
La oscuridad detrás de vosotros se vuelve sólida. La luz del pasillo, donde yace Orim, se \emph{corta} abruptamente. El camino deja de existir, borrado de la memoria del edificio.

Lira golpea la pared de roca con el puño, respirando con dificultad.
\begin{displayquote}
\emph{"Ese camino ya no existe. Y Orim... Orim es parte del muro ahora."}
\end{displayquote}

\subsubsection*{Cierre de la escena}

El grupo avanza en un silencio luctuoso hasta la trampilla de salida. Lira la abre con movimientos mecánicos, endurecida por la pérdida.

\begin{displayquote}
\emph{Si habéis llegado hasta aquí... es que el Velo aún no os ha reclamado del todo. Pero el precio ha sido alto.}
\end{displayquote}

\textbf{Comienza la Escena IV — La Casa Segura.}

\end{multicols*}
