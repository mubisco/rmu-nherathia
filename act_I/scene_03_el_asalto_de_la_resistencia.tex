\section{Escena III --- El Asalto de la Resistencia}
\begin{multicols}{2}

\textbf{Tipo de escena:} Acción caótica, ruptura del orden, primer contacto con la Resistencia.

\begin{readaloud}
El estruendo no es una explosión limpia.

Es un golpe sordo que recorre la piedra como un latido violento. El suelo vibra bajo vuestros pies. Polvo antiguo cae del techo y la antorcha del pasillo parpadea, casi apagándose.

Durante un instante, nadie grita.

Luego llegan los sonidos: metal contra piedra, órdenes cortadas a mitad, un alarido ahogado que se extingue demasiado rápido.

La cerradura de vuestra celda salta con un chasquido seco.

Desde el pasillo, una figura encapuchada os hace una seña urgente con la mano:
\begin{displayquote}
\emph{¡Ahora! ¡Si queréis seguir existiendo, movedos!}
\end{displayquote}
\end{readaloud}

\subsubsection*{Localización: Calabozos inferiores de la Casa del Silencio}

Los pasillos que rodean los calabozos forman un entramado confuso de corredores estrechos, escaleras descendentes y cámaras de servicio. No están pensados para el tránsito habitual, sino para movimientos discretos y traslados sin testigos.

\paragraph{Aspecto general}

Muros de piedra desnuda, ennegrecidos por el humo de antorchas antiguas; puertas metálicas numeradas sin inscripción religiosa; respiraderos estrechos por los que entra aire húmedo del subsuelo.

\paragraph{Sensaciones}

Calor repentino tras el frío de la celda; olor a humo mojado; dificultad para orientarse; el eco constante de pasos y choques que no siempre proceden del mismo lugar.

\paragraph{Indicio del Velo Nocturno}

En algunos tramos, los sonidos llegan antes que los hechos que los producen. Un golpe se oye\ldots{} y solo un instante después se ve la causa.

\subsubsection*{Desarrollo de la escena}

Lira y dos miembros más de la Resistencia han asaltado los calabozos de la Casa del Silencio en busca del viejo Yenar, pues conocen sus capacidades místicas. Han excavado un túnel desde los niveles bajos de la ciudad usando un pequeño talismán que transforma la tierra en lodo durante unos instantes. Sin embargo, el trabajo les ha llevado más tiempo del previsto y la Casa del Silencio ya está en alerta.

La Resistencia ha emergido en un \textbf{pozo de suministros} que conecta con una \textbf{pequeña sala de guardia} antes de los calabozos. Allí han reducido a los primeros guardias y han corrido hacia las celdas: es en ese caos donde encuentran a los personajes.

\begin{npcsheet}[Miembro de la Resistencia][images/lira_doven.png]{Lira Doven}
  \appearance{Mujer joven, cabello oscuro recogido, ropa de cuero gastada, rostro marcado por una cicatriz en la ceja.}
  \voice{Baja, rápida, imperativa; habla lo justo para ser entendida.}
  \motivation{Sacar a los prisioneros marcados antes de que el Velo los reclame.}
  \secrets{Reconoce señales tempranas de ``borrado'' en las personas; cree que uno de los PJ ya ha sido observado.}
\end{npcsheet}

\begin{npcsheet}[Vigía del Silente][images/guardia_silente.png]{Guardia del Silente}
  \appearance{Máscara de hierro, túnica oscura, arma ritual vibrando con un zumbido bajo.}
  \voice{Fría, monocorde; repite consignas incluso bajo ataque.}
  \motivation{Restaurar el orden, contener la ruptura, obedecer sin cuestionar.}
  \secrets{Algunos no reaccionan al dolor como debería hacerlo un ser humano.}
\end{npcsheet}

\paragraph{Confusión inicial}

Lira no se presenta. Está demasiado tarde para explicaciones. Su mirada se fija un instante en los PJ, como si midiera si aún ``están ahí'', y lanza una orden:

\begin{displayquote}
\emph{¡De pie! ¡Fuera! ¡No habléis más de lo necesario!}
\end{displayquote}

Si los PJ intentan preguntar por qué, quiénes son o qué ocurre, Lira responde sin detenerse:

\begin{displayquote}
\emph{Luego. Si llegamos a tener luego.}
\end{displayquote}

\paragraph{Primer contacto}

Mientras corta ataduras o empuja a los PJ fuera de la celda, Lira da instrucciones rápidas:

\begin{itemize}[nosep]
  \item \emph{Seguidme y no os separéis.}
  \item \emph{No miréis atrás si oís vuestro nombre.}
  \item \emph{Si veis niebla en un pasillo cerrado, no entréis.}
\end{itemize}

\paragraph{La ruta de escape}

La única salida viable es regresar por donde la Resistencia ha entrado: hacia la sala de guardia y el pozo de suministros que conecta con el túnel.

\subsubsection*{Escena de combate: Sala de guardia y recuperación de armas}

Al volver hacia la sala de guardia, el grupo se topa con refuerzos: varios guardias, comandados por un \textbf{Vigía del Silente} de rango superior. Sus órdenes llegan cortadas, como si alguien las estuviera repitiendo desde otro lugar.

\paragraph{Objetivo}

Derrotar o incapacitar al grupo de guardias para asegurar la salida hacia el pozo de suministros.

\paragraph{Armas y equipo}

Los PJ están desarmados al inicio. En la sala de guardia hay un armario metálico:

\begin{itemize}[nosep]
  \item Dentro se encuentran \textbf{las armas y pertenencias confiscadas de los PJ}.
  \item Si el grupo no puede abrir el armario a tiempo, Lira o uno de sus compañeros les arroja un arma improvisada (porra, cuchillo, vara corta).
\end{itemize}

\paragraph{Detalles tácticos}

\begin{itemize}[nosep]
  \item La sala es estrecha: combate a corta distancia, con cobertura parcial (mesa de guardia, estanterías).
  \item El Vigía intenta cortar la retirada, no matar: quiere \textbf{capturar} para ``clasificar''.
  \item Algunos guardias se mueven con sincronía inquietante, como si obedecieran un pulso común.
\end{itemize}

\subsubsection*{Mecánicas opcionales (RMU)}

\paragraph{Abrir el armario}

Tirada adecuada (cerraduras, fuerza, maniobra) según el sistema que uses. El éxito permite recuperar equipo completo; el fracaso implica recuperar solo lo esencial antes de que el combate los alcance.

\paragraph{Atletismo / Movimiento}

Para moverse entre cobertura, evitar ser rodeados o alcanzar el pozo de suministros bajo presión.

\paragraph{Miedo}

Al ver a los guardias actuar sin pánico, o al escuchar órdenes repetidas como ecos. Una RR de Miedo fallida implica penalizadores temporales.

\paragraph{Percepción}

Detectar la presencia del talismán de Lira y su efecto: la piedra ``cede'' de forma antinatural cerca del pozo, como si el suelo respirara.

\subsubsection*{Huida por el túnel y derrumbe}

Una vez derrotados los guardias (o al menos abiertos paso), Lira conduce al grupo hacia el pozo de suministros. El descenso es rápido, torpe y lleno de barro: el túnel no fue hecho para caminar, sino para sobrevivir.

\begin{readaloud}
El túnel no tiene forma de túnel. Es una herida en la tierra.

Las paredes son barro espeso que aún no ha decidido si quiere ser piedra o lodo. Vuestros pies se hunden. El aire huele a humedad viva. La oscuridad no es completa: es opaca.
\end{readaloud}

\paragraph{El talismán}

Al avanzar, Lira extrae un pequeño talismán ennegrecido y lo aprieta con fuerza. Susurra algo que no se entiende. La tierra vibra.

\begin{displayquote}
\emph{Ahora.}
\end{displayquote}

\paragraph{Derrumbe}

Justo cuando el grupo atraviesa un tramo concreto, Lira activa el talismán de nuevo. La tierra se vuelve lodo un instante\ldots{} y después colapsa.

\begin{readaloud}
Detrás de vosotros, el túnel se desploma con un rugido húmedo. No es un derrumbe seco: es como si la tierra se cerrara para borrar el camino.

El sonido queda atrapado en el barro. La oscuridad vuelve a sellarse.
\end{readaloud}

El derrumbe corta la persecución y protege la huida, pero deja claro que no hay vuelta atrás por ese camino.

\subsubsection*{Ganchos y detalles del Velo Nocturno}

\begin{itemize}[nosep]
  \item Durante el combate, uno de los guardias repite una orden que nadie ha dicho en voz alta.
  \item En el túnel, los ecos de los pasos suenan como si hubiera \emph{uno más}.
  \item Al derrumbarse, la tierra parece ``recordar'' vuestra forma y luego borrarla.
\end{itemize}

\subsubsection*{Cierre de la escena}

Tras varios minutos de avance en la oscuridad húmeda, el grupo alcanza una trampilla baja, oculta tras madera podrida y piedra vieja. Lira la abre con una sola mano, sin hacer ruido.

Al otro lado hay una habitación pequeña y mal iluminada.

Lira se detiene por primera vez, respira hondo y dice:

\begin{displayquote}
\emph{Si habéis llegado hasta aquí\ldots{} aún sois vosotros.}
\end{displayquote}

\textbf{Comienza la Escena IV — La Casa Segura.}

\end{multicols}
