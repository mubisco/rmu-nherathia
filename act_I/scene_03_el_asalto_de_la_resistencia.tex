\section{Escena III --- El Asalto de la Resistencia}

\begin{multicols}{2}

\textbf{Tipo de escena:} Huida, Combate Táctico (Mente Colmena) y Sacrificio Narrativo.

\begin{readbox}
Lira no os da tiempo a procesar el "glitch" de memoria. Os empuja al pasillo, donde el aire huele a humo y ozono.

\emph{"¡Orim, flanco derecho! ¡Tenemos a los portadores!"} grita ella.

El veterano Orim, con su ballesta pesada humeante, asiente y os lanza un saco pesado con vuestro equipo recuperado.

\emph{"Vestíos sobre la marcha,"} gruñe Orim. \emph{"El edificio está despertando. Los pasillos se están cerrando."}
\end{readbox}

\begin{npcsummary}[Líder de Célula][images/lira_doven.png]{Lira Doven}
    \appearance{Joven, atlética, con la mirada endurecida de quien ha visto morir a demasiados amigos. Lleva armadura de cuero tachonado.}
    \voice{Autoritaria y urgente. No pide, ordena para salvar vidas.}
    \motivation{Salvar lo que queda de la verdad, aunque sea sacando a desconocidos de una celda vacía.}
\end{npcsummary}

\subsubsection*{La Encrucijada}

Corréis por los corredores de piedra negra. El plan de la Resistencia es simple: volar una salida en el muro exterior antes de que la seguridad mágica del Silente selle el nivel.

Pero el camino no está despejado. Al llegar a una intersección amplia, os topáis con una patrulla que bloquea la ruta de escape.

No son guardias normales. Se mueven con una sincronía antinatural, sin hablar. Detrás de ellos, inmóvil como una estatua de hierro, observa un oficial de alto rango: un \textbf{Guardia del Silente}.

\begin{readbox}
Los hombres que bloquean el paso parecen guardias comunes: jubones de cuero sucio, botas gastadas y armas con el filo mellado. Son la carne de cañón de la prisión.

Pero hay algo incorrecto en ellos. No gritan, no se insultan, ni siquiera parpadean. Sus rostros están relajados, con la mandíbula suelta y los ojos vidriosos clavados en la nada.

Cuando se giran hacia vosotros, lo hacen todos a la vez, con la sincronía perfecta de un mecanismo de relojería. No son soldados; son marionetas esperando que alguien tire de los hilos.
\end{readbox}

\emph{"¡Abrid paso!"} grita Lira, desenvainando su espada.

\textit{Id al \textbf{Encuentro Táctico I-B} (Pág. Siguiente) para resolver el combate contra los guardias.}

\subsubsection*{La Mirada del Abismo}

\begin{readbox}
Una vez que el último de los guardias cae, el Guardia del Silente da un paso al frente. No saca armas. Lleva una mano a su garganta, donde cuelga un amuleto de hierro negro: un \textbf{Sello del Dictado}.

\emph{"La celda estaba vacía,"} dice el Verdugo con una voz que suena dentro de vuestras cabezas. \emph{"Pero vosotros estáis llenos de ruido. Debo corregirlo."}

El zumbido psíquico se convierte en una voz atronadora que no entra por los oídos, sino que vibra en los huesos.
\end{readbox}

\begin{mechanics}{Hechizo: Holy Shout (Grito Sagrado)}
El Guardia activa el \textbf{Sello del Dictado} para lanzar \textit{Holy Shout} (Nvl 15).

\textbf{1. El Ataque (Spellcasting Roll - SCR):}
El GM realiza la tirada de lanzamiento del Guardia del Silente.
\begin{itemize}
    \item \textbf{Fórmula:} \texttt{d100 OE + 10 (Intuición) + 16 (Rangos en Attunement)}.
    \item Este resultado establece la \textbf{Dificultad} que deben superar los jugadores.
\end{itemize}

\textbf{2. La Resistencia (Resistance Roll - RR):}
Cada jugador debe realizar una \textbf{RR de Canalización} para igualar o superar el SCR del Guardia.
\begin{itemize}
    \item \textbf{Fórmula:} \texttt{d100OE + Bono RR Can. - 30 (2 x Nvl Ataque) + Mods}.
    \item \textbf{Nivel de Ataque:} 15 (El nivel del hechizo en el amuleto). Esto impone un penalizador base de \textbf{-30}.
    \item \textbf{Modificador de Entorno:} \textbf{-30} (La Casa del Silencio debilita la voluntad).
    \item \textbf{Cálculo Final para el Jugador:}
\end{itemize}

\textbf{Consecuencias:}
Comparar el resultado de la RR con el SCR.
\begin{itemize}
    \item \textbf{Fallo por 1-10:} Aturdido (\textit{Stunned}) 1 asalto.
    \item \textbf{Fallo por >10:} Aturdido 1 asalto por cada 10 puntos de fallo adicional.
    \item \textit{Efecto Narrativo:} La voz divina aplasta sus sentidos, impidiéndoles actuar mientras Orim se prepara.
\end{itemize}
\end{mechanics}

El Guardia levanta su maza negra. Va a ejecutaros uno a uno mientras vuestros cuerpos se niegan a obedecer.

\subsubsection*{El Sacrificio de Orim}

Lira intenta levantaros, pero ella también está temblando por la presión.

Entonces, Orim os empuja hacia atrás. El veterano se arranca una cadena de frascos alquímicos del cinturón. No hay miedo en su rostro, solo una aceptación cansada.

\begin{readbox}
Orim mira a Lira una última vez.
\emph{"Sácalos, chica. Que recuerden."}

Orim carga contra el Guardia del Silente, rompiendo la parálisis con pura fuerza de voluntad, y estampa los frascos contra el pecho del verdugo.
\end{readbox}

\textbf{La Detonación:}
Una explosión de luz blanca y fuego alquímico inunda el pasillo. La onda expansiva rompe la concentración del Guardia y os lanza a través del arco de salida hacia los túneles de desagüe.

\subsubsection*{El Sellado del Camino}

Lira os arrastra por el túnel húmedo. Detrás, el humo de la explosión empieza a disiparse, y oís los pasos metálicos del Guardia, que emerge de las llamas casi intacto. Os sigue.

Lira se detiene un segundo frente a un arco de piedra. Saca un objeto de su cuello: un \textbf{Amuleto de Piedra Gris} tallado con una runa inversa.

\begin{readbox}
Lira arranca el amuleto y lo lanza contra el suelo del túnel, justo entre vosotros y el perseguidor.

\emph{"Olvida que este camino existió,"} susurra.
\end{readbox}

El efecto es inmediato y aterrador. La realidad se pliega. El túnel detrás de vosotros no se derrumba; simplemente \textbf{deja de estar ahí}. Donde había un pasillo, ahora hay roca sólida, lisa y antigua, como si nunca se hubiera excavado. El Guardia, Orim y la prisión han quedado borrados de vuestra ruta.

Lira golpea la pared sólida con el puño una vez, conteniendo las lágrimas.

\emph{"Vamos. La casa segura no está lejos."}

\textbf{Continuar a la Escena IV — La Casa Segura.}

\end{multicols}

% --- SECCIÓN TÁCTICA ---
\newpage

\begin{tacticalscene}{La Encrucijada (I-B)}

\textbf{Configuración del Escenario:}

\begin{itemize}
    \item \textbf{Dimensiones:} Cruce en forma de cruz (+).
    \begin{itemize}
        \item \textbf{Ancho de los pasillos:} 6m (4 casillas).
        \item \textbf{Largo visible:} 9m (6 casillas) en cada dirección desde el centro.
        \item \textbf{Área total del mapa:} 21m x 21m (14x14 casillas).
    \end{itemize}
    \item \textbf{Terreno:} Suelo de piedra pulida (sin penalizadores).
    \item \textbf{Iluminación:} Antorchas en las paredes (Luz normal, radio 6m/4 casillas).
\end{itemize}

\textbf{Regla Especial: La Mente Colmena}
Los guardias actúan como una sola entidad bajo la influencia del Guardia del Silente.
\begin{itemize}
    \item \textbf{Iniciativa Compartida:} Todos los guardias actúan en el mismo turno de Iniciativa (el más alto del grupo).
    \item \textbf{Defensa Coordinada:} Si un PJ ataca a un guardia y hay otro guardia adyacente al objetivo, este gana \textbf{+20 DB} (se cubren mutuamente).
    \item \textbf{Sin Miedo:} Son inmunes a efectos de Miedo o Aturdimiento mientras el Guardia del Silente esté vivo.
\end{itemize}

\tcbline

\textbf{ESCALADO DEL ENCUENTRO}
\begin{itemize}
    \item \textbf{Guardias Poseídos:} 1 por Jugador + 1.
    \item \textbf{Líder:} 1 Guardia del Silente (En retaguardia, zona Norte).
\end{itemize}

\tcbline

\textbf{ENEMIGOS}

\begin{statblock}[images/guardia_poseido.png]{Guardia Poseído}{Nvl 2 (Fighter) | PV 45 | AT 6 | DB 15}
    \textbf{Comportamiento:} Silenciosos y mecánicos. No gritan al morir.
    \textbf{Ataques:}
    \begin{itemize}
        \item \textbf{Espada Corta (Melee):} +50 OB.
        \\ \textit{Tabla:} \textbf{Short Sword}.
    \end{itemize}
\end{statblock}

\begin{statblock}[images/guardia_silente.png]{Guardia del Silente (Líder)}{Nvl 10 (Paladin) | PV ?? | AT 20 | DB 80}
    \textbf{Nota:} Este enemigo es un \emph{Hazard} narrativo en esta escena.
    \textbf{Acción:} Observa. Si es atacado, desvía el golpe sin esfuerzo (Parada Total). Solo actúa (usando el Amuleto) cuando sus esbirros caen.
\end{statblock}

\end{tacticalscene}
