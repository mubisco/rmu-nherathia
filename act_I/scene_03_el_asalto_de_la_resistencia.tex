\section{Escena III --- El Asalto de la Resistencia}
\begin{multicols}{2}

\textbf{Tipo de escena:} Acción caótica, ruptura del orden, primer contacto con la Resistencia.

\begin{readaloud}
El estruendo no es una explosión limpia.

Es un golpe sordo que recorre la piedra como un latido violento. El suelo vibra bajo vuestros pies. Polvo antiguo cae del techo y la antorcha del pasillo parpadea, casi apagándose.

Durante un instante, nadie grita.

Luego llegan los sonidos: metal contra piedra, órdenes cortadas a mitad, un alarido ahogado que se extingue demasiado rápido.

La cerradura de vuestra celda salta con un chasquido seco.

Desde el pasillo, una figura encapuchada os hace una seña urgente con la mano:
\begin{displayquote}
\emph{¡Ahora! ¡Si queréis seguir existiendo, movedos!}
\end{displayquote}
\end{readaloud}

\subsubsection*{Localización: Pasillos inferiores del Templo-Fortaleza}

Los pasillos que rodean los calabozos forman un entramado confuso de corredores estrechos, escaleras descendentes y cámaras de servicio. No están pensados para el tránsito habitual, sino para movimientos discretos y traslados sin testigos.

\paragraph{Aspecto general}

Muros de piedra desnuda, ennegrecidos por el humo de antorchas antiguas; puertas metálicas numeradas sin inscripción religiosa; respiraderos estrechos por los que entra aire húmedo del subsuelo.

\paragraph{Sensaciones}

Calor repentino tras el frío de la celda; olor a humo mojado; dificultad para orientarse; el eco constante de pasos y choques que no siempre proceden del mismo lugar.

\paragraph{Indicio del Velo Nocturno}

En algunos tramos, los sonidos llegan antes que los hechos que los producen. Un golpe se oye… y solo un instante después se ve la causa.

\subsubsection*{PNJ relevantes}

\begin{npcsheet}[Miembro de la Resistencia][images/lira_doven.png]{Lira Doven}
  \appearance{Mujer joven, cabello oscuro recogido, ropa de cuero gastada, rostro marcado por una cicatriz en la ceja.}
  \voice{Baja, rápida, imperativa; habla lo justo para ser entendida.}
  \motivation{Sacar a los prisioneros marcados antes de que el Velo los reclame.}
  \secrets{Reconoce señales tempranas de ``borrado'' en las personas; cree que uno de los PJ ya ha sido observado.}
\end{npcsheet}

\begin{npcsheet}[Vigía del Silente][images/guardia_silente.png]{Guardia del Silente}
  \appearance{Máscara de hierro, túnica oscura, arma ritual vibrando con un zumbido bajo.}
  \voice{Fría, monocorde; repite consignas incluso bajo ataque.}
  \motivation{Restaurar el orden, contener la ruptura, obedecer sin cuestionar.}
  \secrets{Algunos no reaccionan al dolor como debería hacerlo un ser humano.}
\end{npcsheet}

\subsubsection*{Desarrollo de la escena}

El asalto no es una liberación ordenada: es una irrupción desesperada.

Lira y otros miembros de la Resistencia avanzan por los pasillos usando rutas secundarias, abriendo celdas concretas y evitando otras. No liberan a todos los prisioneros; solo a aquellos que consideran aún ``recuperables''.

\paragraph{Primer contacto}

Lira no se presenta. Mientras corta las ataduras o fuerza la puerta, lanza instrucciones rápidas:

\begin{itemize}
  \item \emph{Seguidme y no os separéis.}
  \item \emph{No miréis atrás si oís vuestro nombre.}
  \item \emph{Si veis niebla en un pasillo cerrado, no entréis.}
\end{itemize}

\paragraph{Caos controlado}

Durante la huida, los PJ presencian escenas fragmentarias:

\begin{itemize}
  \item Un guardia derribado que intenta levantarse una y otra vez, repitiendo la misma orden.
  \item Un prisionero liberado que se queda inmóvil, incapaz de recordar cómo caminar.
  \item Un miembro de la Resistencia que desaparece tras una esquina envuelta en humo… sin sonido alguno.
\end{itemize}

\subsubsection*{Decisiones durante la huida}

Permite a los PJ tomar decisiones rápidas que no cambian el resultado global (escapar), pero sí el tono y las consecuencias.

\paragraph{Ayudar a otros prisioneros}

Los PJ pueden intentar liberar a más cautivos.
\begin{itemize}
  \item Éxito: ganan reputación con la Resistencia.
  \item Fracaso: retraso; aumenta la presión y el riesgo de heridos.
\end{itemize}

\paragraph{Enfrentarse a los guardias}

Combates breves, tensos, en espacios cerrados.
\begin{itemize}
  \item Los guardias actúan de forma coordinada y sin pánico.
  \item Algunos parecen no reaccionar al dolor inmediatamente.
\end{itemize}

\paragraph{Seguir órdenes sin cuestionar}

Reduce riesgos inmediatos, pero deja a los PJ con la sensación de no comprender del todo qué está ocurriendo.

\subsubsection*{Mecánicas opcionales (RMU)}

\paragraph{Atletismo / Movimiento}

Para correr por pasillos estrechos, esquivar escombros o saltar tramos dañados. El fallo implica caídas, golpes o separación momentánea del grupo.

\paragraph{Percepción}

Detectar rutas falsas, pasillos contaminados por el Velo o sonidos que no deben seguirse.

\paragraph{Miedo}

Al presenciar comportamientos anómalos (guardias que no reaccionan, desapariciones silenciosas), puede aplicarse una RR de Miedo. El fallo implica penalizadores temporales.

\subsubsection*{Ganchos y detalles del Velo Nocturno}

\begin{itemize}
  \item Algunos pasillos parecen cambiar de longitud cuando no se los mira directamente.
  \item Un PJ tiene la sensación de que alguien camina justo detrás\ldots{} pero nunca hay nadie.
  \item Lira evita responder preguntas sobre ``a quién sirven'' exactamente.
\end{itemize}

\subsubsection*{Cierre de la escena}

Tras una última escalera descendente, el grupo atraviesa una puerta secundaria que se cierra tras ellos con un golpe seco.

El ruido del combate queda atrás, amortiguado por capas de piedra.

Lira se detiene por primera vez, respira hondo y dice:

\begin{displayquote}
\emph{Si habéis llegado hasta aquí\ldots{} aún sois vosotros.}
\end{displayquote}

Sin dar más explicaciones, los guía hacia un pasadizo estrecho que desciende aún más bajo la ciudad.

\textbf{Comienza la Escena IV — La Casa Segura.}

\end{multicols}
