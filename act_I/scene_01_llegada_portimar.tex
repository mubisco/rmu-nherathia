\section{Escena I --- Llegada a Portimar}
\begin{multicols}{2}

\textbf{Tipo de escena:} Introducción, opresión urbana, primer contacto con el culto del Silente.

\begin{readaloud}
La costa de Nherathia siempre huele a sal y a madera podrida, pero aquella mañana el mar traía algo más: un silencio demasiado profundo para ser natural.

La niebla se arremolina alrededor del barco como si tuviera voluntad propia. A medida que se acerca el muelle, los personajes distinguen los contornos deformados de Portimar: tejados inclinados como espinas rotas, torres envueltas en velos de humo, y sobre todo, la inmensa \textbf{Casa del Silencio}, dominando la ciudad como un cuchillo clavado en la roca.

No hay pregoneros, ni estibadores, ni gritos de mercado. Solo vigilancia.

Sombras quietas en los balcones. Puertas entreabiertas que se cierran en cuanto el barco maniobra. Aves que vuelan en círculos, pero nunca se posan.

Cuando la embarcación toca el muelle, un trío de guardias con máscaras de hierro y capuchas oscuras sube a bordo sin pedir permiso. Llevan símbolos circulares grabados en los petos y lucen varas metálicas que vibran con un zumbido bajo.

La voz del capitán tiembla:

\begin{displayquote}
\emph{Bienvenidos\ldots{} a Portimar. Por favor, responded solo cuando os lo permitan.}
\end{displayquote}

\end{readaloud}

\subsubsection*{Localización: El Muelle del Ocaso}

El ``Muelle del Ocaso'' es una zona controlada directamente por miembros de \textbf{La Vigilia del Silente}, una élite paramilitar del culto. No es un lugar para comerciantes, sino para controlar quién entra y quién sale.

\paragraph{Aspecto general}

Madera ennegrecida por la humedad; redes colgando como piel muerta; antorchas que chisporrotean dentro de cajas metálicas, dando una luz amortiguada; un campanario lejano tañe una nota única, que no anuncia horas: es una advertencia.

\imagerim{images/portimar.jpg}[Vista de Portimar al amanecer]

\paragraph{Sensaciones}

Olor a hierro y sal; calma antinatural; ecos de pasos\ldots{} pero no se ve a nadie alrededor.

\paragraph{Primer indicio del Velo Nocturno}

Cuando el barco atraca, los PJ empiezan a tener extrañas sensaciones. Pide una tirada de Percepción secreta a los jugadores, pero ignora el resultado y haz que cada uno ``sienta`` uno de los siguientes efectos:

\begin{itemize}[nosep]
  \item El jugador siente como alguien susurra su nombre a sus espaldas. Al girarse, sólo hay niebla y una sensación fría.
  \item El jugador ve sombras que no se ajustan respecto a lo que las proyectan, se mueven a destiempo.
  \item Pensamientos intrusivos del estilo \emph{``No digas tu nombre``}
  \item El jugador oye una campana aislada, sin patrón. Si pregunta, nadie más lo oye.
  \item El suelo cede pero realmente no se mueve.
  \item Las voces cercanas se oyen claras, pero las lejanas son ininteligibles, aunque deberían escucharse.
\end{itemize}

\vspace{4pt}
Deberán hacer una TR contra Miedo -20. Si la fallan, tendrán una penalización de -10 durante toda la escena. Es el primer aviso de que Portimar está impregnada de ecos antiguos. Si algún PJ comenta lo que ha sentido, un guardia cercano frunce el ceño y responde:
\begin{displayquote}
\emph{Eso es normal. El puerto siempre ha sido así.}
\end{displayquote}

\begin{npcsheet}[Capitán del barco][images/marn_velkan.jpg]{Marn Velkan}
  \appearance{Hombre delgado, barba húmeda, manos temblorosas.}
  \voice{Rápida, nerviosa, evita mirar a los guardias.}
  \motivation{Sobrevivir al desembarco; no quiere problemas, entregará a los PJ si se lo piden.}
  \secrets{Mantiene un acuerdo tácito con la Vigilia del Silente: cada cierto número de arribadas, entrega “pasajeros no registrados” a cambio de que su carga no sea inspeccionada. Lo considera parte del coste del negocio.}
\end{npcsheet}

\subsubsection*{Desarrollo de la escena}

Los guardias suben a bordo, inspeccionan, revisan documentos inexistentes, hasta que uno de ellos se dirige a donde se encuentran los jugadores. Su capa lleva dos bandas de color gris, como indicando un rango mayor. Se dirige a cada uno de los personajes con preguntas capciosas:

\begin{itemize}[nosep]
  \item \emph{¿De dónde venís?}
  \item \emph{¿Habéis soñado algo durante la travesía?}
  \item \emph{¿Conocéis a alguien en Portimar?}
  \item \emph{¿Alguien os ha hablado desde la costa?}
\end{itemize}

No importa lo que respondan: la intención es generar inquietud.

\begin{npcsheet}[Guardia del Silente][images/guardia_silente.jpg]{Vigilante Sereth}
  \appearance{Máscara de hierro sin rasgos, capucha oscura, vara metálica.}
  \voice{Ahogada, casi monótona; parece escuchar siempre algo detrás de sí.}
  \motivation{Cumplir normas, evitar que los recién llegados traigan ecos indeseados.}
\end{npcsheet}

\paragraph{Momento clave: la acusación}

El Vigilante Sereth se detiene de repente frente a uno de los PJ (elige al que sea más expresivo, más extranjero o que haya hablado más):

\begin{displayquote}
\emph{Tú\ldots{} tu sombra no coincide. Ven con nosotros.}
\end{displayquote}

Los jugadores pueden intentar varias opciones para evitar el arresto, pero realmente poco se puede hacer. En ese momento, el capitán revelará que han sido utilizados: \emph{``El trato siempre ha sido claro. La carga entra... y vosotros os lleváis a los que no figuran\ldots{}``}

\subsubsection*{Opciones de los jugadores}

\paragraph{Intentar razonar}

Los guardias ignoran explicaciones y responden:

\begin{displayquote}
\emph{Las palabras son ecos. Los ecos mienten.}
\end{displayquote}

\paragraph{Intentar sobornar}

Pésima idea. Se considera un acto corrupto y agravará la situación.

\subsubsection*{Cierre de la escena}

Los personajes son esposados con grilletes fríos y conducidos por un corredor de piedra húmeda hacia la muralla interior. El último sonido que escuchan al abandonar el muelle es la misma nota grave del campanario, sostenida demasiado tiempo. Tras unos momentos llegan a la ominosa Casa del Silencio, donde son conducidos a los calabozos subterráneos.
\end{multicols}

\vspace{8pt}
\imagerim{images/casa_del_silencio_portimar.jpg}[Casa del Silencio de Portimar]
