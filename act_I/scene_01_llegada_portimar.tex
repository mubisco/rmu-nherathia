\section{Escena I --- Llegada a Portimar}
\begin{multicols}{2}

\textbf{Tipo de escena:} Introducción, opresión urbana, primer contacto con el culto del Silente.

\begin{readaloud}
La costa de Nherathia siempre huele a sal y a madera podrida, pero aquella mañana el mar traía algo más: un silencio demasiado profundo para ser natural.

La niebla se arremolina alrededor del barco como si tuviera voluntad propia. A medida que se acerca el muelle, los personajes distinguen los contornos deformados de Portimar: tejados inclinados como espinas rotas, torres envueltas en velos de humo, y sobre todo, el inmenso \textbf{Templo-Fortaleza del Silente}, dominando la ciudad como un cuchillo clavado en la roca.

No hay pregoneros, ni estibadores, ni gritos de mercado. Solo vigilancia.

Sombras quietas en los balcones. Puertas entreabiertas que se cierran en cuanto el barco maniobra. Aves que vuelan en círculos, pero nunca se posan.

Cuando la embarcación toca el muelle, un trío de guardias con máscaras de hierro y capuchas oscuras sube a bordo sin pedir permiso. Llevan símbolos circulares grabados en los petos y lucen varas metálicas que vibran con un zumbido bajo.

La voz del capitán tiembla:

\begin{displayquote}
\emph{Bienvenidos\ldots{} a Portimar. Por favor, responded solo cuando os lo permitan.}
\end{displayquote}

\end{readaloud}

\subsubsection*{Localización: El Muelle del Ocaso}

El ``Muelle del Ocaso'' es una zona controlada directamente por los \textbf{Vigías del Silente}, una élite paramilitar del culto. No es un lugar para comerciantes, sino para controlar quién entra y quién sale.

\paragraph{Aspecto general}

Madera ennegrecida por la humedad; redes colgando como piel muerta; antorchas que chisporrotean dentro de cajas metálicas, dando una luz amortiguada; un campanario lejano tañe una nota única, que no anuncia horas: es una advertencia.

\paragraph{Sensaciones}

Olor a hierro y sal; calma antinatural; ecos de pasos\ldots{} pero no se ve a nadie alrededor.

\paragraph{Primer indicio del Velo Nocturno}

Cuando los PJ pisan tierra firme, una corriente fría atraviesa la niebla y todos sienten que alguien susurra su nombre. No es real. No puede ser real. Pero uno de los guardias reacciona, como si hubiera escuchado lo mismo. Es el primer aviso de que Portimar está impregnada de ecos antiguos.

\subsubsection*{PNJ relevantes}

\paragraph{Vigilante Sereth (Guardia del Silente)}

\textbf{Apariencia:} Máscara de hierro sin rasgos, capucha oscura, vara metálica que emite un zumbido bajo.

\textbf{Voz:} Ahogada, casi monótona; parece escuchar siempre algo detrás de sí.

\textbf{Motivación:} Cumplir normas, evitar que los recién llegados traigan ``ecos indeseados'', impresionar a su superior.

\textbf{Secretos:} Oye ecos cuando está en la niebla; lo oculta por miedo. Podría ayudar a los PJ si se ven vulnerables.

\paragraph{Marn Velkan (Capitán del barco)}

\textbf{Apariencia:} Hombre delgado, barba húmeda, manos temblorosas.

\textbf{Voz:} Rápida, nerviosa, evita mirar a los guardias.

\textbf{Motivación:} Sobrevivir al desembarco; no quiere problemas, entregará a los PJ si se lo piden.

\textbf{Secretos:} Sabe que los últimos barcos ``extranjeros'' fueron retenidos días enteros. No piensa intervenir cuando comiencen los arrestos.

\subsubsection*{Desarrollo de la escena}

Los guardias suben a bordo, inspeccionan, revisan documentos inexistentes, hacen preguntas capciosas:

\begin{itemize}
  \item \emph{¿De dónde venís?}
  \item \emph{¿Habéis soñado algo durante la travesía?}
  \item \emph{¿Conocéis a alguien en Portimar?}
  \item \emph{¿Alguien os ha hablado desde la costa?}
\end{itemize}

No importa lo que respondan: la intención es generar inquietud.

\paragraph{Momento clave: la acusación}

El Vigilante Sereth se detiene de repente frente a uno de los PJ (elige al que sea más expresivo, más extranjero o que haya hablado más):

\begin{displayquote}
\emph{Tú\ldots{} tu sombra no coincide. Ven con nosotros.}
\end{displayquote}

A partir de aquí, uno de los siguientes detonantes lleva al arresto:

\begin{description}
  \item[A) Malentendido provocado por los guardias:] acusan a un PJ de ocultar un objeto, golpean al capitán para intimidar, exigen una inspección más profunda.
  \item[B) Estibador anónimo:] desde el muelle alguien grita \emph{``¡No dejéis que entren! ¡Esos barcos traen ecos!''}. Los guardias responden con violencia y culpan a los PJ.
  \item[C) Episodio del Velo:] los PJ ven fugazmente una figura entre la niebla. Los guardias no la ven, pero entienden que los personajes sí. Eso basta para considerarlos ``contaminados''.
\end{description}

En todos los casos, los PJ son arrestados por ``riesgo de eco'' y trasladados a los calabozos del Templo-Fortaleza. No deben poder evitar el arresto, pero pueden sufrirlo con dignidad, provocar una pelea (con severas consecuencias) o intentar huir (fracaso garantizado, pero que les da información útil).

\subsubsection*{Opciones de los jugadores}

\paragraph{Intentar razonar}

Los guardias ignoran explicaciones y responden:

\begin{displayquote}
\emph{Las palabras son ecos. Los ecos mienten.}
\end{displayquote}

\paragraph{Intentar sobornar}

Pésima idea. Se considera un acto corrupto y agravará la situación.

\paragraph{Resistirse}

Al ver a varios guardias sin rostro, coordinados, puede aplicarse una maniobra de Miedo (RR de Miedo) en RMU. El fracaso implica penalizadores temporales a acciones por temblor o bloqueo.

\paragraph{Observar detalles}

Permite detectar insignias, jerarquías, ruidos extraños en la niebla, o runas en las varas de los guardias que insinúan magia o rituales.

\subsubsection*{Mecánicas opcionales (RMU)}

\paragraph{Percepción}

Tirada de Percepción (difícil) para notar que uno de los guardias susurra sin mover los labios: indicio del Velo.

\paragraph{Corrupción mental incipiente}

Si un PJ mira demasiado tiempo a la niebla, debe superar una RR contra Canalización. En fracaso leve: escalofrío y +1 Fatiga; en fracaso grave: oye su nombre y sufre un penalizador temporal a acciones que requieran concentración.

\paragraph{Arresto formal}

Los PJ pueden intentar maniobras de Resistencia Física o Atletismo para evitar ser reducidos. Independientemente del resultado, acabarán en el calabozo, pero el éxito evita daños adicionales, mientras que el fracaso puede implicar heridas menores o pérdida de un objeto poco importante.

\subsubsection*{Ganchos y detalles del Velo Nocturno}

\begin{itemize}
  \item El Muelle del Ocaso ha tenido tres desapariciones recientes.
  \item Algunos marineros murmuran que ``las voces bajo el agua'' llaman a la gente.
  \item Un guardia se detiene y pregunta a un PJ: \emph{``¿Qué has visto realmente en la niebla?''} y luego se arrepiente de haberlo preguntado.
  \item Una campana que nadie ha tocado suena brevemente mientras los PJ son conducidos hacia la ciudad.
\end{itemize}

\subsubsection*{Cierre de la escena}

Los personajes son esposados con grilletes fríos y conducidos por un corredor de piedra húmeda hacia la muralla interior. El último sonido que escuchan al abandonar el muelle es la misma nota grave del campanario, sostenida demasiado tiempo.
\end{multicols}
