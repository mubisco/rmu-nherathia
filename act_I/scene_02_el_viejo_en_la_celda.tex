\section{Escena II --- El Viejo en la Celda}
\begin{multicols}{2}

\textbf{Tipo de escena:} Claustrofobia, horror silencioso, revelación parcial del Velo Nocturno.

\begin{readaloud}
La puerta de hierro se cierra con un sonido seco, definitivo.  
No hay eco. La piedra parece tragárselo.

La celda es estrecha, excavada directamente en la roca portuaria. Las paredes están húmedas, pero no gotean: sudan. La luz de una antorcha exterior apenas entra, deformada por los barrotes, proyectando sombras que no coinciden del todo con los cuerpos.

El aire es espeso. No huele mal.  
Huele a nada, como si los olores se hubieran cansado de existir allí.

Durante unos segundos —o minutos— no ocurre nada.

Entonces, alguien respira.

Desde el rincón más oscuro de la celda, una voz ronca rompe el silencio:
\begin{displayquote}
\emph{¿Sois reales\ldots{} o sois otra de las que vienen cuando cierran la puerta?}
\end{displayquote}
\end{readaloud}

\subsubsection*{Localización: Calabozos de la Casa del Silencio}

Los calabozos se encuentran excavados bajo la Casa del Silencio de Portimar. No son un espacio pensado para castigar, sino para retener, clasificar y hacer desaparecer sin ruido.

\paragraph{Aspecto general}

Celdas talladas directamente en la roca, barrotes gruesos cubiertos de óxido húmedo, antorchas separadas a distancias irregulares. No hay símbolos visibles del culto, pero la arquitectura transmite obediencia y silencio.

\paragraph{Sensaciones}

Frío constante; ausencia casi total de olores; una presión leve en los oídos; la sensación persistente de que el espacio es más pequeño de lo que debería ser.

\paragraph{Indicio del Velo Nocturno}

Las sombras no siempre coinciden con las fuentes de luz. En ocasiones, parecen retrasarse un instante o adoptar formas ligeramente distintas a las de los cuerpos que deberían proyectarlas.

\subsubsection*{Desarrollo de la escena}

\textit{Nota para el DJ: Yenar mezcla tiempos verbales, recuerdos propios y ajenos. Nunca responde de forma lógica a las preguntas, pero todo lo que dice es cierto.}

Yenar no se acerca a los PJ. Permanece encogido contra la pared, observándolos con una mezcla de cautela y reconocimiento incierto.

Habla en frases cortas, interrumpidas, como si temiera que alguien —o algo— pudiera oírlo.

\begin{npcsheet}[Prisionero][images/yenar_strig.png]{Yenar Strig}
  \appearance{Anciano extremadamente delgado, cabello blanco apelmazado por la humedad, ojos hundidos pero atentos.}
  \voice{Ronca, baja, fragmentada; habla en susurros entre silencios prolongados.}
  \motivation{Comprender por qué aún existe; advertir a otros antes de desaparecer.}
  \secrets{Ha oído al Velo pronunciar su nombre y luego dejar de hacerlo; percibe ecos de recuerdos ajenos y tiempos superpuestos.}
\end{npcsheet}

\begin{displayquote}
\emph{Antes\ldots{} traían criminales.}\\
\emph{Ahora traen nombres.}
\end{displayquote}

\begin{displayquote}
\emph{Os han registrado, ¿verdad? Eso es peor que estar aquí.}
\end{displayquote}

\begin{displayquote}
\emph{No es una celda. Es una pausa.}
\end{displayquote}

De pronto, sin venir a cuento, Yenar roza la pared con los dedos y murmura:

\begin{displayquote}
\emph{El mapa no está dibujado\ldots{} está gastado.}\\
\emph{No lo hicieron para que se viera, sino para que sobreviviera.}
\end{displayquote}

Si los PJ le preguntan quién es:

\begin{displayquote}
\emph{Yenar Strig. Creo. Eso decía el papel.}
\end{displayquote}

Si preguntan por qué está allí:

\begin{displayquote}
\emph{Porque supe algo\ldots{} o porque alguien decidió que ya no debía saberlo.}
\end{displayquote}

Sin relación aparente con la conversación, añade en voz baja:

\begin{displayquote}
\emph{Cuando salgáis, no sigáis el viento.}\\
\emph{Id hacia donde la niebla se vuelve torpe.}
\end{displayquote}

\paragraph{La verdad incompleta}

A lo largo de la conversación, Yenar deja caer fragmentos de información. Nunca los ordena ni los explica del todo.

\begin{itemize}
  \item \textbf{Las desapariciones no ocurren de noche.}
  \begin{displayquote}
  \emph{Eso es lo que dicen. Pero no es verdad.}
  \end{displayquote}

  \item \textbf{La gente desaparece cuando es reclasificada.}
  \begin{displayquote}
  \emph{Cambian el nombre. Cambian el recuerdo. Y luego\ldots{} no hay nadie a quien echar de menos.}
  \end{displayquote}

  \item \textbf{El Velo Nocturno no llega a Portimar.}
  \begin{displayquote}
  \emph{Siempre ha estado aquí.}
  \end{displayquote}

  \item \textbf{Algunos lo oyen. Otros lo sienten.}
  \begin{displayquote}
  \emph{Yo lo oí. Dijo mi nombre\ldots{} y luego dejó de decirlo.}
  \end{displayquote}
\end{itemize}

En un momento de aparente lucidez, Yenar añade:

\begin{displayquote}
\emph{Las ruinas no se abren.}\\
\emph{Se dejan entrar cuando recuerdas su nombre antiguo.}
\end{displayquote}

\subsubsection*{Fenómenos durante la escena}

Mientras Yenar habla, pueden ocurrir uno o varios de los siguientes detalles:

\begin{itemize}
  \item Un guardia pasa frente a la celda. Yenar se encoge, pero el guardia no reacciona a nada extraño.
  \item Por un instante, uno de los PJ oye su propio nombre susurrado desde la piedra.
  \item La sombra de Yenar proyectada por la antorcha no coincide exactamente con su postura.
\end{itemize}

Si los PJ comentan estos fenómenos, Yenar responde en voz baja:

\begin{displayquote}
\emph{No los miréis cuando hacen eso.}\\
\emph{Si os responden\ldots{} es que os han oído.}
\end{displayquote}

\subsubsection*{Pista sobre la Resistencia}

Si los PJ muestran empatía o logran calmar a Yenar, este deja caer una última advertencia:

\begin{displayquote}
\emph{Hay gente\ldots{} que no ha olvidado del todo.}\\
\emph{Ellos no usan celdas.}
\end{displayquote}

\subsubsection*{Mecánicas opcionales (RMU)}

\paragraph{Presencia / Voluntad}

Tirada para resistir la opresión psicológica del encierro y el discurso de Yenar.  
Un fallo puede implicar sudor frío, temblores o penalizadores temporales a acciones sociales o de percepción.

\paragraph{Percepción}

Tirada difícil para detectar las anomalías reales del entorno: sombras desfasadas, susurros desde la piedra, normalidad inquietante de los guardias.

\paragraph{Empatía / Influencia}

Permite ganar la confianza de Yenar y comprender que sus palabras, aunque confusas, contienen verdades útiles.

\subsubsection*{Ganchos y detalles del Velo Nocturno}

\begin{itemize}
  \item Yenar afirma que algunas celdas “siempre estuvieron vacías”.
  \item Los guardias actúan como si ciertos fenómenos no existieran.
  \item El nombre de un PJ es susurrado una sola vez, sin repetirse.
\end{itemize}

\subsubsection*{Cierre de la escena}

De pronto, Yenar se queda en silencio.

No hay espasmo ni colapso. Su cuerpo sigue allí, pero su mirada queda vacía, como si nunca hubiera habido nadie detrás.

Más tarde, cuando un guardia pasa frente a la celda, observa el interior y dice con total naturalidad:

\begin{displayquote}
\emph{Esta celda siempre estuvo vacía.}
\end{displayquote}

Poco después, golpes lejanos sacuden la piedra. Se oyen gritos apagados y un olor a humo húmedo se filtra por el pasillo.

Desde fuera, alguien susurra con urgencia:
\begin{displayquote}
\emph{Si queréis seguir existiendo\ldots{} agachaos.}
\end{displayquote}

Un estruendo abre la cerradura.

\end{multicols}
