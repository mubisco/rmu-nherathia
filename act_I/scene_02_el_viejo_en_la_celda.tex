\section{Escena II --- El Viejo en la Celda}

\begin{multicols*}{2}

\textbf{Tipo de escena:} Interacción Social, Horror Psicológico y Revelación de Lore.

\begin{readbox}
La puerta de hierro se cierra con un sonido definitivo. La celda es estrecha, excavada en la roca húmeda del puerto.

No estáis solos. En el rincón más oscuro hay un anciano encogido, Yenar Strig. Al principio parece un loco inofensivo, murmurando sobre "Ecos" y "Máscaras".

Sin embargo, durante las primeras horas de encierro, mientras esperáis vuestro destino, notáis que el anciano no está realmente con vosotros. Discute con el aire. Le susurra a sombras que solo él ve, debatiendo con voces que callaron hace décadas. Parece atrapado en una discusión eterna, repitiendo fragmentos de una guerra olvidada.
\end{readbox}

\begin{npcsummary}[El Vidente Borrado][images/yenar_strig.png]{Yenar Strig}
    \appearance{Anciano extremadamente delgado, cabello blanco apelmazado. Sus contornos parecen borrosos si no te fijas bien.}
    \voice{Ronca, baja, fragmentada. Alterna entre la lucidez y el pánico.}
    \motivation{Transmitir el mensaje de los Guardianes antes de que el Velo termine de digerirlo.}
    \secrets{Es el "eco" psíquico de los Tres Guardianes originales. Sabe lo que ellos sabían.}
\end{npcsummary}

\subsubsection*{Fase 1: La Convivencia (Lore)}

Durante este tiempo, Yenar establece los vínculos narrativos clave para el Acto III. Aprovecha para soltar estas frases mientras comparten el espacio:

\begin{itemize}
    \item \textbf{Sobre el Monje (El Gato):} \emph{"El hombre-gato siempre caía de pie. Decía que el suelo es una mentira si sabes cómo pisar. Pero el Silente le quitó el suelo."}
    \item \textbf{Sobre la Maga (La Tejedora):} \emph{"Ella leía libros que no estaban escritos. Ahora sus libros flotan en la oscuridad, esperando ojos que no sangren."}
    \item \textbf{Sobre el Druida (El del Viento):} \emph{"El que hablaba con las raíces... se hizo madera para no romperse. Pero la madera arde, oh sí, la madera arde."}
\end{itemize}

\subsubsection*{Fase 2: La Comida y el Olvido}

Un guardia desliza una bandeja con pan mohoso y agua con sabor metálico.

\begin{readbox}
El hambre os vence. Empezáis a comer en silencio. El sabor metálico del agua parece adormecer vuestra lengua y vuestros pensamientos.

El tiempo se dilata. Uno a uno, dejáis de mirar a vuestros compañeros. Os encerráis en vuestros propios traumas. El silencio de la celda crece hasta que el sonido de la respiración de los demás desaparece.

De pronto, te das cuenta: estás solo. Siempre has estado solo en esta celda. ¿Yenar? ¿Qué es Yenar? Ese nombre no significa nada. Solo es una mancha en la pared.
\end{readbox}

\subsubsection*{Fase 3: El Despertar del Velo}

El Velo os está digiriendo. Para recuperar la consciencia de la realidad y ver a Yenar una última vez, debéis romper el estupor.

\begin{mechanics}{Desafío: Romper la Disociación}
Cada jugador debe elegir un método para "despertar" antes de que sea tarde.

\textbf{Opción A: El Ancla del Dolor (Físico)}
El PJ se muerde la lengua, se clava las uñas o se golpea la cabeza contra la pared para que el dolor le traiga al presente.
\begin{itemize}
    \item \textbf{Coste:} Sufre \textbf{5 PV} de daño (sin sangrado).
    \item \textbf{Tirada:} \textbf{RR Física (Constitución) - Medio} o habilidad de \textbf{Aguante (Endurance)}.
    \item \emph{Éxito:} El dolor limpia la niebla. Ves a Yenar.
\end{itemize}

\textbf{Opción B: La Disciplina Mental (Meditación)}
El PJ se sienta y fuerza su mente a recitar un mantra, una oración o una fórmula lógica para expulsar la intrusión.
\begin{itemize}
    \item \textbf{Coste:} Requiere 2 asaltos de inacción total.
    \item \textbf{Tirada:} \textbf{Foco Mental (Mental Focus) - Difícil (-10).}
    \item \emph{Éxito:} Tu mente se ordena. Recuerdas quién eres y quién está contigo.
\end{itemize}

\textbf{Opción C: El Tótem Emocional (Memoria)}
El PJ se aferra a un objeto personal o un recuerdo doloroso de su pasado para anclar su identidad.
\begin{itemize}
    \item \textbf{Tirada:} \textbf{Foco Mental (Mental Focus) - Medio.} (Se usa esta habilidad para mantener la concentración en el objeto/recuerdo frente al borrado).
    \item \emph{Fallo:} El Velo corrompe el recuerdo. El PJ gana una \textbf{Fobia Menor} o \textbf{Culpa} temporal relacionada con ese recuerdo.
\end{itemize}
\end{mechanics}

\subsubsection*{Fase 4: La Última Profecía}

Si tienen éxito, Yenar vuelve a existir para ellos. El anciano os mira con urgencia desesperada. Sabe que el Velo ha intentado borrarle y que le queda poco tiempo.

Os agarra las manos. Su tacto es frío, como de papel seco.

\begin{displayquote}
\emph{"Escuchadme. El olvido viene a por mí. Guardad esto en vuestra carne si hace falta:"}

\begin{itemize}
    \item \textbf{El Mapa:} \emph{"Id al Norte. Cruzad los Pasos del Gigante. Allí el Silente es ciego."}
    \item \textbf{El Contacto:} \emph{"Buscad la Torre Rota en el valle de Valmor. Decidle al bibliotecario que \textbf{'la tinta está seca'}. Él sabrá qué hacer."}
    \item \textbf{La Clave (Acto III):} \emph{"Las ruinas no tienen cerradura. Escuchad bien: \textbf{'La puerta no está cerrada, vuestros ojos sí. Para entrar donde la luz no existe, debéis dejar de buscarla.'} Recordadlo."}
\end{itemize}
\end{displayquote}

\subsubsection*{Desenlace: El Glitch de la Realidad}

Apenas Yenar termina, se sienta en el rincón y su presencia se "apaga". Sigue ahí físicamente, pero vuestra mente resbala sobre él.

La puerta de la celda **explota** hacia adentro. Humo y polvo llenan la estancia.

Entra una mujer joven con armadura de cuero gastada: **Lira Doven**. Ignora a los PJs y va directa al rincón donde estaba el anciano. Pero sus ojos barren el espacio sin verlo.

\emph{"¡Maldición! ¡El registro decía Celda 4!"} —Grita Lira, girándose hacia vosotros con el arma desenfundada—. \emph{"¿Dónde está el anciano? ¿Dónde está Yenar?"}

\textbf{El Efecto del Velo:}
En ese momento, sin que podáis evitarlo, vuestras bocas se mueven solas, repitiendo la mentira que el Velo ha grabado en la realidad:

\begin{readbox}
\emph{"No hay nadie. Esta celda siempre estuvo vacía."}
\end{readbox}

Lira se queda helada. Os mira a los ojos, ve la niebla en vuestra mirada y luego el destello de terror cuando recuperáis el control de vuestras bocas. Comprende lo que ha pasado.

\emph{"El Velo ya ha reescrito la escena..."} —Murmura, bajando el arma pero agarrándoos del brazo con fuerza—. \emph{"Pero vosotros... vosotros tenéis la mirada del Eco. Si él ya no existe, entonces lo que sabía ahora está en vuestras cabezas. ¡Andando! ¡Si queréis vivir, seguidme!"}

\textbf{Continuar a la Escena III — El Asalto de la Resistencia.}

\end{multicols*}
