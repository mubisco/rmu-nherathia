\section{Escena VI --- El Sacrificio}

\begin{multicols*}{2}

\textbf{Tipo de escena:} Boss Fight (Combate de Jefe), Desafío de Tensión y Dilema Moral.

\begin{readaloud}
La escalera termina en una sala circular excavada en el corazón de la colina. El aire aquí abajo vibra con un zumbido grave que hace doler las muelas.

En el centro, el \textbf{Altar Negro} —un bloque de obsidiana de la Era de los Reyes— pulsa con luz violeta. Sobre él yace el chico, drogado y encadenado, con la piel pálida por el frío antinatural que emana de la piedra.

Kaelen está de pie junto al altar. No parece un lord poderoso; parece un hombre roto. Su túnica ceremonial está empapada de sudor y sus manos tiemblan, pero sostienen con firmeza una daga de cristal volcánico. Una barrera translúcida, alimentada por el Altar, lo rodea como una burbuja de cristal sucio.

Al veros, no grita con odio, sino con pánico absoluto.
\end{readaloud}

\subsubsection*{El Conflicto: La Aritmética de la Muerte}

Kaelen no ataca. Su objetivo es completar el ritual. Intentará convenceros mientras el "Reloj del Ritual" avanza.

\begin{displayquote}
\emph{"¡Imbéciles! ¡No tenéis ni idea de lo que retenemos aquí! ¡Este Altar es lo único que impide que los muertos de nuestros osarios se levanten! ¡Si detengo el cántico, no solo entrará el Velo... nuestros propios ancestros nos comerán vivos! ¡Atrás!"}
\end{displayquote}

\begin{mechanicbox}{Regla de Tamaño (RMU Creature Law)}
Normalmente, atacar piedra con armas de acero requeriría herramientas de minería y mucho tiempo (Character Law, Sec. 6.1). Sin embargo, el Altar está sobrecargado de energía. Los golpes no rompen la roca por fuerza bruta, sino que disrupcionan la integridad mágica que mantiene la piedra unida. Por eso usamos PV y AT en lugar de la regla de degradación por minutos.
Al ser el Altar un objetivo \textbf{Grande} atacado por armas \textbf{Medias}:
\begin{itemize}[nosep]
    \item \textbf{Reducción de Críticos:} La severidad de todos los críticos físicos recibidos se reduce en \textbf{-1 grado}.
    \begin{itemize}[nosep]
        \item Crítico 'A' $\rightarrow$ Solo Daño (sin crítico).
        \item Crítico 'B' $\rightarrow$ Crítico 'A'.
        \item Crítico 'E' $\rightarrow$ Crítico 'D'.
    \end{itemize}
    \item \textbf{Nota:} Esto hace que el Altar sea muy resistente a golpes superficiales, obligando a los jugadores a golpear fuerte para lograr efectos de rotura real.
\end{itemize}
\end{mechanicbox}

\subsubsection*{Mecánica del Encuentro: El Reloj del Ritual}

El combate tiene un límite estricto: el ritual finaliza al final del \textbf{3º Asalto}.

Kaelen está en trance (Concentración al 100\%). No ataca, no para y no se mueve. Su defensa es puramente mágica.

\begin{npcsheet}[Gobernador y Verdugo][images/kaelen_ritual.png]{Kaelen (En Trance)}
    \appearance{Rodeado de runas flotantes y una barrera traslúcida que vibra con el cántico.}
    \motivation{Completar el ritual. No romperá su concentración ni para defenderse.}
    \stats{
        \textbf{Nivel:} 6 (Layman) \\
        \textbf{PV:} 65 \\
        \textbf{AT:} 2 (Túnica Ritual) \\
        \textbf{DB Actual:} \textbf{75} (Mágica) \\
        \textbf{Acción:} Concentración total. Si es atacado cuerpo a cuerpo y fallan, el atacante recibe una \textbf{Descarga Eléctrica ('A' Electricity)} por el campo de fuerza.
    }
\end{npcsheet}

\paragraph{El Escudo del Pacto (Defensa Mágica)}
Kaelen está rodeado por un campo de fuerza visible.
\begin{itemize}[nosep]
    \item \textbf{Defensa Total (DB):} \textbf{+75} (Base 0 + 40 \emph{Greater Shield} + 25 \emph{Protection V} + 10 \emph{Blur}).
    \item \textbf{Inmunidad Táctica:} Kaelen es inmune al \emph{Aturdimiento (Stun)} mientras el Altar esté intacto (el dolor se transfiere a la piedra).
\end{itemize}

\paragraph{El Artefacto: El Altar Negro}
Tratado como un \textbf{Constructo Inmóvil de Tamaño Grande (Big)}.
\begin{itemize}[nosep]
    \item \textbf{Tamaño (Size):} \textbf{Grande (Big)}.
    \item \textbf{Tipo de Armadura (AT):} \textbf{10} (Piedra Sólida).
    \item \textbf{Defensa (DB):} \textbf{-20} (Inmóvil).
    \item \textbf{Integridad (PV):} \textbf{150 PV} (Base 100 x 1.5 por Tamaño Grande).
\end{itemize}

\paragraph{Estrategia para los Jugadores}
\begin{enumerate}
    \item \textbf{Atacar a Kaelen:} Muy difícil de impactar (+75 DB), pero frágil (65 PV). Si muere o cae inconsciente, el ritual para.
    \item \textbf{Atacar al Altar:} Impacto automático o muy fácil, pero absorbe mucho daño (150 PV y reduce críticos). Requiere daño masivo constante.
\end{enumerate}

\subsubsection*{Resolución del Clímax}

\paragraph{Desenlace A: El Ritual se Completa (Fracaso)}
Si pasan 3 asaltos sin detener a Kaelen:
\begin{itemize}
    \item Kaelen clava la daga. El chico muere. Una onda de energía violeta os lanza contra las paredes (\textbf{Golpe 'A' + 10 PV}).
    \item La barrera de la ciudad se estabiliza. Kaelen cae llorando sobre el cuerpo. Os ordena arrestar, pero está demasiado roto para luchar. El pueblo sobrevive, pero el coste moral es vuestro.
\end{itemize}

\paragraph{Desenlace B: El Altar se Rompe (Éxito Canon)}
Si Kaelen cae o el Altar llega a 0 PV:

\begin{readaloud}
Un chasquido agudo, como el de un hueso gigante al romperse, resuena en la sala. El Altar Negro se parte en dos. La luz violeta se apaga instantáneamente, sumiendo la cripta en la penumbra de vuestras antorchas.

Kaelen es lanzado hacia atrás por el contragolpe mágico. Se arrastra por el suelo, mirando los fragmentos del altar con terror absoluto.

\emph{"Habéis apagado la luz... Dioses, habéis despertado al cementerio."}
\end{readaloud}

\paragraph{El Equipo del Gobernador (Loot)}
Kaelen yace inconsciente o muerto. Su túnica ceremonial está arruinada, pero sus objetos de poder siguen intactos.

\begin{itemize}[nosep]
    \item \textbf{Anillo de Protección Menor:} Un anillo de hierro frío con una runa grabada. Otorga un bono de \textbf{+10 a la DB} constante al portador.
    \item \textbf{Llave Maestra:} Abre la caja fuerte del despacho y todas las puertas de la mansión.
\end{itemize}

\subsubsection*{El Secreto de la Cripta: La Tumba del Primer Vigía}

Mientras el polvo del Altar roto se asienta y la realidad empieza a temblar, algo llama vuestra atención en el fondo de la sala, oculto hasta ahora por las sombras del ritual.

El Altar Negro no era lo único aquí abajo. Empotrado en un nicho de la pared de roca viva, hay un sarcófago vertical de piedra mucho más antiguo que la mansión, anterior incluso a la fundación de Valmor. No tiene nombre, solo un relieve desgastado de un guerrero montando guardia eternamente.

La explosión mágica del Altar ha agrietado la tapa del sarcófago, revelando su interior. No hay cuerpo (se ha convertido en polvo hace siglos), pero sus manos esqueléticas aún aferran algo que el tiempo no ha podido tocar.

\begin{tcolorbox}[colback=blue!5, colframe=blue!40, title=Hallazgo: El Arma del Primer Vigía]
\imagerim{images/treasure_weapon.png}
Es un arma de diseño antiguo, forjada en un metal pálido y lechoso que parece brillar con luz propia en la oscuridad de la cripta.

\textbf{El GM debe elegir el tipo de arma que mejor encaje con el grupo (Espada Larga, Maza de Guerra, o Lanza).}

\textbf{Estadísticas (RMU):}
\begin{itemize}[nosep]
    \item \textbf{Material:} \textbf{Aleación Blanca (White Alloy)}. Una técnica de forja perdida de los Altos Hombres. Es increíblemente ligera y resistente.
    \item \textbf{Bonificador:} \textbf{+15 OB}. (Superior a las armas de acero normal +0 o acero alto +10).
    \item \textbf{Propiedades:}
    \begin{itemize}[nosep]
        \item \textbf{Toque Fantasma:} Al ser un arma de calidad Superior y antigua, es capaz de dañar a entidades incorpóreas (Sombras, Espectros) sin penalizador de daño.
        \item \textbf{Luz Tenue:} La hoja emite una luz pálida en un radio de 1 metro cuando se desenvaina, insuficiente para iluminar una sala, pero suficiente para ver dónde se golpea en la oscuridad absoluta.
    \end{itemize}
    \item \textbf{Valor:} Incalculable en el mercado actual, pero mecánicamente unos \textbf{400 mp}.
\end{itemize}
\end{tcolorbox}

\textbf{Narrativa:}
Al tomar el arma, sentís un equilibrio perfecto. No pesa; parece querer moverse. Es un arma forjada en una era donde la humanidad no se escondía de la niebla, sino que la combatía.

\subsubsection*{El Despertar de los Muertos}

El efecto inmediato no es visual, sino auditivo.
Desde algún lugar abajo en el pueblo, cerca de la torre de Erun, se escucha un sonido multitudinario: cientos de uñas arañando piedra desde el interior de los nichos.

Luego, el primer grito humano desde el exterior.

\begin{displayquote}
\emph{"¡Están dentro! ¡Salen de los muros! ¡Ayuda!"}
\end{displayquote}

La barrera ha caído. El Velo ha entrado, y los muertos de Valmor, conservados durante años, se han levantado todos a la vez.

\textbf{Transición:}
Erun (o Garel, si Erun no está) irrumpe en la escalera de la cripta, pálido como la cera.
\emph{"¡Corred! ¡El camino a las montañas es lo único que queda! ¡Todo lo demás es muerte!"}

\textbf{Comienza la Escena VII — Huida bajo el Velo.}

\end{multicols*}
