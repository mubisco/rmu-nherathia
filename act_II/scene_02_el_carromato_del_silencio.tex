\section{Escena II --- El Carromato del Silencio}

\begin{multicols*}{2}

\textbf{Tipo de escena:} Interacción social tensa, comercio (opcional) y gestión de expectativas.

\begin{readaloud}
El sol está alto, pero su luz llega al suelo filtrada por una gasa gris, incapaz de calentar el aire. Lleváis horas caminando por el sendero embarrado, con los nervios a flor de piel tras las noches anteriores.
\end{readaloud}

\subsubsection*{El Encuentro en el Camino}

La visibilidad es reducida debido a la niebla y la vegetación muerta. El GM debe pedir una tirada para detectar lo que se acerca antes de que sea visible.

\textbf{Tirada de Percepción (Oído) - Fácil (+20):}
\begin{itemize}
    \item \textbf{Fallo:} La niebla amortigua el sonido hasta que es demasiado tarde. De repente, una forma masiva surge de la bruma a menos de diez metros frente a vosotros. (Pasa directamente a \emph{"El Encuentro Directo"}).
    \item \textbf{Éxito:} Escucháis algo que no encaja con el silencio de Nherathia.
\end{itemize}

\begin{readaloud}
El crujido rítmico de ejes de madera mal engrasados y el resoplido pesado de bueyes. Algo grande se acerca por el camino en dirección contraria, oculto tras la curva y la niebla. Tardará unos instantes en llegar a vuestra posición.
\end{readaloud}

\paragraph{Un Asalto de Preparación}
Los jugadores tienen un momento para decidir qué hacer antes de que el vehículo sea visible:
\begin{enumerate}
    \item \textbf{Esperar visibles:} Se quedan en el camino, quizás con las manos en alto o armas enfundadas.
    \item \textbf{Preparar emboscada/Esconderse:} Se mueven a los matorrales o ruinas del borde del camino. Requiere una tirada de \textbf{Acechar (Stalking)}.
\end{enumerate}

\subsubsection*{Resolución: El Mercader en la Niebla}

Un carromato pesado aparece por la curva. Va cubierto con lonas de cuero encerado y reforzado con tablones gruesos, como si fuera un vehículo de asedio en miniatura.

\paragraph{A. Si los jugadores están visibles (o fallaron al esconderse)}
El conductor os ve al instante.

\begin{readaloud}
El conductor no lleva látigo, sino una ballesta cargada sobre las rodillas. Al veros, detiene el carromato de golpe con un silbido seco a los bueyes. No saluda. Sus ojos barren el bosque a vuestras espaldas, buscando qué os persigue, antes de clavar su mirada en vosotros. La ballesta apunta a vuestro pecho, firme.
\end{readaloud}

\paragraph{B. Si los jugadores se escondieron}
El GM debe hacer una tirada de \textbf{Percepción (+45)} para Garel contra la tirada de \textbf{Acechar} de los jugadores (Maniobra Enfrentada).

\begin{itemize}
    \item \textbf{Garel falla (Gana el Sigilo):} El carromato pasa de largo. El conductor mira al frente, tenso, sin notar vuestra presencia.
    \textit{(Si los jugadores no hacen nada, pierden el encuentro y la oportunidad de comercio. Si deciden salir ahora, Garel se asustará: el GM debe tirar Iniciativa, Garel podría disparar por reflejo).}

    \item \textbf{Garel tiene éxito (Detecta a alguien):} El carromato se detiene a unos metros. El conductor levanta una ballesta y apunta directamente hacia donde está escondido el PJ que peor ha tirado.
    \begin{displayquote}
    \emph{"Salid de ahí. Si fuerais Ecos, ya habríais aullado. Si sois bandidos, sabed que mis virotes están bañados en aceite de fuego. ¡Manos donde pueda verlas!"}
    \end{displayquote}
\end{itemize}

\subsubsection*{El Mercader: Garel de los Valles}

Una vez superada la tensión inicial, y si los PJ no atacan, Garel baja ligeramente el arma, aunque no destensa la cuerda.

\subsubsection*{Interacción}

Garel no quiere problemas, pero la curiosidad (y la posibilidad de noticias o comercio rápido) le puede. Mantendrá la ballesta a mano hasta que los PJ demuestren que no son bandidos ni "Ecos".

\paragraph{Diálogo y Pistas}
Si los jugadores interactúan, Garel les dará información clave, pero siempre con prisa. Si ve que están agotados (Fatiga acumulada), les hará un gesto hacia un barril pequeño en el lateral del carro.

\begin{npcsheet}[Mercader Itinerante][images/garel_mercader.png]{Garel}
\appearance{Hombre corpulento, barba canosa trenzada, piel curtida por la intemperie. Lleva amuletos de sal y hierro colgados del cuello.}
\voice{Baja y rasposa. Habla rápido, mirando al cielo para calcular cuánto queda de luz.}
\motivation{Llegar a Valmor antes de que caiga el sol. Proteger a su familia.}
\stats{
  \textbf{Nivel:} 3\\
  \textbf{Profesión:} Pícaro (Comerciante) \\
  \textbf{PV:} 55 \\
  \textbf{AT:} 9 (Cuero Rígido) \\
  \textbf{Habilidades:} Comercio +60, Percepción +45
}
\end{npcsheet}

\begin{itemize}
    \item \textbf{El Remedio del Viajero:} \emph{"Parecéis cadáveres que aún no saben que han muerto. Tengo algo para eso. Licor de Raíz Negra. Sabe a tierra, pero os devolverá el aliento. Lo necesitaréis."}
    \item \textbf{Sobre el Camino:} \emph{"¿Vais a pie? Locura. Las sombras se alargan temprano en esta época. Si no tenéis fuego de alquimista, no llegaréis al amanecer."}
    \item \textbf{Sobre Valmor:} \emph{"Es el único lugar seguro a tres días a la redonda. El Gobernador Kaelen mantiene las puertas cerradas y las luces encendidas."}
    \item \textbf{La Advertencia (El Click):} Si le preguntan por qué Valmor es seguro, Garel baja la voz.
    \begin{displayquote}
    \emph{"Porque allí pagan el precio. El Gobernador tiene un trato. No preguntéis cuál es. Solo agradeced que los muros no susurran."}
    \end{displayquote}
\end{itemize}

\subsubsection*{Comercio (Trading)}

Esta es la única oportunidad de los PJ para reabastecerse. Garel vende suministros básicos a precios inflados (la "tasa de riesgo").

\begin{mechanicbox}{Mecánica de Regateo (RMU - Treasure Law):}
Usa la habilidad de \textbf{Comercio (Trading)} de los PJ enfrentada a la de Garel (+60). Aplica el siguiente modificador a la tirada del jugador según cómo abordaron el encuentro:

\begin{itemize}
    \item \textbf{Visibles y tranquilos:} \textbf{+10} (Garel aprecia no haberse llevado un susto).
    \item \textbf{Ocultos y salieron voluntariamente:} \textbf{+0} (Neutral, cautela comprensible).
    \item \textbf{Descubiertos por Garel:} \textbf{-10} (Garel desconfía, cree que eran bandidos indecisos).
\end{itemize}

\textbf{Resolución (Maniobra de Porcentaje Enfrentada):}
\begin{itemize}[nosep]
    \item \textbf{Fallo Absoluto:} Garel se ofende, cierra la lona y azuza a los bueyes. Fin del encuentro.
    \item \textbf{Fallo:} Vende al \textbf{200\%} del precio base.
    \item \textbf{Éxito Parcial:} Vende al \textbf{150\%} del precio base.
    \item \textbf{Éxito:} Vende al precio normal (\textbf{100\%}).
    \item \textbf{Éxito Absoluto:} Vende con descuento (\textbf{80\%}) o regala un \emph{Arlan} extra.
\end{itemize}

\textbf{Inventario disponible (Precios Base):}
\begin{itemize}
    \item \textbf{Suministros:} Raciones de viaje (2 mp), Cuerdas (1 mp), Aceite (5 mp), Virotes/Flechas (2 mp/docena).
    \item \textbf{Licor de Raíz Negra (Poción de Alivio de Fatiga):} Recupera inmediatamente 30 Puntos de Fatiga (incluso los perdidos por el Velo). \textbf{Precio Base: 15 mp} (Escaso). Tiene 3 dosis.
    \item \textbf{Fuego de Alquimista (Frasco):} Un frasco de vidrio grueso con un líquido ámbar que brilla tenuemente y humea al abrirse. Se lanza como arma arrojadiza. Al impactar, ignita un área de 1.5m de radio. Causa un Crítico de Calor 'A' inmediato y prende materiales inflamables (incluso madera húmeda). \textbf{Precio Base: 15 mp} (Escaso). Tiene 3 frascos.
    \item \textbf{Hoja de Arlan (Hierba):} Se aplica sobre heridas. Cura \textbf{3-7 PV (1d5+2)}. \textbf{Precio Base: 2 mp}. Tiene 5 dosis.
    \item \textbf{Nódulo de Rewk (Hierba):} Se prepara en infusión. Cura \textbf{2-20 PV (2d10)}. \textbf{Precio Base: 8 mp}. Tiene 2 dosis.
\end{itemize}
\end{mechanicbox}

\subsubsection*{Cierre de la escena}

Garel mira al cielo con ansiedad. El sol empieza a tocar las copas de los árboles.

\begin{displayquote}
\emph{"Se hace tarde. Y cuando el sol cae, el camino ya no es de los vivos. Si queréis vivir, corred. Valmor está tras esa colina, seguid el humo."}
\end{displayquote}

El carromato arranca con un tirón brusco, dejándoos en el camino mientras las sombras empiezan a estirarse de nuevo.

Al coronar la colina que señaló Garel, veis vuestro destino.

\textbf{Comienza la Escena III — Llegada a Valmor.}

\end{multicols*}
