\section{Escena IV --- La Audiencia y el Sabio}

\begin{tcolorbox}[colback=red!10, colframe=red!70, title=Bifurcación Narrativa: El Estatus de los PJ]
El inicio de esta escena depende de las acciones de los jugadores en la Escena III (La Patrulla del Diezmo).

\textbf{Opción A: Observadores Pasivos (Flujo Normal)}
Si los PJ no intervinieron o no fueron detectados, pasaron la noche en la posada. Comienzan el día como visitantes tolerados. \textit{(Sigue leyendo en "El Desayuno Silencioso").}

\textbf{Opción B: Fugitivos (Si atacaron a la patrulla)}
Si los PJ atacaron a los guardias o impidieron la captura del chico, son buscados. No durmieron en la posada; se escondieron (quizás en el establo de Garel o en las ruinas).
\begin{itemize}
    \item \textbf{Salto de Escena:} Omite "El Desayuno" y "La Audiencia con Kaelen".
    \item \textbf{El Contacto:} Erun, el Sabio, los encuentra en su escondite antes de que lo haga la guardia. Pasa directamente a la sección \textbf{"El Encuentro con el Sabio"} con la siguiente modificación: Erun dice \emph{"Sois unos idiotas valientes. Habéis pateado el avispero, pero quizás seáis justo lo que necesito"}.
\end{itemize}
\end{tcolorbox}

\imagerim{images/mansion_kaelen.png}[Mansión del Gobernador Kaelen en Valmor.]

\begin{multicols*}{2}

\textbf{Tipo de escena:} Interacción social, exposición de lore y planificación del golpe.

\subsubsection*{El Desayuno Silencioso}

Amanece en Valmor. La luz del día revela lo que la noche ocultaba: el pueblo está limpio, las cercas reparadas y los niños juegan en la calle. Es la imagen perfecta de una vida rural pacífica, si ignoras la empalizada rúnica.

En la sala común de la posada, hay una silla vacía en la mesa de una familia local. La madre come gachas mecánicamente con los ojos rojos. El padre mira al vacío. Nadie pregunta dónde está su hijo mayor. Nadie mira la silla vacía.

\textbf{Garel}, sentado con vosotros, ha perdido su alegría comercial. Remueve su cerveza sin beberla.

\begin{displayquote}
\emph{"He visto ciudades asediadas con mejor ambiente que este. Aquí la gente no vive, solo espera su turno. Esa familia... les falta uno. Y el resto del pueblo actúa como si nunca hubiera existido. Mala señal, amigos. Mala señal."}
\end{displayquote}

Antes de que podáis responder, la puerta se abre. Dos guardias con tabardos granates entran. No desenvainan, pero se dirigen a vuestra mesa con autoridad, ignorando a Garel.

\begin{displayquote}
\emph{"El Gobernador Kaelen ha sido informado de vuestra llegada. Solicita vuestra presencia en la Mansión para el registro de forasteros. No es una invitación."}
\end{displayquote}

\subsubsection*{La Mansión del Gobernador}

Situada en la colina más alta, es un edificio de piedra sólida, austero y funcional. No hay lujos, solo mapas, informes de suministros y una atmósfera de trabajo constante. Huele a tinta, cera vieja y a decisiones difíciles.

\paragraph{El Encuentro con Kaelen}
El Gobernador os recibe de pie. No es un villano de risa malévola; es un hombre con ojeras profundas, hombros cargados y una mirada de acero cansado.

\begin{npcsheet}[Gobernador de Valmor][images/kaelen_gobernador.png]{Kaelen}
\appearance{Hombre de mediana edad, cabello gris prematuro, ropa noble pero práctica y sin joyas. Manos manchadas de tinta.}
\voice{Calmada, racional, persuasiva. Habla como un padre decepcionado que hace lo necesario.}
\motivation{Mantener la barrera mágica a cualquier coste. Cree firmemente que el fin justifica los medios.}
\stats{
    \textbf{Nivel:} 6 \\
    \textbf{Profesión:} Layman (Diplomático) \\
    \textbf{Habilidades:} Diplomacia +80, Liderazgo +70, Psicología +60, Administración +90
}
\secrets{Odia lo que hace, pero está convencido de que sin el sacrificio, todos morirían en una noche.}
\end{npcsheet}

\paragraph{El Discurso del Utilitarismo}
Kaelen os estudia. Sabe que sois capaces de sobrevivir fuera, lo cual os hace útiles o peligrosos.

\begin{itemize}
    \item \textbf{Sobre la Seguridad:} \emph{"Mirad por la ventana. ¿Veis a esos niños jugar? Fuera de estos muros, serían ceniza o Ecos. Aquí viven. Comen. Crecen. Yo les doy esa vida."}
    \item \textbf{Sobre el Precio (Si preguntan por el chico):} Kaelen no niega nada. Suspira.
    \emph{"La aritmética es cruel, pero necesaria. Uno muere para que quinientos vivan un mes más. ¿Es justo? No. ¿Es mejor que la extinción total? Sí. El Silente exige un diezmo, y yo se lo pago para que no tome la cosecha entera."}
    \item \textbf{La Oferta:} \emph{"Sois fuertes. Uníos a mi guardia. Ayudadme a mantener el orden y recolectar el diezmo cuando sea necesario. Tendréis comida, techo y un propósito."}
\end{itemize}

\textbf{Mecánica Social (Influencia/Debate):}
Los jugadores pueden debatir. Kaelen no cambiará de opinión (su convicción es absoluta), pero si tienen éxito en una tirada de \textbf{Diplomacia (Difícil)}, les dará un salvoconducto temporal de "Visitantes", permitiéndoles moverse libremente por el pueblo hasta el anochecer sin unirse a la guardia.

\subsubsection*{Interludio: El Mensajero de la Niebla}

Mientras evaluáis vuestra situación (ya sea saliendo de la Mansión o escondidos en las sombras), notáis que algo os observa. No es un guardia, sino un pájaro desaliñado posado en un alero bajo.

Es un cuervo enorme, con plumas que parecen hechas de humo grisáceo y ojos de una inteligencia inquietante. Es un \textbf{Mistraven}, una criatura semimagica nativa de estas tierras.

El cuervo, al ver que tenéis su atención, grazna con un sonido que imita una tos humana seca. Deja caer un objeto a vuestros pies y alza el vuelo hacia la zona baja del pueblo, graznando una sola palabra con voz rasposa:

\begin{displayquote}
\emph{"¡Tinta! ... ¡Tinta!"}
\end{displayquote}

\textbf{El Mensaje:}
El objeto es una página arrancada de un libro antiguo de historia, amarillenta y frágil. En los márgenes, alguien ha escrito con letra temblorosa y tinta fresca:

\begin{tcolorbox}[colback=yellow!10, colframe=black!50, title=Nota Arrugada]
\emph{"El Gobernador os ofrece cadenas de oro. Yo os ofrezco la verdad del acero.\\
No os fieis del silencio. Buscadme donde los muertos duermen, en la Torre Rota.\\
La puerta estará abierta al mediodía, cuando las sombras son más cortas.\\
Venid solos."}
\end{tcolorbox}

\subsubsection*{El Encuentro con el Sabio}

Siguiendo las instrucciones (o al cuervo, si sois fugitivos), os dirigís a la zona baja del pueblo al mediodía.

La niebla se aferra a la base de la empalizada. Allí, encajada entre el muro defensivo y las casas pobres, encontráis la torre en ruinas que visteis en el mapa o desde la colina.

La puerta de madera podrida está entreabierta. En la oscuridad del interior, el cuervo os observa desde una pila de libros.

Un anciano surge de las sombras y cierra la puerta rápidamente tras vosotros, echando un pesado cerrojo. Es \textbf{Erun}, el bibliotecario.

\begin{npcsheet}[Bibliotecario y Ex-Resistente][images/erun_sabio.png]{Erun el Bibliotecario}
    \appearance{Anciano encorvado, gafas de lentes gruesas y agrietadas, túnica manchada de polvo. El cuervo 'Tinta' se posa en su hombro.}
    \voice{Nerviosa, susurrante. Habla con la urgencia de quien ha guardado silencio demasiado tiempo.}
    \motivation{Redimir su cobardía del pasado ayudando a destruir lo que él mismo toleró.}
    \stats{
        \textbf{Nivel:} 6 \\
        \textbf{Profesión:} Scholar (Erudito) \\
        \textbf{Habilidades:} Lore: History (Nherathia) +90, Runes +80, Lore: Region (El Velo) +75, Lore: Creature (Entidades) +60.
    }
    \secrets{Fue un contacto de la Resistencia en Portimar hace décadas, pero huyó al interior por miedo cuando las purgas comenzaron.}
\end{npcsheet}

\paragraph{La Conexión con Portimar}
Erun acaricia al cuervo mientras os examina.

\begin{displayquote}
\emph{"Sabía que vendríais. O al menos, rezaba para que alguien viniera antes de que mi cobardía me matara."}
\end{displayquote}

Erun señala el mapa incompleto de los PJ (o saca uno propio si no lo muestran).

\begin{itemize}
    \item \textbf{El Eslabón Perdido:} \emph{"Hace tres años, un corredor de la Resistencia llegó a mi puerta. Venía de Portimar, herido y medio loco por el Velo. Buscaba las Ruinas del Norte, guiado por las profecías de un tal Yenar, un vidente prisionero. El corredor murió esa misma noche... o quizás Kaelen se aseguró de que no hablara. Pero me dejó sus notas."}
    \item \textbf{El Destino:} \emph{"Sé dónde están las ruinas. Puedo completar vuestro mapa con la ruta segura a través de las montañas."}
\end{itemize}

\paragraph{El Trato: Sangre por Tinta}
Erun pone una mano sobre el mapa, impidiendo que lo toméis todavía.

\begin{displayquote}
\emph{"Os daré la ruta. Pero tenéis que hacer algo que yo no tuve el valor de hacer. Kaelen sacrificará al chico esta noche, cuando la luna alcance el cenit. Tenéis que detenerlo."}
\end{displayquote}

\paragraph{La Verdad del Ritual y el Artefacto}
Si los jugadores preguntan sobre la barrera o Kaelen, Erun explica la naturaleza técnica del horror:

\emph{"Kaelen no es un mago, es un político. No tiene poder propio. Lo que usa es el \textbf{Altar Negro} bajo la mansión. Es un artefacto de los Primeros Habitantes, una batería que convierte vida en escudos. El ritual es solo la llave para encenderlo. Si destruís el Altar, el ritual se rompe para siempre."}

\textbf{Dilema Moral:}
Erun no miente sobre el coste. \emph{"Si el Altar se rompe, la barrera caerá. El Velo entrará. Habrá caos. Muchos morirán esta noche. Pero los que sobrevivan serán libres para huir, como vosotros. ¿Vale la vida de un inocente más que la seguridad de una prisión?"}

\subsubsection*{Preparación del Golpe}

Los jugadores deben decidir si aceptan. Si lo hacen, Erun les da información clave para la \textbf{Escena V}:
\begin{itemize}
    \item \textbf{Entrada:} \emph{"La puerta principal está vigilada. Pero la entrada de servicio de la cocina... la cerradura es vieja. O podéis probar por las ventanas del primer piso si sois ágiles."}
    \item \textbf{Horario:} \emph{"El ritual comienza a medianoche. Tenéis hasta entonces para prepararos."}
\end{itemize}

\textbf{Cierre de la escena:}
Cae la noche. Las luces de Valmor se apagan por orden del Gobernador. Solo la ventana de la mansión permanece encendida. El reloj de arena ha empezado a correr.

\textbf{Comienza la Escena V — La Sombra en la Mansión.}

\end{multicols*}

\imagerim{images/torre_erun.png}[Torre de Erun, el Sabio.]
