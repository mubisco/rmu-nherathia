\section{Escena IV --- La Audiencia y el Sabio}

\begin{multicols*}{2}

\textbf{Tipo de escena:} Interacción social, intriga política y revelación del objetivo.

\begin{readaloud}
Amanece en Valmor. La luz del día revela lo que la noche ocultaba: el pueblo está limpio, las cercas reparadas y los niños juegan en la calle. Es la imagen perfecta de una vida rural pacífica.

Excepto por un detalle.

En el desayuno de la posada, hay una silla vacía en la mesa de una familia local. La madre llora en silencio mientras come gachas mecánicamente. El padre mira al vacío. Nadie pregunta dónde está su hijo mayor. Nadie mira la silla vacía.
El "precio" se ha pagado, y el pueblo ha comprado un día más de sol.

Antes de que podáis terminar vuestra comida, dos guardias de tabardo granate entran en la posada. No desenvainan, pero se dirigen a vuestra mesa con autoridad.

\begin{displayquote}
\emph{"El Gobernador Kaelen solicita vuestra presencia en la Mansión. No es una invitación."}
\end{displayquote}
\end{readaloud}

\subsubsection*{Localización: La Mansión del Gobernador}

Situada en la colina más alta, es un edificio de piedra sólida, austero y funcional. No hay lujos, solo mapas, informes de suministros y una atmósfera de trabajo constante.

\paragraph{Sensaciones}
Olor a tinta y cera vieja; silencio de biblioteca; la sensación de estar en el despacho de un general bajo asedio.

\subsubsection*{El Encuentro con Kaelen}

El Gobernador os recibe de pie, revisando un libro de contabilidad. No es un villano de risa malévola; es un hombre con ojeras profundas, hombros cargados y una mirada de acero cansado.

\begin{npcsheet}[Gobernador de Valmor][images/kaelen_gobernador.png]{Kaelen}
\appearance{Hombre de mediana edad, cabello gris prematuro, ropa noble pero práctica y sin joyas. Manos manchadas de tinta.}
\voice{Calmada, racional, persuasiva. Habla como un padre decepcionado que hace lo necesario.}
\motivation{Mantener Valmor a salvo a cualquier coste. Cree firmemente que el fin justifica los medios.}
\stats{\textbf{Nivel:} 6 \quad \textbf{Profesión:} Layman (Diplomático) \\ \textbf{Habilidades:} Diplomacia +80, Liderazgo +70, Psicología +60}
\secrets{Odia lo que hace, pero está convencido de que sin el sacrificio, todos morirían en una noche.}
\end{npcsheet}

\paragraph{El Discurso del Utilitarismo}
Kaelen sabe quiénes sois (forasteros armados). No os amenaza con violencia, sino con la lógica.

\begin{itemize}
    \item \textbf{Sobre la Seguridad:} \emph{"Mirad por la ventana. ¿Veis a esos niños jugar? Fuera de estos muros, serían ceniza o Ecos. Aquí viven. Yo les doy esa vida."}
    \item \textbf{Sobre el Precio:} Si los PJ mencionan la silla vacía o los gritos.
    \emph{"La aritmética es cruel, pero necesaria. Uno muere para que cien vivan. ¿Es justo? No. ¿Es mejor que la alternativa? Sí. El Silente exige un diezmo, y yo se lo pago para que no tome la cosecha entera."}
    \item \textbf{La Oferta:} \emph{"Sois capaces. Lo veo en vuestra postura. Uníos a mi guardia. Ayudadme a mantener el orden. Tendréis comida, techo y, lo más importante en Nherathia: un propósito."}
\end{itemize}

\textbf{Mecánica Social (Influencia/Debate):}
Los jugadores pueden intentar debatir con él. Kaelen no cambiará de opinión (su convicción es absoluta), pero si tienen éxito, les dará un salvoconducto temporal ("Visitantes") en lugar de exigir una respuesta inmediata.

\subsubsection*{El Encuentro con el Sabio}

Al salir de la mansión (ya sea aceptando, rechazando o posponiendo la oferta), un anciano os hace señas desde la puerta de una torre en ruinas adyacente al cementerio local. Es \textbf{Erun}, el bibliotecario.

Os arrastra al interior de su torre, llena de libros mohosos y pergaminos rescatados de la humedad.

\begin{npcsheet}[Bibliotecario y Rebelde][images/erun_sabio.png]{Erun el Bibliotecario}
\appearance{Anciano encorvado, gafas de lentes gruesas y agrietadas, túnica manchada de polvo de libros.}
\voice{Nerviosa, susurrante, pero se vuelve firme cuando habla de historia o magia.}
\motivation{Romper el ciclo de sacrificios y recuperar el conocimiento perdido de los Primeros Habitantes.}
\stats{\textbf{Nivel:} 3 \quad \textbf{Profesión:} Scholar (Erudito) \\ \textbf{Habilidades:} Historia +75, Runas +60, Tradición del Velo +50. Inútil en combate.}
\end{npcsheet}

\paragraph{La Revelación del Mapa}
Erun ha visto el \textbf{Mapa Incompleto} (si los jugadores lo llevan visible) o simplemente intuye que buscan una salida.

\begin{displayquote}
\emph{"Kaelen cree que salva al pueblo, pero solo lo está engordando para el matadero. El Velo siempre tiene hambre, y cada vez pide más."}
\end{displayquote}

Erun pone sus manos temblorosas sobre el mapa de los jugadores (o saca uno propio si ellos no lo muestran).

\begin{itemize}
    \item \textbf{El Destino:} \emph{"Buscáis las Ruinas del Norte. Lo sé. Las leyendas dicen que allí el Silencio se rompe. Yenar... el loco de Portimar... él me escribió sobre esto hace años, antes de que lo atraparan."}
    \item \textbf{El Problema:} \emph{"Puedo completar vuestro mapa. Sé dónde están los pasos seguros en las montañas. Pero no puedo irme mientras Kaelen siga alimentando a la oscuridad."}
\end{itemize}

\paragraph{El Pacto de Erun}
El Sabio os hace una oferta directa. Él no es un guerrero, es un ratón de biblioteca. Os necesita a vosotros.

\begin{displayquote}
\emph{"Kaelen realiza el Ritual de la Protección esta noche. Usa un altar oculto bajo su mansión para canalizar la vida de un inocente hacia la niebla. Destruid el altar. Detened el sacrificio. Si lo hacéis, os daré la ruta segura a las Ruinas y provisiones para el viaje."}
\end{displayquote}

\subsubsection*{Dilema Moral}

Erun es honesto sobre las consecuencias:
\emph{"Si rompemos el pacto, el Velo atacará. Valmor perderá su protección mágica. Muchos morirán esta noche... pero serán libres. ¿Estáis dispuestos a pagar ese precio?"}

\textbf{Cierre de la escena}
Los jugadores tienen ahora un objetivo claro (el altar bajo la mansión) y un límite de tiempo (esta noche). Deben decidir si prepararse para el asalto, intentar convencer al pueblo (difícil) o buscar otra vía.

\textbf{Comienza la Escena V — El Ritual de la Noche.}

\end{multicols*}
