\section{Escena III --- La Luz tras la Empalizada}

\begin{multicols*}{2}

\textbf{Tipo de escena:} Interacción social, contraste ambiental y falsa seguridad.

\begin{readaloud}
Siguiendo el rastro de humo y las indicaciones de Garel, coronáis la última colina justo cuando el sol termina de ahogarse en el horizonte.
Abajo, en un valle poco profundo, se alza Valmor.

No es lo que esperabais.

En medio de la neblina gris de Nherathia, Valmor brilla como una brasa en la nieve. Una empalizada de troncos robustos, reforzada con bases de piedra, rodea el pueblo. Pero lo más llamativo no es la madera, sino lo que hay pintado en ella: cada cinco metros, un poste está marcado con glifos de cal blanca y ocre que parecen vibrar levemente contra la oscuridad que avanza.

No hay silencio aquí. Incluso desde la distancia, escucháis el ladrido de perros, el golpe de un martillo de herrero y voces humanas. Vida. Un insultante exceso de vida en medio de la tierra muerta.
\end{readaloud}

\subsubsection*{Localización: La Puerta Norte de Valmor}

Una estructura defensiva sólida. Dos torres de vigilancia flanquean una puerta de roble reforzado. Hay arqueros en las almenas, pero no miran hacia el bosque con miedo, sino con una disciplina militar rígida.

\paragraph{Sensaciones}
Olor a humo de leña y guiso especiado; la calidez que emana del interior del pueblo choca con el frío del exterior; sensación de ser observado, no por monstruos, sino por ojos humanos calculadores.

\subsubsection*{El Control de Entrada}

Al acercaros, una voz potente os da el alto desde la empalizada. Cuatro guardias con tabardos granates (el color del Gobernador) salen a vuestro encuentro, ballestas en mano pero apuntando al suelo.

El líder, un hombre con el rostro marcado por viruela, os examina. No busca armas, busca signos de "infección" del Velo (ojos negros, piel pálida, sombras extrañas).

\begin{npcsheet}[Capitán de la Guardia][images/capitan_valmor.png]{Capitán Harek}
\appearance{Rostro picado de viruela, armadura de cuero tachonado bien cuidada, capa granate.}
\voice{Profesional, seca, sin la superstición habitual de los campesinos.}
\motivation{Mantener la paz del Gobernador y asegurar que nadie traiga "problemas" de fuera.}
\stats{
  \textbf{Nivel:} 4 \\
  \textbf{Profesión:} Guerrero \\
  \textbf{PV:} 75 \\
  \textbf{AT:} 9 (Cuero Rígido) \\
  \textbf{Habilidades:} Percepción +50, Liderazgo +40, Espada Ancha +75
}
\end{npcsheet}

\paragraph{Interacción}
Harek es pragmático. Si los PJ parecen peligrosos pero cuerdos, son bienvenidos.
\begin{itemize}
    \item \textbf{El Peaje:} \emph{"La entrada cuesta 5 monedas de cobre por cabeza. El Gobernador Kaelen ofrece seguridad, y la seguridad se paga."}
    \item \textbf{Las Armas:} \emph{"Podéis conservar el acero. Aquí todos duermen con una daga bajo la almohada. Pero si desenfundáis sin causa, os colgaremos de la empalizada como advertencia para las sombras."}
    \item \textbf{La Advertencia:} \emph{"Buscad alojamiento en 'El Escudo'. Y un consejo: cuando suene la campana de medianoche, no estéis en la calle. Las patrullas no hacen preguntas a esa hora."}
\end{itemize}

\subsubsection*{El Interior: Una Normalidad Forzada}

Al cruzar el umbral, el cambio es físico. El aire es más cálido. Las calles están pavimentadas con adoquines (algo raro en esta región) y hay faroles de aceite en cada esquina.

La gente camina rápido, saluda en voz alta y ríe con fuerza.
\textbf{Tirada de Percepción / Intuición (Difícil):}
\begin{itemize}
    \item \textbf{Éxito:} Notas que la alegría es frenética, casi histérica. Beben demasiado rápido, ríen demasiado fuerte. Nadie mira hacia las ventanas oscuras de los pisos superiores. Es como si el silencio fuera el enemigo y trataran de llenarlo con ruido.
\end{itemize}

\subsubsection*{La Posada "El Escudo del Descanso"}

Un edificio de piedra y madera de dos plantas. La sala común está abarrotada, caliente y ruidosa. Hay comida real (estofado de venado, pan fresco) y cerveza.

\paragraph{Rumores en la Posada (Gather Information)}
Si los jugadores escuchan o preguntan (sin ser demasiado directos):
\begin{itemize}
    \item \emph{"La cosecha ha sido buena. Kaelen sabe cómo tratar con la tierra."} (Admiración con miedo).
    \item \emph{"Dicen que la niebla se tragó una granja entera en el este. Pobres diablos, no quisieron mudarse al pueblo."}
    \item \emph{"¿El Sabio? El viejo Erun vive en la torre vieja, junto al cementerio. Kaelen lo tolera porque sabe leer las estrellas, pero está loco. Dice que el precio es demasiado alto."} (Pista clave para la siguiente escena).
\end{itemize}

\subsubsection*{Cierre de la escena}

Tras acomodarse y quizás comer algo caliente por primera vez en días, el ambiente cambia.
Una campana solitaria tañe en la mansión del Gobernador, situada en la parte alta del pueblo.

El ruido de la posada muere al instante.
El posadero, un hombre gordo y sudoroso, apaga la mitad de las lámparas.

\begin{displayquote}
\emph{"Medianoche. A las habitaciones. El Gobernador va a... asegurar la noche."}
\end{displayquote}

Nadie discute. En segundos, la sala común se vacía. Desde vuestra ventana, veis patrullas de capas granates moviéndose por las calles vacías, llevando no armas, sino cadenas y varas de incienso negro.

\textbf{Comienza la Escena IV — La Audiencia y el Sabio.}

\end{multicols*}
