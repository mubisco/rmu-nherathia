\section{Escena V --- La Sombra en la Mansión}

\begin{tcolorbox}[colback=red!10, colframe=red!70, title=El Estado de Alerta]
El nivel de dificultad de esta infiltración depende de las acciones previas de los jugadores en el pueblo:

\textbf{Nivel de Alerta: Normal (Visitantes)}
Si no hubo combates en la calle y respetaron el toque de queda.
\begin{itemize}
    \item \textbf{Guardias Exteriores:} 2 en la puerta principal (aburridos), 1 patrullando el perímetro.
    \item \textbf{Ventaja:} Los PJ pueden usar habilidades sociales (\emph{Duping}) para acercarse antes de actuar.
\end{itemize}

\textbf{Nivel de Alerta: Alto (Fugitivos)}
Si atacaron a la patrulla del diezmo o fueron descubiertos por Garel/Guardias.
\begin{itemize}
    \item \textbf{Guardias Exteriores:} 4 en la puerta principal, 2 patrullando con \textbf{Mastines del Velo} (ver \emph{Creature Law}).
    \item \textbf{Ventaja Táctica:} Kaelen está desesperado acelerando el ritual, por lo que la seguridad \emph{interior} es menor (ha enviado a casi todos fuera y confía en sus runas para el sótano).
\end{itemize}
\end{tcolorbox}

\begin{multicols*}{2}

\textbf{Tipo de escena:} Infiltración, sigilo táctico, desactivación de trampas mágicas y tensión creciente.

\begin{readaloud}
La noche ha convertido a Valmor en una isla de piedra ahogada en un mar gris. Desde vuestra posición, miráis hacia arriba: la Mansión del Gobernador se alza solitaria en la cima del túmulo, separada del pueblo por una pendiente pronunciada de roca desnuda.

Abajo, el pueblo duerme un sueño intranquilo. Arriba, la mansión es una silueta negra contra el cielo sin estrellas, sólida como un búnker.

Tal y como dijo Erun, una única ventana en la planta baja emite una luz enfermiza y parpadeante. El resto del edificio está a oscuras. El viento trae jirones de niebla que se aferran a los muros de piedra, y con ellos, un zumbido grave que no se oye, se siente en los dientes: el Altar Negro está despierto.
\end{readaloud}

\subsubsection*{Fase 1: La Aproximación}

Subir la pendiente del túmulo sin ser vistos requiere superar el perímetro exterior. La mansión domina el terreno elevado, dando ventaja a los guardias.

\textbf{Opciones de los Jugadores:}

\paragraph{A. La Entrada de Servicio (El Consejo de Erun)}
Erun mencionó una puerta lateral usada para los suministros. Es la zona más oscura.
\begin{itemize}
    \item \textbf{Obstáculo:} 1 Guardia de la \emph{Guardia Granate} hace guardia estática.
    \item \textbf{Resolución (RMU):}
    \begin{itemize}
        \item \textbf{Sigilo:} Maniobra Enfrentada de \textbf{Acechar (Stalking)} de los PJ vs \textbf{Percepción} del guardia.
        \item \textbf{Combate Silencioso:} Si atacan, deben incapacitarlo en \textbf{1 asalto} antes de que pueda usar su acción para gritar o tocar el cuerno.
    \end{itemize}
\end{itemize}

\paragraph{B. Escalada por el Muro Norte (La Ruta Difícil)}
La parte trasera de la mansión da al vacío (la caída más pronunciada de la colina hacia el valle). No hay guardias, pero la roca es traicionera y húmeda.
\begin{itemize}
    \item \textbf{Mecánica:} Maniobra de \textbf{Escalar (Climbing) - Difícil (-10)}.
    \item \textbf{Riesgo:} Un fallo implica una caída de 10 metros (\textbf{Crítico de Impacto 'A'} + daño) y el ruido alerta a los perros si los hay.
\end{itemize}

\paragraph{C. Bluff Social (Solo si son Visitantes)}
Llamar a la puerta principal fingiendo tener información urgente.
\begin{itemize}
    \item \textbf{Mecánica:} \textbf{Engañar (Duping) - Muy Difícil (-20)}. Los guardias tienen órdenes estrictas de no molestar al Gobernador durante la "Vigilia", salvo emergencia extrema.
\end{itemize}

\subsubsection*{Fase 2: El Interior y la Runa}

Una vez dentro (presumiblemente por la cocina o un pasillo lateral), la casa está en silencio sepulcral. No hay sirvientes; Kaelen los despide durante el ritual.

\paragraph{Botín Opcional: El Despacho del Gobernador}
Si los jugadores dedican tiempo a registrar el despacho de Kaelen (requiere una tirada de Percepción o Buscar), encontrarán una pequeña caja fuerte detrás de un cuadro.
\begin{itemize}
    \item \textbf{Mecánica:} Abrir Cerraduras (Difícil -10) o romperla (Fuerza Muy Difícil -20, ruidoso). La llave la tiene Kaelen (ver Escena VI).
    \item \textbf{Contenido:}
    \begin{itemize}
        \item \textbf{La Recaudación:} Una bolsa con \textbf{150 monedas de plata} y \textbf{20 de oro} (impuestos del pueblo).
        \item \textbf{Documentos:} Cartas de Kaelen que prueban que buscaba desesperadamente una forma de romper el pacto sin matar al chico, justificando su rol de "villano a su pesar".
    \end{itemize}
\end{itemize}

La entrada al sótano no es secreta: es una puerta de roble reforzada con bandas de hierro negro, situada al final de la despensa.

\textbf{El Peligro Invisible:}
Gracias a Erun, sabéis que Kaelen no confía solo en el acero. Ha grabado un glifo de protección en la puerta.

\begin{mechanicbox}{Desactivando la Runa de Sangre}
\textbf{1. Detección Visual:}
Tirada de \textbf{Percepción (Vista) - Medio}.
\begin{itemize}
    \item \emph{Éxito:} Ven un trazo tenue brillando en la madera.
    \item \emph{Fallo:} Si tocan la puerta sin verla, la runa estalla.
\end{itemize}

\textbf{2. Identificación:}
Si la ven, una tirada de \textbf{Runas (Runes) - Fácil} confirma que es una runa de alarma explosiva. (Si Erun les avisó, el éxito es automático).

\textbf{3. Desactivación:}
Los PJ pueden intentar raspar el glifo para romper el circuito mágico.
\begin{itemize}
    \item \textbf{Habilidad Ideal:} \textbf{Runas (Runes) - Medio}. El personaje entiende el flujo mágico y lo corta.
    \item \textbf{Habilidad Alternativa:} \textbf{Trampas (Traps) - Muy Difícil (-20)}. Es un mecanismo mágico, difícil para un ladrón convencional.
    \item \textbf{Fallo:} La runa estalla. \textbf{Ataque de Electricidad 'A'} a quien toque la puerta y un ruido de trueno que alerta a Kaelen (+20 a su Iniciativa en la siguiente escena).
\end{itemize}
\end{mechanicbox}

\subsubsection*{Enemigos: La Guardia de Honor}

Si los PJ hicieron ruido al entrar o activaron la alarma, se encontrarán con la última línea de defensa en el pasillo antes de la puerta del sótano.

\begin{npcsheet}[Guardia de Élite][images/guardia_granate.png]{Guardia Granate (Nvl 4)}
    \appearance{Veteranos con cicatrices, armadura de \textbf{Brigandina} (AT 9) bien cuidada y capas de lana pesada.}
    \motivation{Fanáticos pragmáticos. Creen que protegen a sus propias familias al detener a los PJ.}
    \stats{
        \textbf{Nivel:} 4 \quad \textbf{PV:} 75 \\
        \textbf{AT:} 9 (Brigandina) \quad \textbf{DB:} 20 (Escudo) \\
        \textbf{OB:} +80 (Espada Ancha), +50 (Ballesta Ligera) \\
        \textbf{Habilidades:} Percepción +40, Disciplina +60.
    }
    \secrets{Si la moral se rompe, no huyen; se interponen en el camino para morir ganando tiempo.}
\end{npcsheet}

\textbf{Cantidad:}
\begin{itemize}
    \item \textbf{Sigilo Exitoso:} 0 enemigos (están patrullando fuera).
    \item \textbf{Alarma Activada:} 2 Guardias bloquean el pasillo con escudos levantados.
    \item \textbf{Refuerzos:} Si el combate dura más de 3 asaltos, llegan 1d3 guardias adicionales desde el exterior.
\end{itemize}

\subsubsection*{El Descenso a la Verdad}

Tras superar la puerta rúnica, una escalera de caracol tallada en la roca viva de la colina desciende hacia la oscuridad.

El aire se vuelve frío, mucho más frío que la noche exterior. Es un frío antiguo.
Desde abajo, llega una voz. No es el canto maníaco de un cultista, sino la voz cansada y metódica de un contable que suma pérdidas.

\begin{displayquote}
\emph{"...la sangre es la tinta. La vida es el papel. Escribo el nombre de la barrera sobre la piel del inocente. No por odio, sino por necesidad. Acepta el frío, chico. Aceptalo y serás el escudo..."}
\end{displayquote}

Al final de la escalera, una luz violeta pulsa rítmicamente. Estáis a tiempo, pero el ritual está en su fase final.

\textbf{Comienza la Escena VI — El Sacrificio.}

\end{multicols*}
