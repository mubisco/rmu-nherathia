\section{Escena I --- El Camino de los Susurros}

\begin{multicols*}{2}

\textbf{Tipo de escena:} Supervivencia, gestión de recursos (Fatiga) y horror atmosférico.

\begin{readaloud}
Portimar ha quedado atrás, tragada por la niebla costera. Ahora, ante vosotros se extienden las Tierras Cansadas: colinas grises y bosques de árboles raquíticos que parecen retorcerse de dolor.

Tenéis caballos y un mapa incompleto, pero ninguna certeza. El camino es apenas una cicatriz de barro entre la maleza. Durante el día, el silencio es absoluto; no hay pájaros, solo el viento moviendo las ramas secas. Pero lo peor no es el paisaje, es el peso del aire. Caminar aquí cuesta el doble. Respirar cansa. El Velo no solo oculta el sol; presiona contra vuestra voluntad.
\end{readaloud}

\subsubsection*{Regla Especial: La Fatiga del Velo}

El ambiente de Nherathia es hostil para la vida. Mientras los personajes estén fuera de un Santuario (como Valmor), se aplica la siguiente regla de la casa:

\begin{tcolorbox}[colback=gray!10, colframe=black!70, title=Desgaste Sobrenatural]
\textbf{Recuperación Bloqueada:} Los Puntos de Fatiga (PF) perdidos por eventos de viaje, frío, miedo o navegación fallida \textbf{no se recuperan} con el descanso breve habitual. Solo se recuperan durmiendo 8 horas completas en un lugar protegido o mediante magia de \emph{Alivio de Fatiga}.
\end{tcolorbox}

\subsubsection*{El Desafío del Viaje}

El viaje hasta el primer punto de descanso requiere superar la hostilidad del terreno. Cada día de viaje implica una tirada grupal de \textbf{Navegación / Supervivencia} para mantener la ruta correcta y evitar perder tiempo y recursos.

\textbf{Tirada de Navegación / Supervivencia (Medio +0):}
El líder del grupo hace la tirada. Otros pueden ayudar (+10).
\begin{itemize}
    \item \textbf{Éxito (101+):} Encuentran la ruta directa. El grupo pierde la fatiga normal de viaje.
    \item \textbf{Fallo Parcial o Fallo:} El camino desaparece en la niebla. El grupo vaga en círculos durante 4 horas antes de reorientarse.
    \textbf{Consecuencia:} Todos deben hacer una tirada de \textbf{Aguante (Endurance) - Medio}. Si fallan, reciben un penalizador de \textbf{-15 por Fatiga} adicionales (no recuperables).
\end{itemize}

\subsubsection*{El Desgaste: Las Noches Previas}

El viaje hasta Valmor no es un paseo, es una erosión. Durante las primeras jornadas (dos a caballo o cuatro a pie), el Velo no ataca con garras, sino con ambiente.

\paragraph{Recurso Escaso: La Búsqueda de Fuego}
El Velo humedece la madera y oculta las ramas secas, haciendo que encender una hoguera sea un lujo, no una garantía.
Cada noche, los personajes pueden intentar buscar leña suficiente para mantener un fuego protector.

\textbf{Tirada de Supervivencia / Búsqueda (Muy Difícil -20):}
\begin{itemize}
    \item \textbf{Éxito:} Logran encender un fuego fatuo pero cálido. Esto otorga un modificador de \textbf{-2 a la tirada de Eventos Nocturnos}.
    \item \textbf{Fallo:} La madera silba y humea sin prender, o simplemente no encuentran nada seco. Pasan la noche a oscuras.
\end{itemize}

\paragraph{Eventos Nocturnos (Tirar 1d10 cada noche previa)}
Aplica el modificador de -2 si lograron encender fuego.

\begin{itemize}
    \item \textbf{$\le$ 0 (El Consuelo de la Llama):} Gracias al fuego, la oscuridad no logra penetrar en el campamento. El grupo descansa sin incidentes sobrenaturales (aunque la Fatiga no se recupera por la regla del Velo).
    \item \textbf{1-4 (Frío Antinatural):} La temperatura cae bruscamente. Tirada de \textbf{RR Física (Constitución)} o sufrir -10 de penalización por fatiga (ver regla especial) hasta descansar 8 horas en un Santuario.
    \item \textbf{5-8 (Pesadillas del Olvido):} El personaje sueña que se desvanece. Tirada de \textbf{RR Mental (Autodisciplina)}. Fallo: -15 por Fatiga (ver regla especial) y olvida el nombre de un PNJ conocido.
    \item \textbf{9-10 (Silencio Absoluto):} Nadie puede hablar. La comunicación es imposible y el aislamiento mental es total. La paranoia aumenta (sin efecto mecánico, solo narrativo). \emph{Nota: Este resultado es imposible si hay fuego.}
\end{itemize}

\subsubsection*{Clímax: La Última Noche en las Ruinas}

Es la última noche. Según el mapa, Valmor debería estar tras las colinas que cruzaréis mañana. Estáis agotados, con la Fatiga acumulada de los días anteriores pesando en los músculos.

Os refugiáis en los muros derruidos de una antigua granja de piedra para cortar el viento helado. La estructura no tiene techo, pero ofrece una posición defendible.

Aquí, los jugadores deben tomar una decisión crítica sobre su supervivencia.

\paragraph{La Decisión: Fuego o Sombra}

\textbf{Opción A: Encender Fuego}
\emph{"El fuego calienta el cuerpo y el alma, pero en la oscuridad total de Nherathia, es un faro que grita: 'Aquí hay vida'."}
\begin{itemize}
    \item \textbf{Ventaja:} Los PJ recuperan calor (evitan penalizadores por frío en el combate) y tienen luz para combatir sin penalizadores.
    \item \textbf{Riesgo:} Las Sombras les detectan automáticamente. El combate es inevitable y pierden la sorpresa.
\end{itemize}

\textbf{Opción B: Permanecer en la Oscuridad}
\emph{"El frío muerde, pero la oscuridad os oculta... quizás."}
\begin{itemize}
    \item \textbf{Ventaja:} Posibilidad de evitar el combate o emboscar si tienen éxito en \textbf{Sigilo} grupal vs Percepción de las Sombras (+50 por oscuridad).
    \item \textbf{Riesgo:}
    1. \textbf{El Frío:} Antes del combate, todos hacen una RR Física. Fallo = 5 PV de daño y 10 Fatiga adicionales.
    2. \textbf{Combate a Ciegas:} Si las Sombras les detectan (tienen Visión Nocturna), los PJ lucharán con penalizadores severos por oscuridad (-50 a -100) y las Sombras tendrán bonos al ataque sorpresa.
\end{itemize}

\subsubsection*{Encuentro: Las Sombras del Velo}

Si encienden fuego (o si fallan el sigilo en la oscuridad), el Velo responde.
De repente, los caballos (si quedan) se agitan violentamente, rompiendo sus ataduras para huir hacia la negrura, donde sus relinchos se cortan con un sonido húmedo.

Luego, desde el límite de vuestra visión, las sombras se despegan de las paredes de la ruina. No son proyecciones; tienen volumen. Son siluetas de humo denso con cuencas vacías. Esta vez no quieren asustaros. Tienen hambre.

\begin{npcsheet}[No-Muerto Incorpóreo][images/sombra_velo.png]{Sombra del Velo (Nvl 2)}
    \appearance{Silueta humanoide de humo negro. Se mueven sin tocar el suelo.}
    \motivation{Apagar cualquier luz y consumir el calor de los vivos.}
    \stats{
        \textbf{Nivel:} 2 \quad \textbf{PV:} 40 \quad \textbf{AT:} 1 (Incorpóreo) \\
        \textbf{OB:} +50 (Toque helado) \quad \textbf{DB:} 30 (Por desenfoque) \\
        \textbf{Ataque Especial:} Si aciertan y hacen daño, el objetivo debe hacer una RR Física o recibir un penalizador por Fatiga de -5 por el frío innatural.
    }
    \secrets{Solo reciben daño completo de magia, fuego o armas mágicas (como la daga de Acero Estelar). Las armas normales hacen mitad de daño.}
\end{npcsheet}

\paragraph{Tácticas de la Manada}
El DJ debe usar \textbf{2 Sombras por cada PJ} para abrumarlos por número.
\begin{itemize}
    \item \textbf{Si hay fuego:} La mitad de las sombras gastan sus acciones en "sofocar" la hoguera con sus cuerpos incorpóreos. Si logran apagarla, todas ganan +20 OB y los PJ sufren penalizadores de visión.
    \item \textbf{Si no hay fuego:} Atacan desde el sigilo a los miembros más débiles o rezagados.
\end{itemize}

\subsubsection*{Resolución del Encuentro}

\paragraph{Opción A: Combatir y Vencer}
Si logran disipar a las sombras o resistir hasta que estas se cansen de presas difíciles (al reducir a la mitad sus números):
Las formas se disuelven en jirones de niebla con un lamento sordo. El fuego (si queda) vuelve a arder con fuerza. Han sobrevivido, pero el frío persiste.

\paragraph{Opción B: Huir hacia la Noche}
Si la situación es insostenible, los jugadores pueden optar por abandonar el campamento y correr a ciegas.

\begin{mechanicbox}{Mecánica de Huida (RMU)}
Huir del combate sin una retirada ordenada provoca un \textbf{Ataque de Oportunidad} de los enemigos cercanos (con bono de +35 por Espalda). Luego, deben superar el terreno.

\textbf{Tirada de Maniobra de Movimiento (Running) - Muy Difícil (-20):}
(Penalizador por oscuridad y terreno desconocido [1]).
\begin{itemize}
    \item \textbf{Éxito:} El personaje logra dejar atrás a las sombras, perdiéndose en la niebla.
    \item \textbf{Fallo:} El personaje tropieza en la oscuridad. Sufre un \textbf{Crítico de Desequilibrio (Unbalancing) 'A'} [2] y pierde 1d10 PV adicionales por golpes y rasguños.
    \item \textbf{Coste General:} Al huir, abandonan todo el equipo de acampada que no llevaran encima y sufren \textbf{20 Puntos de Fatiga} adicionales (no recuperables) por el esfuerzo agónico.
\end{itemize}
\end{mechanicbox}

\subsubsection*{Cierre de la escena: Un Amanecer Gris}

El combate termina, o la huida cesa, pero la adrenalina deja paso a un frío que cala los huesos.
El amanecer en Nherathia no trae esperanza, solo visibilidad. La niebla se levanta ligeramente, convirtiendo el mundo en una acuarela de grises y marrones muertos.

Os ponéis en marcha de nuevo, con el cuerpo dolorido y la sensación de que algo os observa desde la espesura. El mapa indica que seguís la ruta correcta, pero el camino parece estirarse, monótono y silencioso.

Hasta que, a media mañana, el silencio se rompe.
No es el aullido de un monstruo, ni el susurro del Velo. Es un sonido mundano, rítmico y pesado: madera crujiendo contra piedra y el resoplido de bestias de carga. Algo se acerca por el camino frente a vosotros.

\textbf{Comienza la Escena II — El Carromato del Silencio.}

\end{multicols*}
