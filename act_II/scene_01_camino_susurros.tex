\section{Escena I --- El Camino de los Susurros}

\begin{multicols*}{2}

\textbf{Tipo de escena:} Supervivencia, exploración, combate contra el entorno y horrores nocturnos.

\begin{readaloud}
Portimar ha quedado atrás, tragada por la niebla costera. Ahora, ante vosotros se extienden las Tierras Cansadas: colinas grises y bosques de árboles raquíticos que parecen retorcerse de dolor.

Tenéis caballos y un mapa incompleto, pero ninguna certeza. El camino es apenas una cicatriz de barro entre la maleza. Durante el día, el silencio es absoluto; no hay pájaros, solo el viento moviendo las ramas secas.

Pero es al caer el sol cuando entendéis por qué nadie viaja de noche en Nherathia. La niebla no desciende del cielo; brota del suelo como sudor frío. Las sombras de los árboles se alargan hacia vosotros, ignorando la posición de la luna.
\end{readaloud}

\subsubsection*{El Viaje: Reglas de Supervivencia}

El viaje hasta Valmor dura aproximadamente tres días a caballo (o cinco a pie si perdieron las monturas). Este trayecto no es un trámite; es una prueba de desgaste.

\paragraph{Navegación (Supervivencia / Orientación)}
El Velo altera la geografía sutilmente. Los caminos que aparecen en el mapa a veces terminan en barrancos que no deberían estar ahí.
\begin{itemize}[nosep]
    \item \textbf{Fallo:} El grupo pierde horas dando vueltas en círculos. Aumenta la Fatiga en 5 puntos.
    \item \textbf{Fallo Absoluto:} Se adentran en una zona de "Aire Muerto". Deben hacer una TR de Constitución o sufrir náuseas (-10 a la actividad) hasta que descansen.
\end{itemize}

\paragraph{La Presión del Silencio}
Cada noche que pasan a la intemperie, el Velo intenta erosionar su voluntad. Pide a cada jugador que describa un recuerdo feliz de su pasado. Luego, hazles tirar Resistencia (Voluntad/Presencia).
\begin{itemize}[nosep]
    \item \textbf{Fallo:} El personaje despierta recordando el evento, pero incapaz de recordar el rostro de las personas que estaban con él. El Velo ha empezado a comer. (-5 a Resistencia vs Magia Mental acumulable).
\end{itemize}

\subsubsection*{Encuentro: La Primera Noche}

Es la segunda noche de viaje. Han acampado en las ruinas de una antigua granja de piedra para protegerse del viento. Han encendido un fuego (necesario para el calor y la moral, pero peligroso porque atrae miradas).

\begin{readaloud}
El fuego chisporrotea, luchando contra una humedad antinatural. De repente, los caballos se agitan violentamente. Uno de ellos rompe su atadura y huye hacia la oscuridad, pero su galope se corta en seco con un chillido agónico que dura apenas un segundo.

Luego, silencio.

Desde el límite de la luz de vuestra hoguera, veis formas que se separan de la oscuridad. No tienen cuerpo, son siluetas hechas de humo denso y frío, con ojos que son pozos de vacío.
\end{readaloud}

\paragraph{Enemigos: Las Sombras Hambrientas}
Son \textbf{Ecos Menores} (basados en \emph{Shadows/Wraiths} de nivel bajo). No buscan carne, buscan calor y vida.
\begin{itemize}[nosep]
    \item \textbf{Táctica:} Intentan rodear el fuego. Si logran apagar la hoguera (acción de asalto), atacan con ventaja (+20 OB) en la oscuridad total.
    \item \textbf{Debilidad:} El fuego las mantiene a raya. La \textbf{Daga de Acero Estelar} (si la tienen) brilla con luz propia en su presencia y les causa daño crítico extra.
\end{itemize}

\begin{npcsheet}[No-Muerto Incorpóreo][images/sombra_velo.png]{Sombra del Velo (Nvl 2)}
\appearance{Silueta humanoide de humo negro, sin rasgos faciales salvo cuencas vacías.}
\motivation{Apagar cualquier luz y consumir el calor de los vivos.}
\stats{
    \textbf{Nivel:} 2 \\
    \textbf{PV:} 40 \\
    \textbf{AT:} 1 (Incorpóreo) \\
    \textbf{OB:} +40 (Toque helado) \\
    \textbf{DB:} 30 (Por desenfoque)
}
\secrets{Solo pueden ser dañadas por magia, fuego o armas mágicas/superiores. Las armas normales atraviesan el humo sin efecto.}
\end{npcsheet}

\subsubsection*{Resolución del Encuentro}

\paragraph{Si vencen:}
Las sombras se disipan con un lamento que suena como viento en una grieta. El fuego vuelve a arder con fuerza. Han sobrevivido, pero han aprendido la lección: la noche es territorio enemigo.

\paragraph{Si huyen:}
Deben abandonar el campamento y correr a ciegas. Tira en la tabla de \textbf{Maniobra de Movimiento (Difícil)} para ver si se hieren o pierden equipo en la huida. Llegarán a Valmor al amanecer, exhaustos y con penalizadores (-20), pero vivos.

\subsubsection*{Cierre de la escena: Un Amanecer Gris}

El combate termina, o la huida cesa, pero la adrenalina deja paso a un frío que cala los huesos.
El amanecer en Nherathia no trae esperanza, solo visibilidad. La niebla se levanta ligeramente, convirtiendo el mundo en una acuarela de grises y marrones muertos.

Os ponéis en marcha de nuevo, con el cuerpo dolorido y la sensación de que algo os observa desde la espesura. El mapa indica que seguís la ruta correcta, pero el camino parece estirarse, monótono y silencioso.

Hasta que, a media mañana, el silencio se rompe.
No es el aullido de un monstruo, ni el susurro del Velo. Es un sonido mundano, rítmico y pesado: madera crujiendo contra piedra y el resoplido de bestias de carga. Algo se acerca por el camino frente a vosotros.

\textbf{Comienza la Escena II — El Carromato del Silencio.}

\end{multicols*}
