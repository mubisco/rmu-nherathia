\section{Escena VII --- Huida bajo el Velo}

\begin{multicols*}{2}

\textbf{Tipo de escena:} Persecución, obstáculos ambientales, horror masivo y transición de capítulo.

\begin{readaloud}
El silencio que sigue a la ruptura del altar dura apenas un latido. Luego, el mundo grita.

No es solo el sonido de la madera de la empalizada estallando bajo una presión invisible. Es el sonido de la atmósfera misma cambiando. El aire se vuelve gélido y huele a tierra de cementerio recién cavada.

Desde la entrada de la cripta, en la cima de la colina, veis el horror: la niebla del Velo no entra flotando; se derrama sobre el pueblo como una inundación. Llena el valle desde abajo hacia arriba.

Las luces de las casas en la ladera parpadean y se extinguen una a una, engullidas por la marea gris. Y con la oscuridad, llegan los alaridos. El Osario, ubicado en la zona baja, ha sido el primero en caer... y sus habitantes están subiendo la cuesta.
\end{readaloud}

\subsubsection*{El Último Acto de Erun}

Erun, que os esperaba nervioso en el umbral de la cripta, os mira con una mezcla de horror y determinación. La niebla ya lame los escalones inferiores. Saca un tubo de cuero y un fardo de tela pesada de su túnica y os los tiende con urgencia.

\begin{displayquote}
\emph{"¡Tomadlo! He marcado la ruta. Los Pasos del Gigante... al norte. Es el único camino que el Silente no vigila. Y llevad esto; lo guardaba para mi propia huida, pero mis piernas ya no aguantarán el viaje."}
\end{displayquote}

\textbf{El Legado del Erudito (Loot):}
\begin{itemize}
    \item \textbf{Mapa de las Colinas del Norte:} Muestra la ruta hacia las Ruinas Primigenias (Capítulo III).
    \item \textbf{3x Dosis de Hoja de Arlan (Hierba):} Preparadas en cataplasma. Al aplicarlas, detienen el sangrado inmediatamente y curan \textbf{1d5+2 PV} (Ver \emph{Treasure Law} [2]).
    \item \textbf{2x Bayas de Ankii (Hierba):} Unas bayas pequeñas y secas de color púrpura oscuro. Al comer una, el personaje recibe los beneficios físicos de una noche completa de sueño (elimina toda la \textbf{Fatiga} y recupera Power Points, aunque no cura heridas a largo plazo). Sabor muy amargo. (Ver \emph{Treasure Law} [1]).
    \item \textbf{1x Varita de Luz Estelar (Wand of Sudden Light):} Una vara corta de madera blanca. Tiene \textbf{5 cargas}. Al romper una muesca, lanza un destello de luz pura (Rango 30m).
    \begin{itemize}
        \item \emph{Efecto:} Ciega a enemigos vivos 1 asalto. Afecta a \textbf{Sombras} y No-muertos sensibles a la luz como un ataque de \textbf{Bola de Luz (Light Law)} con OB +20.
    \end{itemize}
\end{itemize}

Si los jugadores intentan salvarlo, Erun se niega, retrocediendo hacia la biblioteca de la mansión.

\begin{displayquote}
\emph{"Soy viejo y lento. Os mataría a todos. Además... alguien tiene que intentar salvar los libros antes de que la humedad lo pudra todo. ¡Corred! ¡Bajad antes de que la marea cubra la salida!"}
\end{displayquote}

\subsubsection*{El Descenso: Corriendo hacia la Muerte}

Para salir de Valmor, debéis bajar desde la Cima (Mansión) hasta la Base (Empalizada), atravesando el pueblo que se está convirtiendo en una zona de guerra.

\textbf{Mecánica: El Descenso en 3 Etapas}
Cada etapa requiere una tirada grupal o individual. Si fallan, sufren daño o retrasos (Fatiga).

\paragraph{Etapa 1: La Ladera Alta (El Pánico)}
Las calles estrechas están bloqueadas por aldeanos aterrorizados que intentan subir a la mansión buscando seguridad.
\begin{itemize}
    \item \textbf{Tirada:} \textbf{Maniobra de Fuerza (Brawn) - Medio} para abrirse paso, o \textbf{Intimidación - Difícil} para que se aparten.
    \item \textbf{Fallo:} Quedan atrapados en la masa. 2 asaltos de retraso y \textbf{10 Puntos de Fatiga} por el forcejeo.
\end{itemize}

\paragraph{Etapa 2: La Zona Media (La Niebla)}
La niebla os alcanza. La visibilidad se reduce a 1 metro. El suelo adoquinado está resbaladizo por el hielo repentino.
\begin{itemize}
    \item \textbf{Tirada:} \textbf{Acrobacias / Atletismo - Difícil (-10)}.
    \item \textbf{Fallo:} Caída aparatosa. \textbf{Crítico de Impacto 'A'} y el personaje pierde su arma principal en la niebla (requiere 1 asalto recuperarla).
\end{itemize}

\paragraph{Etapa 3: La Base (El Choque)}
Cerca de la salida, os cruzáis con la vanguardia de los muertos que suben desde el cementerio de la torre.
\begin{itemize}
    \item \textbf{Encuentro:} 1d3 \textbf{Penitentes} (ver Escena I del Cap. I) o Esqueletos por cada PJ.
    \item \textbf{Resolución:} No es necesario matarlos a todos. Basta con una maniobra de \textbf{Evasión} exitosa o abrir un hueco a golpes para seguir corriendo.
\end{itemize}

\subsubsection*{El Guardián de la Salida}

Justo antes de abandonar el perímetro roto de Valmor, donde la empalizada ha caído, algo bloquea el camino.

El Capitán Harek y sus guardias están allí. Pero ya no defienden el pueblo. Están muertos, de pie, con la piel gris y los ojos llenos de humo negro. Detrás de ellos, una forma más grande, hecha de oscuridad sólida, flota sobre el suelo.

\textbf{Opciones de los Jugadores:}
\begin{enumerate}
    \item \textbf{Luchar:} Es un enemigo formidable para un grupo agotado. Si la vencen, el camino queda despejado.
    \item \textbf{Distracción:} Usar el \emph{Fuego de Alquimista} (si lo compraron a Garel) o magia de luz intensa. La Sombra Mayor retrocede ante luz fuerte (aturdida 1 asalto), permitiendo huir.
    \item \textbf{Sacrificio:} Si un PJ o PNJ se queda atrás para contenerla, los demás escapan automáticamente.
\end{enumerate}

\begin{npcsheet}[Avatar Menor del Velo][images/sombra_mayor.png]{La Sombra Mayor (Greater Shadow)}
    \appearance{Una figura alta, encapuchada, sin rostro. Su presencia baja la temperatura hasta congelar el aliento.}
    \stats{
        \textbf{Nivel:} 7 \quad \textbf{PV:} 90 \quad \textbf{AT:} 1 (Incorpóreo) \\
        \textbf{DB:} 50 (Desenfoque) \\
        \textbf{OB:} +70 (Toque de Vacío: Daño de Frío + Drenaje de Constitución)
    }
    \motivation{Impedir que los "Rompedores del Pacto" escapen. El Velo exige retribución.}
\end{npcsheet}

\subsubsection*{Cierre del Capítulo}

Lográis cruzar la línea. Corréis hacia el bosque y las colinas, con los pulmones ardiendo y el frío mordiendo vuestros talones.

Cuando alcanzáis la primera cresta, os detenéis un instante para mirar atrás.

Valmor ya no existe.

Donde antes había una aldea con luces cálidas, ahora solo hay un lago de niebla agitada y silenciosa que ha llenado el valle hasta el borde. La única luz es la luna pálida reflejada en ese mar de nubes. El Velo ha reclamado su deuda.

Miráis el mapa de Erun. El camino al norte es duro, a través de montañas donde se dice que duermen cosas peores que las sombras. Pero es vuestra única opción.

La \textbf{brújula de Orim} gira loca por un momento y luego se detiene, firme, señalando hacia los picos nevados.

El Pacto se ha roto. La caza ha comenzado.

\textbf{FIN DEL CAPÍTULO II}

\end{multicols*}
